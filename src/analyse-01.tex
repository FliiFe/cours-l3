\ifsolo
    ~

    \vspace{1cm}

    \begin{center}
        \textbf{\LARGE Fonctions holomorphes} \\[1em]
    \end{center}
    \tableofcontents
\else
    \chapter{Fonctions holomorphes}

    \minitoc
\fi
\thispagestyle{empty}

% prof: emmanuel.peyre@univ-grenoble.alpes.fr
% site du cours: https://etudes.ens-lyon.fr/course/view.php?id=4387


\section{Définition}

\begin{dfn}[Notation]
    On munit $\C$ le corps des complexe de sa base $(1, i)$ et de la structure euclidienne donnée par  $|\;\cdot\;|$
\end{dfn}

\begin{dfn}
    Soit $U$ un ouvert de $\C$, $a \in  U$, et $f:U\longrightarrow \C$. On dit que $f$ est holomorphe\index{holomorphe} en  $a$ (ou encore, dérivable au sens complexe)  si et seulement si \[
        \frac{f(z)-f(a)}{z-a}
    \] 
    admet une limite quand $z\to a$. Cette limite sera appelée dérivée de $f$ en  $a$, notée  $f'(a)$.
\end{dfn}

\begin{rem}
    \begin{enumerate}[label=(\arabic*)]
    \item $g:U\longrightarrow \C, a \in \bar U$. On dit que $g(z)\xrightarrow[z\to a]{}\ell$ si \[\forall  \epsilon >0, \exists  \eta >0, \forall  z \in  U, |z-a|\leq \eta \implies |g(z)-\ell|<\epsilon\]
    \item Si $f$ est holomorphe en  $a$, alors elle est continue en  $a$
\end{enumerate}
\end{rem}

\begin{prop}
    Soit $U$ un ouvert de  $ \C$, $f:U \longrightarrow \C$, $a \in  U$ et $\lambda \in  \C$. Les assertions suivantes sont équivalentes \begin{enumerate}[label=(\emph{\roman*})]
    \item $f$ est holomorphe en  $a$, de dérivée  $\lambda$ en $a$
    \item $f(z)=f(a)+\lambda (z-a)+o(|z-a|)$
    \item L'application  $f$ est différentiable en  $a$ et  $\diff f_a$ est la multiplication par  $\lambda$ 
    \item $u=\Re(f), v=\Im(f)$ sont différentiables en  $a$ et vérifient les conditions de \textsc{Cauchy-Riemann}\index{Cauchy-Riemann}  \[
            \frac{\partial u}{\partial x}(a)= \frac{\partial v}{\partial y}(a), \qquad  \frac{\partial u}{\partial y}(a)=-\frac{\partial v}{\partial x}(a), \qquad \text{ et } \frac{\partial u}{\partial x}(a)+i \frac{\partial v}{\partial x}(a)=\lambda
    \] 
\end{enumerate}
\end{prop}

\begin{proof}
    $(i \iff   ii)$ \begin{align*}
        \frac{f(z)-f(a)}{z-a}\xrightarrow[z\to a]{} \lambda &\iff \frac{f(z)-f(a)}{z-a}=\lambda+o(1) \\
                                                            & \iff  f(z)=f(a)+ \lambda(z-a)+o(|z-a|)
    \end{align*}

    $(ii \iff  iii)$ définition de la différentiabilité

    $(iii \iff  iv)$ On décompose $z=x+iy, \lambda=\alpha+i\beta$. La jacobienne est donnée par \[
        \mathcal  M(\diff f_a)=
        \begin{pmatrix}
            \displaystyle \frac{\partial u}{\partial x}(a) & \displaystyle \frac{\partial u}{\partial y}(a) \\
            \displaystyle \frac{\partial v}{\partial x}(a) & \displaystyle \frac{\partial v}{\partial y}(a) 
        \end{pmatrix}
    \] et \[
    \mathcal  M(h\longmapsto \lambda h) =
    \begin{pmatrix}
        \alpha & -\beta \\
        \beta & \alpha
    \end{pmatrix}
    \]
\end{proof}

\begin{dfn}
    Soit $U$ un ouvert de  $\C$ et $f:U\longrightarrow \C$. On dit que $f$ est holomorphe sur  $U$\index{holomorphe}  si et seulement si $f$ est holomorphe en tout point de  $U$. Dans ce cas, $U\longrightarrow \C, a\longmapsto f'(a)$ s'appelle application dérivée de $f$\index{dérivée} 

    On note $\mathcal  H(U)$ l'ensemble des application holomorphes sur $U$
\end{dfn}

\section{Exemples}

$U$ désigne un ouvert de $\C$.

\begin{itemize}
    \item Une fonction constante est holomorphe de dérivée nulle
    \item $z\longmapsto z$ est holomorphe de dérivée $-1$
    \item  $z\longmapsto \bar z$ n'est pas holomorphe
\end{itemize}

\begin{prop}
    Si $f \in  \mathcal  H(U), g \in \mathcal  H(u)$, \begin{enumerate}[label=(\alph*)]
        \item $f+g \in  \mathcal H(U)$ et $(f+g)'=f'+g'$
        \item  $fg \in  \mathcal H(U)$ et $(fg)'=f'g+fg'$
        \item  Si $f^{-1}( \left\{  0 \right\}  )=\emptyset$ alors $\sfrac1f \in  \mathcal  H(U)$ et \[
         \left( \frac{1}{f} \right)'=-\frac{f'}{f^2 }
        \] 
    \item Si  $P=a_0+\cdots +a_dX^d \in  \C[X]$ alors $P \in  \mathcal  H(\C)$ et \[
            P':z\longmapsto  \sum_{k=1}^{d} ka_kX^{k-1}
    \]
    \item Si $Q \neq 0$ alors $\sfrac PQ \in  \mathcal  H(\C \setminus Q^{-1}(\left\{ 0 \right\} ) )$
    \end{enumerate}
\end{prop}

\begin{rem}
    Sur $U \setminus  v^{-1}(\left\{ 0 \right\} )$,
\[
    \left( \frac{u}{v} \right)'=\frac{u'v-uv'}{v^2 }
\] 
\end{rem}

\begin{prop}
    Soient $U, V$ des ouverts de  $\C$, $f \in  \mathcal  H(u), g \in  \mathcal  H(V)$. On suppose $f(U) \subset V$. Alors $g\circ f$,  \[
        (g\circ f)'=f'\times (g'\circ f)
    \] 
\end{prop}

\section{Inégalité des accroissements finis}

\begin{prop}
    Soit $U$ un ouvert de  $\C$ et $f \in  \mathcal  H(U)$. Soit $I\subset \R$ un intervalle et $\gamma:I\longrightarrow U$ dérivable. Alors \[
        (f\circ \gamma)'=\gamma'\times f'(\gamma)
    \] 
\end{prop}

\begin{proof}
Différentielle d'une composée: \[
    \diff_t(f\circ \gamma)= \diff_{\gamma(t)}f\circ \diff_t \gamma
\] 
et \[
    \diff_{\gamma(t)}f: z \in  C \longmapsto  f'(\gamma(t))z\in C, \qquad \qquad \diff_t \gamma: u \in  \R \longmapsto u \gamma'(t) \in  \C
\] 
\end{proof}

\begin{thm}[Inégalité des accroissements finis\index{inégalité des accroissements finis}]
    Soit $U$ un ouvert de  $\C$ et $f \in  \mathcal  H(U)$. Soient $a, b \in  U$ tels que le segment $[a, b]\subset U$. S'il existe  $M \in  \R$ tel que $\forall  z in [a, b], |f'(z)|<M$ alors \[
        |f(b)-f(a)|\leq M|b-a|
    \] 
\end{thm}

\begin{proof}
On considère l'application  \[
\begin{array}{rrcl}
    \gamma:& [0,1] & \longrightarrow & [a,b] \\
           & t & \longmapsto & \displaystyle (1-t)a+tb
\end{array}
\]
Cette fonction est dérivable de dérivée $b-a$. Dans ce cas,  $f\circ \gamma$ est dérivable de dérivée $(b-a)f'\circ \gamma$. Puis, $\forall t \in  [0,1], |(f\circ \gamma)'(t)|\leq |b-a|M$, et on applique le résultat pour les fonctions à valeurs vectorielles (ci-dessous)
\end{proof}

\begin{prop}[Inégalité des acroissements finis pour les fonctions à valeurs vectorielles]
    Soient $a, b \in  \R, a<b$, $E$ un e.v.n,  $f:[a, b] \to  E$, $g:[a, b] \to \R$ continues dérivables à droite et telles que \[
        \forall  t \in  ]a, b[, \|f_d'(t)\|\leq g_d'(t)
    \]
Alors, \[
    \|f(b)-f(a)\|_E\leq g(b)-g(a)
\] 
\end{prop}

\begin{proof}
Soit $\epsilon >0$, et \[
\begin{array}{rrcl}
    \varphi_\epsilon:& [a,b] & \longrightarrow & \R \\
               & t & \longmapsto & \displaystyle \|f(t)-f(a)\|_E-(g(t)-g(a))-\epsilon(t-a)
\end{array}
\] 
Soit $E_\epsilon=\{t \in  [a, b], \varphi_\epsilon(t)\leq \epsilon\}$ fermé de $[a, b]$. Il existe  $\eta>0$ tel que  $[a, a+\eta]\subset E_\epsilon$. On note  $c=\max E_\epsilon>a$. Supposons par l'absurde que  $c<b$. Par hypothèse,  \[
    \exists  t \in  ]c, d[, \left\|\frac{f(t)-f(c)}{t-c}-f_d'(c)\right\|\leq \frac{\epsilon}{2}\quad \text{ et } \quad  \left| \frac{g(t)-g(c)}{t-c}-g_d'(c) \right|\leq \frac{\epsilon}{2}
\] 
et $\|f_d'(c)\|\leq g_d'(c)$. Alors 
\[
    \|f(t)-f(c)\|\leq \left( \|f_d'(c)\|+\frac{\epsilon}{2} \right) (t-c)\leq \left( g_d'(c)+ \frac{\epsilon}{2} \right)(t-c)\leq g(t)-g(c)+\epsilon{t-c}
\] 
Comme $c \in  E_\epsilon$ \[
    \|f(c)-f(a)\|\leq  g(c)-g(a)+\epsilon(c-a)+\epsilon
\] 
donc \[
    \|f(t)-f(a)\|\leq g(t)-g(a)+\epsilon(t-a)+\epsilon
\] 
donc $t \in  E_\epsilon$ absurde par définition de $c$ donc $c=b$.
\end{proof}

\begin{cor}
    Soit $U$ un ouvert de  $\C$, $f \in  \mathcal  H(U)$ telle que $f'$ est l'application constante nulle alors  $f$ est \textbf{localement constante}\index{localement constante (application)} (constante sur un voisinage de tout point)
\end{cor}

\begin{rem}
    Une fonction localement constante sur un ouvert connexe est constante. Pour le voir: on se fixe $a \in  U$ et $f^{-1}(\{f(a)\})$ est un ouvert fermé non vide de $U$ donc vaut  $U$.
\end{rem}

\section{Théorème d'inversion locale}

\subsection{Énoncé du théorème}

\begin{thm}[Inversion locale\index{inversion locale (théorème)}]
    Soit $U$ un ouvert de  $\C$ et $f \in  \mathcal  H(U)$. Soit $a \in  U$. On suppose \begin{enumerate}[label=(\emph{\roman*})]
    \item $f'(a)\neq 0$
    \item $f'$ continue
\end{enumerate}
Alors, il existe un ouvert $V\subset U$ qui contient $a$ et tel que l'application  $f\left|_{V}\right.:V\longrightarrow f(V)$ est bijective et $f^{-1}:f(V)\longrightarrow V$ est holomorphe. En outre, \[
    \forall  b \in  f(V), (f^{-1})'(b)=\frac1{f'(f^{-1}(b))}
\] 
\end{thm}

\begin{rem}
    On verra plus tard que si $f \in  \mathcal  H(U)$, $f'$ est continue (la seconde condition est automatique).
\end{rem}

\subsection{Complément pour les espaces vectoriels normés}

\begin{dfn}
    Un evn $E$ est dit complet\index{espace complet} si toute suite de Cauchy converge.
\end{dfn}

\begin{dfn}
    Une application $f:X\subset F \longrightarrow E$ est  dite  $k$-contractante pour  $k \in  ]0, 1[$ si elle est $k$-lipschitzienne.
\end{dfn}

\begin{thm}[Théorème du point fixe\index{théorème du point fixe}]
    Soit $k \in  ]0,1[$, soit $E$ un evn complet, soit  $F$ un fermé non vide de  $E$.  \begin{enumerate}[label=(\alph*)]
        \item Si $f:F\longrightarrow F$ est $k$-contractante alors  $f$ admet un unique point fixe. Ce point fixe est limite de  $(x_n)_n$ où  $x_{n+1}=f(x_n)$ et $x_0 \in  F$ arbitraire.
        \item Soit $(f_n)_n$ une suite de fonctions  $k$-contractante de $F$ dans $F$ qui converge simplement vers  $f$. Alors, $f$ est  $k$-contractante et la suite des points fixes  $(x^\star_n)$ converge vers  $x^\star$ l'unique point fixe de  $f$
    \end{enumerate}
\end{thm}

\begin{proof}~
\begin{enumerate}[label=(\alph*)]
    \item L'unicité provient de la relation $\|a-b\|=\|f(a)-f(b)\|\leq k\|a-b\|$. Pour l'existence, on remarque \[
            \|x_{n+2}-x_{n+1}\|\leq k \|x_{n+1}-x_n\| \leq k^{n+1} \|x_1-x_0\|
    \] 
    de sorte que pour $q\geq p$, \[
        \|x_q-x_p\| \leq \|x_1-x_0\|\sum_{n=p}^{q-1} k^n\leq \frac{k^p}{1-k}\|x_1-x_0\|\xrightarrow[p\to +\infty]{}0
    \]
    Donc $(x_n)_n$ est de Cauchy et converge vers  $x \in  F$ qui est un point fixe (évident car $f$ continue en  $x$).
    \item \[
            \|x^\star-x_n^\star\|= \|f(x^\star)-f_n(x^\star_n)\|\leq \|f(x^\star)-f_n(x^\star)\| + k\|x^\star-x_n^\star\|
    \] 
    donc $\displaystyle \|x^\star - x_n^\star\|\leq \frac{1}{1-k} \|f(x^\star)-f_n(x^\star)\|\xrightarrow[n\to+\infty]{}0$
\end{enumerate}
\end{proof}

\subsection{Preuve du théorème}

\begin{proof}[Preuve du théorème d'inversion locale]
Pour $b \in  \C$, on pose \[
\begin{array}{rrcl}
    g_b:& U & \longrightarrow & \C \\
        & z & \longmapsto & \displaystyle z- \frac{f(z)-b}{f'(a)}
\end{array}
\] 
Alors $g_b(z)=z \iff  f(z)=b$, et $g_b \in  \mathcal  H(U)$, $g_b'$ continue,  \[
    g_b'(z)=1-\frac{f'(z)}{f'(a)} \text{ indépendant de }b
\]
et $g_b'(a)=0$. Il existe donc $r>0$ tel que  $\bar \B(a, r)\subset U$. \[
    \forall  b \in  \C, \forall  z \in  \bar{\B}(a, r), \quad  |g_b'(z)|\leq \frac{1}{2}
\] 
Soit $0<r'<r$.  \[
    |g_b(z)-a|\leq |g_b(z)-g_b(a)|+|g_b(a)-a|\leq \frac{1}{2} |z-a|+ \left| \frac{b-f(a)}{f'(a)} \right|
\] 
Soit $b \in  \B \left(f(a), \frac{r'}2 |f'(a)|\right)$. Pour un tel $b$, l'application  $g_b$ induit une application $\sfrac{1}{2}$-contractante  de $\bar{\B}(a, r)$ dans $\B(a, r)\subset \bar{\B}(a, r)$.
Il existe donc un unique $z \in  \bar{\B}(a, r)$ tel que $g_b(z)=z \iff  f(z)=b$, et $z \in  \B(a, r)$.


Notons $W=\B \left(f(a), \frac{r'}2 |f'(a)|\right)$ et  $V=f^{-1}(W)\cap \B(a, r)$. L'application $f:V\longrightarrow f(V)=W$ est bijective. 
Comme
\[
\left|g_{b}(z)-g_{b'}(z)\right|=\frac{1}{\left|f'(a)\right|}\left|b-b'\right|
\]
pour tous $b, b' \in \C$ et tout $z \in U$, il résulte du point (b) du théorème du point fixe et $\diff u$ critère séquentiel de continuité que l'application $f^{-1}: W \rightarrow V$ est continue.

Il reste à démontrer qu'elle est holomorphe. Soit $b \in W$ et soit $z \in W-\{b\} .$ Comme $f^{-1}(b) \in$ $B(a, r)$, il résulte de $(1)$ que $f'\left(f^{-1}(b)\right) \neq 0 .$ De plus, on a les relations
\[
\frac{f^{-1}(z)-f^{-1}(b)}{z-b}=\left(\frac{f\left(f^{-1}(z)\right)-f\left(f^{-1}(b)\right)}{f^{-1}(z)-f^{-1}(b)}\right)^{-1}
\]
qui converge vers $f'\left(f^{-1}(b)\right)^{-1}$ puisque $f^{-1}$ est continue.

\end{proof}
