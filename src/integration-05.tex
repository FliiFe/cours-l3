\ifsolo
    ~

    \vspace{1cm}

    \begin{center}
        \textbf{\LARGE Mesures produit et théorème de Fubini} \\[1em]
    \end{center}
    \tableofcontents
\else
    \chapter{Mesures produit et théorème de Fubini}

    \minitoc
\fi
\thispagestyle{empty}

L'objectif de ce chapitre est de considérer des mesures et des intégrales sur des espaces produit. Par exemple, si $(X, \mathcal  A, \mu)$ et $(Y, \mathcal  B, \nu)$ sont des espaces mesurés, on veut donner un sens à \[
    \int_{X\times Y}f(x, y)\mu(\diff x)\nu(\diff y)
\] 

\section{Tribus produit}

\begin{dfn}
    Soit $(X, \mathcal  A)$ et $(Y, \mathcal  B)$ des espaces mesurables. On note $\mathcal  A\otimes \mathcal  B$ la plus petite $\sigma$-algèbre sur  $X\times Y$ contenant les ensembles de la forme  $A\times B$ avec  $A \in  \mathcal  A, B \in  \mathcal  B$. On l'appelle $\sigma$-algèbre produit\index{tribu produit}\index{sigma-algebre produit@$\sigma$-algèbre produit} de  $\mathcal  A$ et $\mathcal  B$
\end{dfn}

\begin{prop}
La  $\sigma$-algèbre  $\mathcal  A\otimes \mathcal  B$ est la plus petite $\sigma$-algèbre  $ \mathcal  C$ telle que les applications \[
\begin{array}{rrcl}
    \pi_X:& X\times Y & \longrightarrow & X \\
          & (x, y) & \longmapsto & \displaystyle x
\end{array}
\] 
\[
\begin{array}{rrcl}
    \pi_Y:& X\times Y & \longrightarrow & Y \\
          & (x, y) & \longmapsto & \displaystyle y
\end{array}
\] 
sont mesurables.
\end{prop}

\begin{proof}
Si ces deux applications sont mesurables, pour  $A \in \mathcal  A$, $B \in  \mathcal  B$, alors \[
    A\times B = \pi_X^{-1}(A)\cap \pi_Y^{-1}(B) \in  \mathcal  C
\] 
donc $\mathcal  A\otimes \mathcal  B\subseteq \mathcal  C$

La réciproque est facile (mêmes arguments).
\end{proof}

\begin{rem}
Le produit de $\sigma$-algèbre est associatif. On étend donc à un produit quelconque.
\end{rem}

\begin{ex}
    Sur $\R^2$, on a deux $\sigma$-algèbres naturelles:  $\mathcal  B(\R^2)$ et $\mathcal  B(\R)\otimes \mathcal  B(\R)$.
\end{ex}

\begin{prop}
Soient $X$,  $Y$ deux espaces topologiques à base dénombrable d'ouverts. Alors, si  $X\times Y$ est muni de la topologie produit, on a  \[
    \mathcal  B(X\times Y)=\mathcal  B(X)\otimes \mathcal  B(Y)
\] 
\end{prop}

\begin{proof}
    On a que $\pi_X$ et $\pi_Y$ sont des applications continues pour les topologies correspondantes. Elles sont donc mesurables de $\mathcal  B(X\times Y)$ vers $\mathcal  B(X)$ et $\mathcal  B(Y)$ respectivement. Donc, vu la propriété précédente, \[
        \mathcal  B(X)\otimes \mathcal  B(Y)\subseteq\mathcal B(X\times Y)
    \] 
    Pour la réciproque, on utilise le fait que $X\times Y$ est lui-même à base dénombrable d'ouverts  $(O_i\times O_j')$ donc tout ouvert de  $X\times Y$ est dans  $\mathcal  B(X)\otimes \mathcal  B(Y)$ comme union dénombrable de produits d'ouverts.
\end{proof}

\begin{dfn}
Soit $C\subset X\times Y$ et  $x \in  X, y \in  Y$, alors on considère \[
    C_x=\pi_Y(C\cap (\left\{ x \right\}\times Y ))= \left\{ y' \in Y, \quad  (x, y') \in C \right\} 
\] 
et de même pour $C^y$. On appelle ces ensembles les sections de  $C$\index{sections}. Si $f: X\times Y \longrightarrow  Z$ et $(x, y) \in  X\times Y$, on note \[
    f_x(y)=f(x, y)
\] 
et \[
    f^y(x)=f(x, y)
\] 
\end{dfn}

\begin{prop}
    Si $C \in  \mathcal  A\otimes \mathcal  B$ et $f:X\times Y \longrightarrow (Z, \mathcal  C)$ est mesurable, alors $C_x \in  \mathcal  B, C^y \in  \mathcal  A$ et $f_x, f^y$ sont mesurables.
\end{prop}

\begin{proof}
Les ensembles $C=A\times B$ vérifient la proposition puisque  $C_x=\emptyset$ si  $x \not \in  A$, et $C_x=B$ sinon, idem pour  $C^y$.

De plus,  \[
\left\{ C \in  \mathcal  A\otimes \mathcal  B, \qquad  \forall x \in  X, \forall  y \in  Y, C_x \in  \mathcal  B, C^y \in  \mathcal  A \right\} 
\] 
est une $\sigma$-algèbre. Donc, c'est  $\mathcal  A\otimes \mathcal  B$. Si maintenant $C \in  \mathcal C$, \[
    f_x^{-1}(C)= \left\{ y \in  Y, \qquad  f_x(y)\in C \right\} = (f^{-1}(C))_x \in \mathcal  B
\] 
\end{proof}

\section{Mesure produit}

\begin{thm}
    Soient $(X, \mathcal  A, \mu)$ et $(Y, \mathcal  B, \nu)$ deux espaces mesurés $\sigma$-finis. Alors, il existe une unique mesure $\mu\otimes \nu$ sur $(X\times Y, \mathcal  A\otimes \mathcal  B)$, appelée mesure produit\index{mesure produit} telle que \[
        \forall  A \in  \mathcal  A, \forall  B \in  \mathcal  B, \qquad  \mu\otimes \nu(A\times B)=\mu(A)\nu(B)
    \] 
\end{thm}

\begin{proof}[Preuve d'unicité de la mesure produit]
L'unicité provient d'un théorème d'unicité des mesures puisque $\left\{ A\times B, \quad  A \in  \mathcal  A, B \in  \mathcal  B \right\} $ est un $\pi$-système.
\end{proof}

\begin{lmm}
Si $C \in  \mathcal  A\otimes \mathcal  B$ alors les applications \[
    x \in  X \longmapsto \nu(C_x) \in  \bar{\R_+}
\] 
et \[
    y \in  Y \longmapsto \mu(C^y) \in  \bar{\R_+}
\] 
sont mesurables
\end{lmm}

\begin{proof}[Preuve du lemme]
Si $C =  A\times B$ avec $A \in  \mathcal  A, B \in  \mathcal  B$, alors \[
    \nu(C_x)=\nu(B)\1_A(x)
\] 
est mesurable. Par ailleurs, supposons d'abord $\nu$ finie. Notons  $\mathcal  M$ la famille des $C \in  \mathcal  A\otimes \mathcal  B$ vérifiant la conclusion du lemme. Si $C\subseteq C'$ sont dans  $\mathcal  M$ alors \[\nu((C'\ \setminus  C)_x)=\nu(C'_X\setminus  C_x)=\nu(C'_x)-\nu(C_x)\] est mesurable en $x$

Si  $(C_n)$ est une suite croissante d'éléments de  $\mathcal  M$ alors \[
    \nu \left( \left( \ubigcup C_n \right)_n \right)=\nu\left(\ubigcup(C_n)_x\right)=\lim_{n\to +\infty}\nu((C_n)_x)
\] 
est mesurable comme limite simple de mesurables. $\mathcal  M$ est donc une classe monotone et (lemme de Dynkin) $\mathcal  M=\mathcal  A\otimes \mathcal  B$

Si $\nu$ est infinie, on raisonne sur chaque ensemble d'un recouvrement par mesurables de mesure finie.
\end{proof}

\begin{proof}[Preuve d'existence de la mesure produit]
Posons, pour $C \in  \mathcal  A\otimes \mathcal  B$, \[
    \xi(C)=\int_X \nu(C_x)\diff \mu(x)
\] 
Alors, $\xi$ est une mesure:  \begin{itemize}
    \item $\xi(\emptyset ) = 0$
    \item Si les  $C_n$ sont deux à deux disjoints,  \[
            \xi \left(  \bigcup C_n\right)=\int_X \nu \left(  \left(\bigcup C_n\right)_x\right)\diff\mu(x)=\int_X \sum_{n\geq 0} \nu((C_n)_x)\diff\mu(x)\overset{\text{Beppo-Levi}}=\sum_{n\geq 0}\xi(C_n)
    \] 
\end{itemize}
    De plus, si $A \in  \mathcal A, B \in  \mathcal  B$ alors \[
        \xi(A\times B)=\int_X\nu(B)\1_A\diff\mu=\mu(A)\nu(B)
    \] 
\end{proof}

\begin{cor}
On a \[
    \mu\otimes \nu(C)=\int_X \nu(C_x)\mu(\diff x)=\int_Y\mu(C^y)\nu(\diff y)
\] 
\end{cor}

\section{Théorème de Fubini}

\begin{thm}[Fubini-Tonelli\index{Fubini-Tonelli (théorème)}]
    Soit $(X, \mathcal  A, \mu)$ et $(Y, \mathcal  B, \nu)$ des espaces mesurés $\sigma$-finis et $f:X\times Y\longrightarrow \bar{\R_+}$ mesurable. Alors \begin{enumerate}
        \item Les fonctions $x \longmapsto \int_y f_x\diff \nu$ et $x \longmapsto \int_X f^y\diff \mu$ sont mesurables
        \item On a que \[
                \int_X \diff\mu \left( \int_Y f_x\diff \nu \right)=\int_Y \diff\nu \left( \int_X f^y\diff \mu \right)
        \] 
    \end{enumerate}
\end{thm}

\begin{rem}
On note \[
    \int_X\diff \mu \left(  \int_Y f_x\diff \nu\right) = \int_X \diff\mu(x) \int _Y \diff\nu(y) f(x, y)
\] 
\end{rem}

\begin{proof}
C'est vrai pour les $\1_C$ avec  $C \in  \mathcal  A\otimes \mathcal  B$ par construction de $\mu\otimes \nu$. On étend aux fonctions simples puis aux fonctions quelconques par convergence monotone.
\end{proof}

\begin{thm}[Fubini-Lebesgue\index{Fubini-Lebesgue (théorème)}]
    Soient $(X, \mathcal  A, \mu)$ et $(Y, \mathcal  B, \nu)$ des espaces mesurés $\sigma$-finis et  $f: X\times Y \longrightarrow  \C$ intégrable. Alors \begin{itemize}
        \item Pour $\mu$-presque tout  $x$,  $f_x$ est intégrable
        \item Pour $\mu$-presque tout  $y$,  $f^y$ est intégrable
    \end{itemize}
    Les fonctions définies par  \[I(x)=\int_Yf(x, y)\nu(\diff y)\] et \[J(y)=\int_Xf(x, y)\mu(\diff x)\] sont intégrables et leurs intégrales sont égales.
\end{thm}

\begin{proof}
    Appliquons Fubini-Tonelli à $|f|$. On a \[\int_{X\times Y}|f|\diff \mu\otimes \nu=\int_X\diff \mu(x)\int_Y\diff \nu(y)|f(x,y)|=\int_Y\diff\nu(y)\int_X\diff\mu(x)|f(x,y)|\] donc $\mu$-presque partout, \[\int_Y\diff\nu(y)|f(x,y)|<+\infty.\] Ainsi,  $f_x\in \mathcal L^1(\nu)$ et on conclut en appliquant le théorème de Fubini-Tonelli à $(\Re f)^+, (\Re f)^-, (\Im f)^+, (\Im f)^-$
\end{proof}

\section{Intégration par parties}

Si $F : \R \longrightarrow  \R$ est croissante continue à droite, on note \[F(x-)=\lim_{\substack{y\to x\\y<x}} F(y)\leq F(x)\] et \[\Delta f=F(x)-F(x-).\]

Rappelons qu'il existe une unique mesure $\diff F$ sur  $ \R$ telle que $\diff F(]a, b])=F(b)-F(a)$ (mesure de Stieltjes). Par exemple, si  $F$ est  $\mathcal  C^1$, $\diff F(x)=f(x)\diff x$, c'est-à-dire \[\diff F(A)=\int_Af(x)\diff x\]

\begin{thm}[Intégration par parties\index{intégration par parties}]
    Soient $F, G : \R \longrightarrow  \R$ croissantes continues à droite et $a<b$. Alors, \[F(b)G(b)=F(a)G(a)+\int_a^bF(x-)\diff G(x)+\int_a^bG(x-)\diff F(x)+\sum_{a<x\leq b}\Delta F(x)\Delta G(x)\]
    Lorsque  $F, G$ sont  $\mathcal  C^1$, on retrouve la formule usuelle d'intégration par parties.
\end{thm}

\begin{rem}
    L'ensemble des discontinuités de $\Delta F$ ou  $\Delta G$ est au plus dénombrable (classique)
\end{rem}

\begin{proof}
    \begin{align*}\int_{]a, b]}\diff F\otimes \diff G&=\diff F(]a, b])\diff G(]a, b])\\&=(F(b)-F(a))(G(b)-G(a))\\&=\int_{]a, b]^2}\diff F\otimes \diff G\1_{x=y}+\int_{]a, b]^2}\diff F\otimes \diff G\1_{x<y}+\int_{]a, b]^2}\diff F\otimes \diff G\1_{x>y} \\ &\overset{\mathclap{\text{F-T}}}=\int_{]a, b]}\diff F(x)\underbrace{\int_{]a, b]}\diff G(y)\1_{x=y}}_{\diff G( \left\{ x \right\} )=\Delta G(x)}+ \int_{]a, b]}\diff G(y)\underbrace{\int_{]a, b]}\diff F(x)\1_{x<y}}_{\diff F(]a, b])=F(y-)-F(a)}\\&\hspace{3cm} +\int_{]a, b]}\diff F(x)\underbrace{\int_{]a, b]}\diff G(y)\1_{y<x}}_{G(x-)-G(a)} \\ &= \sum_{x \text{ tq }\Delta G(x)>0}\underbrace{\diff F( \left\{ x \right\}  )}_{\Delta F(x)} \Delta G(x) + \left(\int_{]a,b]}\diff G(y)F(y-)-F(a)(G(b)-G(a))\right) \\ &\hspace{3cm}+\left(\int_{]a,b]}\diff F(x)G(x-)-G(a)(F(b)-F(a))\right)
    \end{align*}
    d'où la formule recherchée.
\end{proof}

\begin{rem}
Le terme de discontinuité est nul dès que $F$ et  $G$ n'ont pas de discontinuité commune
\end{rem}

\section{Convolution}

\index{convolution}

Soient $f, g : \R^d \longrightarrow  \R$ mesurables et $x \in  \R$. L'intégrale \[\int_{\R^d}f(y)g(x-y)\diff y\] a un sens dès lors que l'on a \[\int_{\R^d}|f(y) g(x-y)|\diff y<+\infty.\] On note alors cette intégrale \[f\star g(x)\]

\begin{lmm}
    Si $f$ est positive intégrable et  $y \in  \R^d$ alors \[\int_{\R^d} f(-x)\diff x=\int_{\R^d}f(x+y)\diff x=\int_{\R^d}f(x)\diff x\]
\end{lmm}

\begin{lmm}
    Si $a>0$,  et $f$ est mesurable positive ou bien intégrable, \[\int_{\R^d}f(ax)\diff x=\frac1{a^d}\int_{\R^d}f(x)\diff x\]
\end{lmm}

\begin{rem}
    Par le premier lemme, si $f\star g(x)$ est bien définie, $g\star f(x)$ aussi et ces deux intégrales sont égales.
\end{rem}

\begin{thm}
Soient $f, g$ deux fonctions intégrables. Alors  $f\star g$ est définie  $\lambda$-presque partout et on a  \[
\|f\star g\|_1\leq \|f\|_1 \|g\|_1
\] 
Autrement dit, $\star : \L^1\times \L^1 \longrightarrow \L^1$ est bilinéaire de norme $\leq 1$.
\end{thm}

\begin{proof}
    Pour la tribu produit, $(x, y)\mapsto(y, x-y)$ et $(u, v)\mapsto (f(x), g(y))$ sont mesurables. Le théorème de Fubini-Tonelli donne \[x \longmapsto \int_{\R^d}|f(y)g(x-y)|\diff y\] est mesurable et \begin{align*}
    \int_{\R^d}\diff x\int_{\R^d}|f(y)g(x-y)|\diff y&=\int_{\R^d} \diff y|f(y)|\int_{\R^d} \diff x |g(x-y)| \\ &=\int_{\R^d}\diff y |f(y)| \|g\|_1= \|f\|_1 \|g\|_1
\end{align*}
Comme $f$ et  $g$ sont intégrables,  $f\star g$ est bien définie  $\mu$-presque partout. Puis, \[\int_{\R^d}|f\star g(x)|\diff x=\int_{\R^d}\left|\int_{\R^d}f(y)g(x-y)\diff y\right|\diff x\leq \|f\|_1 \|g\|_1\]
\end{proof}

\begin{thm}
    Soient $p \in  [1, +\infty[$ et $q \in  ]1, +\infty]$ conjugués, et $f \in  \L^p, g \in  \L^q$. Alors $f\star g(x)$ est bien définie pour tout  $x \in  \R^d$ et uniformément continue, et \[|f\star g(x)|\leq \|f\|_p \|g\|_q\]
    Enfin, si $p, q \in  ]a, +\infty[$, alors $f\star g$ est de limite nulle à l'infini.
\end{thm}

\begin{lmm}
    Pour $f $ mesurable et  $x, h \in  \R^d$, on note $\tau_hf(x)=f(x+h)$. Soit  $p \in  [1, +\infty[$. Alors, \[
    h \in  \R^d \longmapsto \tau_h f \in  \L^p
    \] 
    est uniformément continue pour tout $f \in  \L^p$.
\end{lmm}

\begin{proof}
    On a \[\|\tau_h f-\tau h'f\|_p= \|\tau_{h-h'}f-f\|_p.\] Il suffit de montrer que \[\|\tau_hf-f\|_p \xrightarrow[h\to 0]{}0\]
    Soit $\epsilon>0$ et $g: \R^d\longrightarrow \C$ continue à support compact telle que $ \|f-g\|_p\leq \frac\epsilon2$ (existe par densité). Alors, 
    \begin{align*}
        \|\tau _h f-f\|_p&\leq \|\tau_hf-\tau_hg\|_p+\|\tau _h g-g\|_p + \|g-f\|_p \\ &= 2\|g-f\|_p + \|\tau_h g-g\|_p \\ &\leq \frac\epsilon2+\|\tau_hg-g\|_p
    \end{align*}
    or $g$ est uniformément continue (Heine) donc il existe $\delta>0$ tel que \[|h|\leq  \delta \implies |\tau_h g(x)-g(x)|\leq\frac\epsilon{\lambda(B)^{1 / \alpha}} \]

    Supposons que $|h|\leq 1$. Alors \[\int_{\R^d}|\tau_h g(x)-g(x)|^p\diff x=\int_B|\tau_h g(x)-g(x)|^p\diff x\] où $B$ est une boule fermée qui contient le $1$-voisinage du support de  $g$, de sorte que pour tout  $|h|\leq \min(1, \delta)$, $\|\tau_hg-g\|_p^p\leq \epsilon^p\lambda(B)$. Alors on a $\|\tau_h (f)-f\|_p\leq \epsilon$ pour le bon choix de $\delta$
\end{proof}

\begin{proof}[Preuve du théorème]
    L'inégalité de Hölder donne \[|f\star g(x)|\leq \|f\|_p \|g\|_q<+\infty\] donc cette quantité a un sens. Alors,
    \begin{align*}
        |f\star g(x+h)-f\star g(x)|&\leq \int_{\R^d}|f(x+h-y)-f(x-y)|\,|g(y)|\diff y\\&\leq \|\tau_h f-f\|_p \|g\|_q \xrightarrow[h\to 0]{}0
    \end{align*}
    Enfin, l'inégalité triangulaire donne \[
        |f\star g(x)|\leq \|f\|_p \|g\|_q
    \] 
    donc $f\star g$ est bornée. Si enfin  $p, q \in  ]1, +\infty[$, alors on peut choisir $\varphi, \psi$ continues à supports compacts telles que $\|f-\varphi\|_p$ et $\|g-\psi\|_q$ sont petits. Alors,
    \begin{align*}
        |f\star g(x)-\varphi\star \psi(x)|\leq |(f-\varphi)\star g(x)|+|\varphi\star (g-\psi)(x)| \\ &\leq \|f-\varphi\|_p\|g\|_q + \|\varphi\|_p\|g-\psi\|_q
    \end{align*}
    donc en choisissant bien $\varphi, \psi$, on a \[|f\star g(x)|\leq |\varphi\star \psi(x)|+\epsilon=\epsilon\] pour $|x|$ assez grand
\end{proof}

\section{Approximation de l'unité}

Il n'y a pas d'élément neutre pour la convolution, mais on peut s'en approcher.

\begin{dfn}
    Une approximation de $\delta_0$ est une suite  $(\varphi_n)_n$ de fonctions positives mesurables telles que \[ \forall  n \in  \N, \qquad  \int_{\R^d}\varphi_n(x)\diff x=1\] et \[\forall \delta>0, \qquad  \int_{|x|>\delta }\varphi_n(x)\diff x \xrightarrow[n\to+\infty]{}0\]
\end{dfn}

\begin{thm}
    Si $(\varphi_n)$ est une approximation de $\delta_0$ et $f$ est bornée, alors \begin{itemize}
        \item Si $f$ est uniformément continue,  $\varphi_n\star f$ converge uniformément vers  $f$ sur  $\R^d$
            \item Si $f$ est continue partout sur  $K$ compact, alors  $\varphi_n\star f$ converge uniformément sur  $K $ vers $f$.
    \end{itemize}
\end{thm}

\begin{proof}
La quantité $\varphi_n\star f$ est bien définie car $f \in  \mathcal L^\infty, \varphi_n \in  \mathcal L^1$, puis \begin{align*}
\varphi_n\star f(x)-f(x)&= \int_{\mathbb{R}^d}\varphi_n(y)(f(x-y)-f(x))\diff y\\
& = \int_{|y|\leq \delta} \varphi_n(y) (f(x-y)-f(x))\diff y + \int_{|y|>\delta}\varphi_n(y)(f(x-y)-f(x))\diff y\\
\end{align*}
\begin{itemize}
    \item Si $f$ est uniformément continue, alors pour  $ \epsilon>0$ il existe $\delta>0$ tel que  $|y|\leq \delta \to \forall x \in  \R^d, |f(x-y)-f(x)|\leq \frac\epsilon2$. Pour ce choix de $\epsilon, \delta$, on a \[|\varphi_n\star f(x)-f(x)|\leq \frac\epsilon2\int_{\R^d}\varphi_n+2\|f\|_\infty\int_{|y|>\delta}\varphi_n(y)\leq \epsilon\] pour $n$ assez grand. 
        \item $f$ est uniformément continue sur  $K$, c'est pareil.
\end{itemize}
\end{proof}

\begin{thm}
    Soit $(\varphi_n)$ une approximation de l'unité et  $f \in  \mathcal  L^p(\R^d, \mathcal  B(\R^d), \lambda), p \in  [1, +\infty[$. Alors, $\varphi_n\star f$ converge vers $f$ dans  $\mathcal  L^p$.
\end{thm}

\begin{proof}
    \[\left(\int_{\R^d} \varphi_n(y)|f(x-y)|\diff y\right)^p \leq \int_{\R^d}|f(x-y)|^p\varphi_n(y)\diff y\] par l'inégalité de Jensen car $ \varphi_n(y)\diff y $ est une mesure de probabilité. Donc, le théorème de Fubini-Tonelli donne \[
    \int_{\R^d} \diff x \left(\int_{\R^d}\varphi_n(y)|f(x-y)|\diff y\right)^p\leq \int_{\R^d}\varphi_n(y)\diff y\int_{\R^d}|f(x-y)|^p\diff x=\|f\|_p^p.
\]
Donc $\varphi_n\star f(x)$ est définie pour presque tout  $x$ et est dans  $\mathcal  L^p$. On a alors pour presque tout $x$
 \begin{align*}
|\varphi_n\star f(x)-f(x)|^p &= \left| \int_{\R^d}\varphi_n(y)(f(x-y)-f(x))\diff y \right|^p\\ &\leq \int_{\R^d} \varphi_n(y)|f(x-y)-f(x)|^p\diff y.
\end{align*}
Alors, \[\|\varphi_n\star f-f\|_p\leq \int_{\R^d}\varphi_n(y)\diff y\|\tau_yf-f\|_p^p\] par fubini-Tonelli. Par uniforme continuité de $h\mapsto \tau_hf$, pour $\epsilon>0$ fixé, il existe $\delta>0$ tel que  \[
|h|\leq \delta \implies \|\tau_hf-f\|_p^p\leq \frac{\epsilon^p}2
\] 
de sorte que \[\|\varphi_n\star f-f\|_p^p\leq \frac{\epsilon^p}2+\int_{|y|>\delta}\varphi_n(y)\diff y (2\|f\|_p)^p\leq \epsilon^p\] pour $n$ assez grand.
\end{proof}

\begin{cor}[Weierstrass\index{Weierstrass (théorème)}]
    Si $f:[a, b]\longrightarrow \C$ est continue, elle est limite uniforme de polynômes.
\end{cor}

% TODO: preuve du corrolaire

\begin{proof}
Soit \[
    \varphi_n(x)=c_n(1-x^2 )^n\1_{|x|\leq 1}\geq 0 \qquad \text{ où } \qquad \frac1{c_n}=\int_{[-1,1]}(1-x^2 )^n\diff x.
\] 
On a bien \[
\int_{|x|>\delta}\varphi_n\diff\lambda \xrightarrow[n\to+\infty]{}0.
\]
Supposons maintenant que $[a, b]\subset ]0,1[$ et  $f:[a, b] \longrightarrow  \C$ est continue. On prolonge $f$ en une fonction continue à support dans  $[0,1]$. Par le théorème,  $\varphi_n\star f \longrightarrow f$ uniformément sur le compact $[a, b]$. Or
 \begin{align*}
     \varphi_n\star f(x)&=\int_{\R} \varphi_n(x-y)f(y)\diff y\\ &=\int_{[0,1]}\varphi_n(x-y)f(y)\diff y\\&= c_n\int_{[0,1]}(1-(x-y)^2 )^n f(y)\diff y \1_{|x-y|\leq 1}
\end{align*}
et c'est un polynôme en $x$.
\end{proof}

\begin{thm}
    L'ensemble $\mathcal  C^\infty(\R^d, \C)$ est dense dans $\mathcal  L^p(\R^d, \mathcal  B(\R^d), \lambda)$ pour tout $p \in  [1, \infty[$.
\end{thm}

\begin{proof}[Idée de preuve]
Choisir $\varphi_n$ approximation de  $\delta_0$,  $\mathcal  C^\infty$ et utiliser le fait que \[
    \forall  \alpha, \quad  \frac{\partial^\alpha}{\partial x^\alpha}(\varphi_n\star f)= \frac{\partial^\alpha \varphi_n}{\partial x^\alpha}\star f
\] 
\end{proof}

\begin{thm}[Féjer\index{Féjer (théorème)}]
Soit $f:\R\longrightarrow \C$ une fonction continue  $2\pi$-périodique. Soit \[
    \hat{f}(n)=\frac{1}{2\pi}\int_0^{2\pi}f(x)e^{-inx}\diff x, \quad  n \in  \Z
\]
Alors \[
    \frac{1}{N+1}\sum_{n=0}^N\sum_{k=-n}^n \hat{f}(k)e^{ikx} \xrightarrow[N\to +\infty]{}f(x)
\] 
uniformément sur $[0, 2\pi]$ ou sur $\R$.
\end{thm}

\begin{proof}
    On va se placer sur l'espace $\mathcal  L^2([0,2\pi[, \frac\lambda{2\pi})$, qu'on voit comme contenant les fonctions continues $2\pi$-périodiques (en restreignant à une période). On adopte dans ce contexte la définition de la convolution suivante: \[
        f\star g(x)= \frac{1}{2\pi}\int_0^{2\pi}f(y)g(x-y)\diff y
    \] 
    On a \[
        \sum_{k=-n}^n \hat{f}(k)e^{ikx} = \frac{1}{2\pi}\int_0^{2\pi} \sum_{k=-n}^n f(y)e^{ik(x-y)}\diff y=D_n\star f(x)
    \]
    où $D_n$ est le noyau de Dirichlet \[
        D_n(x)= \frac{e^{i(n+1)x}-e^{-inx}}{e^{ix}-1}= \frac{\sin((n+\sfrac12)x)}{\sin(\sfrac x2)}
    \] 
    Remarquons que \[
        \frac{1}{2\pi}\int_0^{2\pi} D_n(x)\diff x=\sum_{k=-n}^n\frac{1}{2\pi}\int_0^{2\pi} e^{inx}\diff x=1
    \]
    On a alors \[
        \frac{1}{N+1}\sum_{n=0}^N D_n\star f(x)=\frac{1}{2\pi}\int_0^{2\pi}\left(\frac{1}{N+1}\sum_{n=0}^ND_n(x-y)\right) f(y)\diff y=F_n\star f(x)
    \] 
    où $F_n$ est le noyau de Féjer \[
        F_N(x)=\frac{1}{N+1}\sum_{n=0}^N D_n(x)
    \]
    qui est tel que \[
        \frac{1}{2\pi}\int_0^{2\pi}F_n\diff \lambda=1.
    \] 
    Comme \begin{align*}
    F_N(x) &= \frac{1}{(N+1)\sin(\sfrac{x}{2})} \sum_{n=0}^{N} \Im\left(e^{i(n+\sfrac12)x}\right)\\
&= \frac{1}{(N+1)\sin(\sfrac{x}{2})}\Im \left( \frac{e^{i(N+\sfrac32)x}-e^{i\sfrac{x}{2}}}{e^{ix}-1}\right),
    \end{align*}
    on a 
    \begin{align*}
F_N(x) &= \frac{1}{(N+1)\sin(\sfrac{x}{2})}\Im \left( \frac{e^{i(N+1)x}-1}{2i\sin(\sfrac{x}{2})}\right)\\
&= \frac{1}{(N+1)\sin(\sfrac{x}{2})^2}\Im \left( e^{i \sfrac{N+1}{2}x}\frac{e^{i \sfrac{N+1}{2}x}-e^{i \sfrac{N+1}{2}x}}{2i}\right)\\
&= \frac{\sin(\sfrac{N+1}{2})x}{(N+1)\sin(\sfrac{x}{2})^2}\geq 0
\end{align*}
\end{proof}
