\ifsolo
    ~

    \vspace{1cm}

    \begin{center}
        \textbf{\LARGE Espaces fonctionnels} \\[1em]
    \end{center}
    \tableofcontents
\else
    \chapter{Espaces fonctionnels}

    \minitoc
\fi
\thispagestyle{empty}

\section{L'espace \texorpdfstring{$\mathbf L^1((X, \mathcal  A, \mu), \C)$ }{L1((X, A, mu), C)}}

\begin{dfn}
    On note $\L1((X, \mathcal  A, \mu), \C)$, abrégé en $\L1(\mu)$, l'espace des classes d'équivalence modulo la relation d'égalité  $\mu$-presque partout sur les fonctions intégrables. On confondra une fonction et sa classe d'équivalence, et pour  $f \in \L1(\mu)$, on pose \[
    \|f\|_1=\int_X|f|\diff\mu
    \] 
\end{dfn}

\begin{rem}
    L'espace $\L1(\mu)$ est un  $ \C$-ev borné.
\end{rem}

% Si $O$ est un ouvert de $\R^n$, on note  $\L1(O)=\L1(O, \B(O), \mathrm{Leb})$. Si $X$ est un ensemble, on note  $\ell^1(X)=\L1((X, \mathcal  P(X), \mathrm{den}), \C)$.
%TODO: regarder ça

\begin{rem}
Dans la suite, on notera $\diff x$ pour la mesure de Lebesgue.
\end{rem}

