\ifsolo
    ~

    \vspace{1cm}

    \begin{center}
        \textbf{\LARGE Construction de mesures, mesure de Lebesgue} \\[1em]
    \end{center}
    \tableofcontents
\else
    \chapter{Construction de mesures, mesure de Lebesgue}

    \minitoc
\fi
\thispagestyle{empty}

Dans ce chapitre, on va montrer l'existence et l'unicité de la mesure de Lebesgue $\lambda_d$ sur  $(\R^d, \mathcal  B(\R^d))$ telle que \[
    \forall  a_1<b_1, \cdots , \forall  a_d<b_d, \qquad  \lambda_d \left( \prod_{i=1}^d ]a_i, b_i[ \right)= \prod_{i=1}^{d} (b_i-a_i) 
\]
On va par ailleurs développer des outils généraux très utiles en théorie de l'intégration

\section{Unicité: lemme des classes monotones}

\begin{dfn}
    Une classe monotone\index{classe monotone} sur un ensemble $X$ est une famille  $\mathcal  M\subset \mathcal  P(X)$ telle que \begin{enumerate}
        \item $X \in  \mathcal  M$
        \item Pour tous $A, B \in  \mathcal  M$, $A\subseteq B \implies B \setminus A \in  \mathcal  M$
        \item Pour tout  $A \in  \mathcal  M$, $(A_i\subseteq A_{i+1})\implies \ubigcup_{i\geq 0}A_i \in  \mathcal  M$
    \end{enumerate}
    On parle parfois de $\lambda$-système ou de classes de Dynkin.
\end{dfn}

\begin{lmm}[Lemme de Dynkin ou Lemme $\lambda-\pi$]
    \index{Dynkin (lemme)}\index{lambda-pi@$ \lambda-\pi$ (lemme)} Soit $\mathcal  P\subseteq \mathcal  P(X)$ une famille stable par intersection finie\footnotemark. Alors la plus petite classe monotone $\lambda(\mathcal  P)$ contenant $\mathcal  P$ coïncide avec $\sigma(\mathcal  P)$
\end{lmm}

\footnotetext{on parle alors parfois de $\pi$-système}

\begin{rem}
L'ensemble des classes monotones sur $X$ est stable par intersection.
\end{rem}

\begin{proof}
    On montre que $\lambda(\mathcal  P)$ est stable par intersection finie. Soit $A \in  \mathcal  P$, notons \[
        \mathcal  C_1 = \left\{ B \in  \lambda (\mathcal  P), \quad  \suchthat \quad  A\cap B \in  \lambda (\mathcal  P) \right\} 
    \]
    Alors, $\mathcal  P\subseteq \mathcal  C_1$ et c'est de plus une classe monotone car $X \in  \mathcal  C_1$, si $B\subseteq B'$ dans  $\mathcal C_1$ alors \[
        (B' \setminus B)\cap A = \underbrace{(B'\cap A)}_{\in \lambda(\mathcal  P)} \setminus  \underbrace{(B\cap A)}_{\in  \lambda(\mathcal P)} \in \lambda(\mathcal  P)
    \] 
    donc $B' \setminus B \in \mathcal  C_1$. Finalement, si $(B_i)_i$ est croissante dans  $\mathcal  C_1$ alors \[
        A\cap \left( \ubigcup_{i\geq 0} B_i \right)= \ubigcup_{i\geq 0}(A\cap B_i)
    \] 
    et cette union est croissante donc dans $\lambda(\mathcal  P)$, d'où $\ubigcup B_i \in  \mathcal  C_1$. On a donc $\lambda(\mathcal  P)\subseteq \mathcal  C_1\subseteq \lambda(\mathcal  P)$ donc $\mathcal  C_1 - \lambda(\mathcal  P)$. Par les mêmes arguments, pour $A \in  \lambda(\mathcal  P)$, \[
        \mathcal  C_2 = \left\{ B \in  \lambda(\mathcal  P) \quad \suchthat \quad  A\cap B \in  \lambda(\mathcal  P)  \right\} 
    \] 
    on a $\mathcal  C_2=\lambda(\mathcal  P)$. Ainsi, $\lambda(\mathcal  P)$ est stable par intersection finie, et par complémentation, $\lambda(\mathcal  P)$ est stable par union finie. Si  $(A_i)$ est une suite dans  $\lambda(\mathcal  P)$, on écrit \[
        \bigcup A_i=\ubigcup_{n\geq 1} \left(\bigcup_{i=1}^nA_i\right) \in \lambda(\mathcal  P)
    \] 
    donc $\lambda(\mathcal  P)$ est une $\sigma$-algèbre et donc  $\lambda(\mathcal  P)\subset \sigma(\mathcal  P)$. Comme $\sigma(\mathcal  P)$ est une $\sigma$-algèbre, c'est aussi une classe monotone d'où l'autre inclusion.
\end{proof}

\begin{cor}[Unicité des mesures]
    Soit $\mu$ et  $\nu$ deux mesures que  $(X, \mathcal  A)$ espace mesurable. Soit $\mathcal  P \subseteq \mathcal  P(X)$ une famille stable par intersection finie. Si \begin{itemize}
        \item $\sigma(\mathcal  P)=\mathcal  A$
        \item $\forall  A \in  \mathcal P, \mu(A)=\nu(A)$
        \item $ \exists  X_0\subseteq X_1\subseteq \cdots $ dans $\mathcal  P$ telle que $\mu(X_k)=\nu(X_k)<\infty$ et  $X=\ubigcup X_k$
    \end{itemize}
    alors $\mu=\nu$
\end{cor}

\begin{rem}
    Si $\mu, \nu$ sont finies alors la troisième condition signifie seulement  $\mu(X)=\nu(X)$
\end{rem}

\begin{proof}[Preuve du corrolaire]
    Supposons $\mu(X)$ ($=\nu(X)$) fini. Alors \[
        \mathcal  M= \left\{ A \in  \mathcal  A \;\suchthat\; \mu(A)=\nu(A) \right\} 
    \] 
    est une classe monotone (à vérifier, par exemple $A\subseteq B\implies \mu(B \setminus  A)=\mu(B)-\mu(A)=\nu(B)-\nu(A)=\nu(B \setminus  A)$) donc c'est $\sigma(\mathcal P)=\mathcal  A$. Dans le cas général, on raisonne sur les mesures traces sur les $X_k$.
\end{proof}

\begin{ex}
    La mesure de Lebesgue, si elle existe, est unique. En effet, les pavés forment un $\pi$-système et engendrent  $\mathcal  B(\R^d)$. On peut prendre $X_k=]-k,k[^d$
\end{ex}

\section{Mesures extérieures}

\begin{dfn}
    Une mesure extérieure\index{mesure extérieure} sur un ensemble $X$ est une application  \[
    \begin{array}{rrcl}
        \mu^\star:& \mathcal  P(X) & \longrightarrow &[0, +\infty]  \\
         & E & \longmapsto & \displaystyle \mu^\star(E)
    \end{array}
    \]
    telle que \begin{enumerate}
        \item $\mu^\star(\emptyset)=0$
        \item Si  $A\subseteq B\subseteq X$ alors $\mu^\star(A) \leq \mu^\star (B)$
        \item $\forall  A_1, A_2, \cdots \subseteq X$, \[
                \mu^\star \left(\bigcup_{i\geq 1}A_i\right) \leq \sum \mu^\star(A_i)
        \]
    \end{enumerate}
\end{dfn}

\begin{rem}
    Notons que si $E, A\subseteq X$ et $\mu^\star$ est une mesure extérieure, alors  $\mu^\star(E)\leq \mu^\star(E\cap A)+\mu^\star(E \setminus  A)$
\end{rem}

\begin{dfn}
Une partie $A$ de  $X$ est dite mesurable par rapport à  la mesure extérieure $\mu^\star$ si  \[
    \forall  E \in  \mathcal  P(X), \quad  \mu^\star(E)=\mu^\star(E\cap A)+\mu^\star(E \setminus  A)
\] 
On note $\mathcal  M(\mu^\star)$ l'ensemble des parties $\mu^\star$-mesurables.
\end{dfn}


 \begin{thm}
     Si $\mu^\star$ est une mesure extérieure sur l'ensemble  $X$, alors  $\mathcal M(\mu^\star)$ est une $\sigma$-algèbre et  $\mu^\star$ est une mesure sur $(X, \mathcal  M(\mu^\star ))$
\end{thm}

\begin{lmm}
    Si $\mu^\star(A)=0$ ou  $\mu^\star(A^c)=0$ alors  $A$ est mesurable.
\end{lmm}

\begin{proof}
    Par monotonie, $\mu^\star(E\cap A)+\mu^\star(E\cap A^c)\leq \mu^\star(E)$ (l'un des deux termes de la somme est toujours nul, l'autre est petit)
\end{proof}

\begin{lmm}
    $\mathcal M(\mu^\star)$ est une algèbre.
\end{lmm}

\begin{proof}
    $\emptyset \in  \mathcal  M(\mu^\star)$ par le lemme précédent, la stabilité par complémentaire est évidente. Montrons que c'est stable par union finie. Soient $A_1, A_2 \in  \mathcal M(\mu^\star)$. Alors \begin{align*}
        \mu^\star(E\cap (A_1\cup A_2))&=\mu^\star(E\cap (A_1\cup A_2)\cap A_1)+\mu^\star(E\cap (A_1\cup A_2)\cap A_1^c)\\&=\mu^\star(E\cap A_1)+\mu^\star(E\cap A_2\cap A_1^c)
    \end{align*}
    et \begin{align*}
        \mu^\star(E\cap (A_1\cup A_2))+\mu^\star(E\cap (A_1\cup A_2)^c) &= \mu^\star(E\cap A_1)+\mu^\star(E\cap A_1^c\cap A_2)+\mu^\star(E\cap A_1^c\cap A_2^c)\\&=\mu^\star(E\cap A_1)+\mu^\star(E\cap A_1^c)\\&=\mu^\star(E)
    \end{align*}
\end{proof}

\begin{lmm}
    $\mathcal  M(\mu^\star)$ est stable par union dénombrable d'ensembles deux à deux disjoints.
\end{lmm}

\begin{proof}
    Soit $A_0, \cdots \in \mathcal M(\mu^\star)$ deux à deux disjoints. Montrons que $ \forall  n\geq 0, \forall  E\subseteq X$, \[
        \mu^\star(E)= \sum_{i=0}^{n} \mu^\star(E\cap A_i)+\mu^\star\left(E\cap\left(\bigcup_{i=0}^nA_i\right)^c\right)
    \] 
    C'est vrai par récurrence. % TODO: cette preuve
    Puis, par monotonie, \[
        \mu^\star(E)\geq \sum_{i=0}^n\mu^\star(E\cap A_i)+\mu^\star\left(E\cap \left(\bigcup_{i\geq 0}A_i\right)^c\right)
    \]
    On fait tendre $n\to +\infty$ et par $\sigma$-sous-additivité,  \[
        \mu^\star(E)\geq \sum_{i=0}^{\infty}\mu^\star(E\cap A_i)+\mu^\star\left(E\cap \left(\bigcup_{i\geq 0}A_i\right)^c\right)\geq  \mu^\star\left(E\cap \left(\bigcup_{i\geq 0}A_i\right)\right) + \mu^\star\left(E\cap \left(\bigcup_{i\geq 0}A_i\right)^c\right)
    \] 
    ce qui conclut.
\end{proof}

\begin{proof}[Preuve du théorème]
    Une union dénombrable s'écrit toujours comme union dénombrable d'ensembles disjoints, donc $\mathcal  M(\mu^\star)$. Montrons que $\mu^\star$ est une mesure sur  $\mathcal  M(\mu^\star)$. Avec la dernière inégalité dans la preuve du dernier lemme, pour  des $A_i$ deux à deux disjoints et  $E$ leur union,  \[
        \mu^\star(E)\geq \sum_{i=1}^{+\infty}\mu^\star(A_i)\geq \mu^\star(E)
    \]
\end{proof}

\section{Application à la mesure de Lebesgue}

Fixons $d\geq 1$. On a vu que la mesure de Lebesgue est unique, si elle existe. Soit $E\subseteq \R^d$, notons \[
    \lambda^\star(E)=\inf \left\{ \sum_{i\geq 1}\mathrm{Vol}(P_i) \quad  \suchthat \quad  E\subset \bigcup_{i\geq 1}P_i, \; P_i\text{ pavé ouvert } \right\} 
\] 

\begin{thm}
    $\lambda^\star$ est une mesure extérieure, et  $\mathcal  M(\lambda^\star)\supseteq \mathcal  B(\R^d)$
\end{thm}

\begin{proof}
    La monotonie est claire, $\lambda^\star(\emptyset)=0$ aussi. Soient  $E_1, \cdots \subseteq \R^d$ et soit $(P_i^n)_i$ un recouvrement de $E_n$ par des pavés ouverts de sorte que  \[
        \sum_{i\geq 1}\mathrm{Vol}(P_i^n)\leq \lambda^\star(E_n)+\frac{\epsilon}{2^n}
    \] 
    où $ \epsilon>0$ est fixé. Alors $(P_i^n)_{i,n}$ recouvre $E=\bigcup E_n$ donc \[
        \lambda^\star \left( \bigcup_{n\geq 1}E_n \right) \leq \sum_{n,i\geq 1} \mathrm{Vol}(P_i^n) \leq \sum_{n\geq 1} \left( \lambda^\star(E_n)+ \frac{\epsilon}{2^n} \right)\leq \epsilon + \sum_{n\geq 1}\lambda^\star(E_n)
    \] 
    et comme $\epsilon$ est arbitraire,  $\lambda^\star$ est sous-additive. Montrons que les demi-espaces $D_a=\R^{i-1}\times ]-\infty, a[\times \R^{d-i}$ sont dans $\mathcal  M(\lambda^\star)$, ce qui suffira car cette partie engendre les boréliens. Soit $E \subseteq \R^d$ recouvert par des pavés ouverts $(P_k)_k$. Alors  $(P_k\cap D_a)$ sont des pavés ouverts recouvrant  $E\cap D_a$. Puis si on pose  \[
        P_k'=P_k\cap\left(\R^{i-1}\times \left]a-\frac{\epsilon}{2^kc_k}\right[\times \R^{d-i}\right)
    \]
    avec \[c_k=\prod_{j\neq i}(b_j^{(k)}-a_j^{(k)})\] alors les $P_k'$ sont des pavés ouverts recouvrant  $E\cap D_a^c$ et alors  \begin{align*}
    \lambda^\star(E)\leq \lambda^\star(E\cap D_a)+\lambda^\star(E\cap D_a^c)&\leq \sum_{k\geq 1}\mathrm{Vol}(P_k\cap D_a)+\sum_{k\geq 1}\mathrm{Vol}(P_k')\\&\leq \sum_{k\geq 1}\mathrm{Vol}(P_k\cap D_a)+\mathrm{Vol}(P_k\cap D_a^c)+ \frac{\epsilon}{c_k2^k}c_k\\&\leq \epsilon+\sum_{k\geq 1}\mathrm{Vol}(P_k)
\end{align*}
\end{proof}

\begin{prop}
    Pour tout pavé $P$,  $\lambda^\star(P)=\mathrm{Vol}(P)$
\end{prop}

\section{Ensembles Lebesgue-mesurables et complétion des mesures}

\begin{dfn}
    Soit $(X, \mathcal A, \mu)$ un espace mesuré. Une partie  $A\subset X$ est  $\mu$-négligeable s'il existe  $B \in  \mathcal  Q$ tel que $A\subseteq B$ et  $\mu(B)=0$. On note  $\mathcal  N$ ou $\mathcal N_{\mu}$ l'ensemble des parties $\mu$-négligeables\index{partie négligeable}. On note  $\bar{\mathcal  Q} =\sigma(\mathcal  Q\cup \mathcal  N)$.
\end{dfn}

\begin{prop}
    Soit $(X, \mathcal  Q, \mu)$ un espace mesuré. Alors, \[
        \bar{\mathcal  Q} = \left\{ A\subset X, \quad  \exists B,B' \in  \mathcal  Q, B\subseteq A\subseteq B'\text{ et } \mu(B'\setminus  B)=0 \right\} 
    \] 
    De plus, $\mu$ se prolonge de façon unique en une mesure sur  $(X, \bar{\mathcal  Q})$. On appelle $(X, \bar{\mathcal  Q}, \mu)$ l'espace complété de $(X, \mathcal  Q, \mu)$
\end{prop}

\begin{proof}
    Soit $\mathcal  Q'$ l'ensemble de l'énoncé. Il est clair que $\mathcal  Q\subset \mathcal  Q'$. Puis, $\mathcal  Q'$ est bien une $\sigma$-algèbre (exercice). Puis,  $B=\emptyset$ donne  $\mathcal  N\subset  \mathcal  Q'$ donc $\bar{\mathcal  Q}\subset \mathcal  Q'$.

    Soit $A \in  \mathcal A'$, $B$ et  $B'$ un encadrement de  $A$ de différence symétrique négligeable. Alors, en notant  $A=A \setminus  B\cup B$ on a $A \in  \left\{ Q\cup N, Q \in  \mathcal  Q, N \in  \mathcal  N \right\}\subset \bar{\mathcal  Q} $ car $A \setminus  B\subset B'\setminus  B$ est négligeable et $B \in  \mathcal  Q$. On en déduit finalement $\mathcal  Q'=\bar{\mathcal  Q}$.

    En posant $\mu(A)=\mu(B)=\mu(B')$, on étend la mesure à la tribu prévue, et cette définition ne dépend pas du choix de  $B, B'$ car pour un autre choix  $\hat{B},\hat{B'}$, on a toujours $\mu(B)=\mu(B')=\mu(\hat{B})=\mu(\hat{B'})$. C'est aussi le seul choix possible de valeur pour $\mu(A)$ par croissance de la mesure. On forme ainsi bien une mesure (exercice)
\end{proof}

\begin{thm}
    L'ensemble des mesurables de  $\lambda^\star$, noté  $\mathcal  M(\lambda^\star)$ est la tribu complétée de $\mathcal B(\R^d)$ par la mesure $\lambda$. On l'appelle tribu de Lebesgue.
\end{thm}

\begin{rem}
\[
    \mathcal  B(\R^d)\subsetneq \mathcal  M(\lambda^\star) \subsetneq \mathcal  P(\R^d)
\] 
La première inclusion stricte se montre par un argument de cardinalité, la seconde exploite l'axiome du choix.
\end{rem}

\begin{proof}[Démonstration du théorème]
    Si $A \in  \mathcal N$, alors il existe $B\supseteq A$ borélien de mesure nulle, donc  $\lambda^\star(A)\leq \lambda^\star (B)=0$ et on a vu dans un lemme que cela entraine que $A$ est mesurable. Donc,  $\bar{\mathcal  B(\R^d)}\subset \mathcal  M(\lambda^\star)$.

    Soit $A \in  \mathcal  M(\lambda^\star)$. Supposons que $\lambda^\star(A)<+\infty$ (sinon on s'y ramène en intersectant avec $]-N,N[$ et en faisant  $N\to +\infty$). Dans ce cas, pour tout $n\geq 1$, on peut trouver $(P_i^{(n)})_{i\geq 1}$ pavés ouverts qui recouvrent  $A$ avec  \[
        \sum_{i\geq 1}\mathrm{Vol}(P_i^{(n)})\leq \lambda^\star(A)+ \frac1{2^n}
    \] 
    Soit  \[
        B=\bigcap_{n\geq 1}\bigcup _{i\geq 1}P_i^{(n)} \in  \mathcal  B(\R^d)
    \] 
    On a bien $A\subset B$ et  $\lambda^\star(A)\leq \lambda(B)$. Puis, \[
        \forall  n\geq 1, \lambda(B) \leq  \lambda \left( \bigcup_{i\geq 1} P_i^{(n)} \right)\leq \sum_{i\geq 1} \mathrm{Vol}(P_i^{(n)})\leq \lambda^\star(A)+\frac1{2^n}
    \] 
    donc $\lambda(B)=\lambda^\star(A)$. De même, on trouve  $\hat{B} \subset A$ borélien de même mesure que $A$, ce qui montre que  $A \in  \bar{\mathcal  B(\R^d)}$
\end{proof}

\section{Propriétés de la mesure de Lebesgue}

\subsection{Invariance par translation}

On note $\lambda_d$ la mesure de Lebesgue en dimension  $d$.

 \begin{thm}
La mesure de Lebesgue est invariante par translation: \[
    \forall  A \in  \mathcal  B(\R^d), \forall x \in  \R^d, \quad  \lambda(x+A)=\lambda(A)
\]
De plus, toute autre mesure invariante par translation et finie sur les compacts est proportionnelle à $\lambda$
\end{thm}

\begin{proof}
    Le volume des pavés est invariant par translation, donc $\lambda^\star$ est invariante par translation et $\lambda$ aussi.

    Soit $\mu$ une mesure finie sur les compacts et invariante par translation. Alors, on pose  $c=\mu([0,1[^d)<+\infty$. Le pavé  $[0,1]^d$ s'écrit comme union disjointe de  $n^d$ translatés de  $[0,\sfrac1n[^d$, qui sont donc tous de mesure $\mu([0,\sfrac1n[^d)=\sfrac c{n^d}$, pour tous  $n\geq 1$.

    Tous les pavés de la forme $P=\prod [a_i, b_i[$ avec des  $a_i,b_i$ rationnels s'écrit comme union disjointe de translatés de  $[0,\sfrac1n[^d$ avec  $n$ un multiple des dénominateurs des  $a_i,b_i$. Ainsi,  $\mu$ et  $c \lambda$ coïncident sur les pavés à bornes rationnelles, qui engendrent les boréliens.
\end{proof}

\subsection{Régularité de la mesure de Lebesgue}

\begin{thm}
    Soit $A \in  \mathcal  B(\R^d)$. Alors \[
        \lambda(A)=\inf \left\{ \lambda(U), \quad  A\subset U, U \text{ouvert} \right\} 
    \]
    et \[
        \lambda(A)= \sup \left\{ \lambda(K), \quad  K\subset A, K\text{compact} \right\}.
    \]
    On parle de régularité respectivement extérieure et intérieure
\end{thm}

\begin{dfn}
    Si $f:(X, \mathcal  A)\longrightarrow (Y, \mathcal  B)$ est mesurable et $\mu$ est une mesure sur  $(X, \mathcal  A)$ alors on pose pour tous $B \in  \mathcal  B$ \[
        f_\star \mu(B)=\mu(f^{-1} (B))
    \]
    C'est la mesure image de $\mu$ par  $f$.
\end{dfn}

\begin{rem}
L'existence d'une mesure sur l'espace de départ donne l'existence d'une mesure sur l'espace d'arrivée. La réciproque est fausse en général.
\end{rem}

\begin{proof}[Démonstration du théorème]
~
\begin{itemize}
    \item \emph{Régularité extérieure}. Il est clair que $\lambda(A)\leq \lambda(U)$ si $U$ est un ouvert qui contient  $A$. Si $\lambda(A)=+\infty$, il n'y a rien à montrer. Sinon, on se donne  $ \epsilon>0$ et $(P_i)_{i\geq 1}$ une famille de pavés ouverts qui recouvrent $A$ et tels que  \[
            \sum_{i\geq 1} \mathrm{Vol}(P_i) \leq \lambda(A)+\epsilon
    \] 
    Alors, l'union de ces pavés, notée $\mathcal  U$, est un ouvert contenant $A$ et tel que  \[
        \lambda(\mathcal  U)\leq \sum_{i\geq 1}\lambda(P_i)\leq \lambda(A)+\epsilon
    \] 
    et c'est vrai pour tout $ \epsilon>0$.
\item \emph{Régularité intérieure}. Comme $\lambda(A\cap ]-N, N[^d) \xrightarrow[n\to+\infty]{}\lambda(A)$, on peut supposer $A\subset ]-N, N[^d$. On note  $B=[-N, N]^d \setminus  A$. Vu la régularité extérieure, il existe un ouvert $\mathcal  U$ qui contient $B$ tel que  $\lambda(\mathcal  U)\leq \lambda(A)+\epsilon$ pour $\epsilon>0$ fixé. Alors, $K=[-N, N]^d \setminus  \mathcal  U$ est un compact contenu dans $A$ et  $A \setminus  K\subset \mathcal U \setminus  B$ donc $\lambda (A \setminus  K)\leq  \lambda (\mathcal  U \setminus  B)\leq  \epsilon$ donc $\lambda(K)\geq \lambda(A)-\epsilon$
\end{itemize}
\end{proof}

\section{Lien avec l'intégrale de Riemann}


\begin{rem}
    En construisant l'intégrale de Riemann, on a découpé l'espace de départ (fonctions en escalier). Pour construire l'intégrale de Lebesgue, c'est l'espace d'arrivée qu'on a découpé (fonctions simples)
\end{rem}

\begin{dfn}
    Soient $a<b$ des réels. Une fonction $f:[a,b] \longrightarrow  \R$ est une fonction en escalier s'il existe une subdivision $a=x_0<\cdots <x_N=b$ telle que $f_{]x_i, x_{i+1}[}$ est constante égale à $y_{i+1}$ pour tout $i$. \index{fonction en escalier}
\end{dfn}

\begin{dfn}
    L'intégrale d'une fonction en escalier, notée $I(f)$ est définie par \[
        I(f)=\sum_{i=1}^N y_i(x_{i}-x_{i-1})
    \] 
\end{dfn}

\begin{dfn}
    Si $f:[a, b] \longrightarrow  \R$ est bornée on dit que $f$ est Riemann-intégrable lorsque  \[
        \sup_{\substack{h\leq f\\h\text{ en escalier }}} I(h)=\inf_{\substack{h\geq f\\h\text{ en escalier }}} I(h)
    \]
    et on note $I(f)$ cette valeur commune
\end{dfn}

\begin{rem}
On a déjà vu que les fonctions continues par morceaux Riemann-intégrables sont Lebesgue-intégrables là où les intégrales coïncident. On veut caractériser les fonctions Riemann-intégrables
\end{rem}

\begin{prop}
    Si $f:[a, b]\longrightarrow \R$ est Riemann-intégrable alors elle est mesurable par rapport à $(\R, \bar{\mathcal B(\R)})$ et \[
        \int_{[a,b]}f\diff \lambda=I(f)
    \] 
\end{prop}

On a en fait le résultat suivant, plus fort mais non démontré dans ce cours

\begin{thm}
    Une fonction bornée $f:[a,b] \longrightarrow \R$ est Riemann-intégrable si et seulement si l'ensemble de ses points de discontinuité est Lebesgue-négligeable.
\end{thm}

\begin{proof}[Démonstration de la propriété]
Par définition, on peut trouver $h_n \leq f\leq \hat{h}_n$ des fonctions en escalier avec \[
    \lim_{n\to +\infty} I(h_n)=\lim_{n\to +\infty} I(\hat{h}_n)=I(f)
\] 
Le max et le min de deux fonctions en escalier sont encore des fonctions en escalier, donc quitte à considérer $\max(h_1, \cdots , h_n)$ et $\min (\hat{h}_1, \cdots \hat{h}_n)$, on peut supposer que les suites de fonctions sont respectivement croissante et décroissante. Alors, ces deux suites ont des limites simples $h, \hat{h}$ boréliennes telles que $h\leq f\leq \hat{h}$. Puis, \[
h_0\leq h_n\leq f\leq \hat{h}_n\leq \hat{h}_0
\] 
donc toutes les fonctions sont bornées (par $h_0, \hat{h}_0$), et par convergence dominée, \[
    \int_{[a, b]}h_n\diff \lambda \xrightarrow[n\to+\infty]{}\int_{[a, b]}h\diff \lambda
\]
\[
    \int_{[a, b]}\hat{h}_n\diff \lambda \xrightarrow[n\to+\infty]{}\int_{[a, b]}\hat{h}\diff \lambda
\] 
or les deux quantités de gauche sont respectivement égales à  $I(h_n)$ et  $I(\hat{h}_n)$ qui convergent vers $I(f)$ donc  \[
    I(f)= \int_{[a,b]} h\diff \lambda=\int_{[a, b]}\hat{h}\diff \lambda
\]
Puis, on a $h\leq \hat{h}$ donc $\hat{h}-h\geq 0$ et $\int_{[a, b]}(\hat{h}-h)\diff \lambda=0$ donc $h=\hat{h}$ $\lambda$-presque partout sur $[a, b]$. Puis, \[ \left\{ x , \quad  f(x)\neq h(x) \right\} \subseteq \left\{ x, \quad  \hat{h}(x) \neq  h(x) \right\} \] et ce second ensemble est un borélien de mesure nulle donc le premier est un ensemble négligeable. On en déduit que $f=h$ en dehors d'un négligeable  $N$, donc  $f$ est mesurable dans la tribu de Lebesgue:  \[
h^{-1}(B)\setminus  N \subseteq f^{-1} (B) \subseteq h^{-1}(B)\cup N
\] 
\end{proof}

\begin{rem}
D'après le théorème, on peut intégrer plus de fonctions avec l'intégrale de Lebesgue qu'avec l'intégrale de Riemann.
\end{rem}

\section{Mesure de Stieltjes}

\begin{prop}
    Soit $F: \R \longrightarrow \R$ croissante et continue à droite. Alors il existe une unique mesure sur $(\R, \mathcal  B(\R))$ notée $\diff F$ dite mesure de Stieltjes\index{mesure de Stieltjes}\index{Stieltjes (mesure)} associée à  $F$ telle que  $\diff F(]a, b[)=F(b)-F(a)$, pour tous  $a<b$.
\end{prop}

\begin{ex}
Si $F=\id$, alors  $\diff F=\lambda$
\end{ex}

\begin{proof}[Idée de démonstration]
On utilise le lemme de Dynkin pour l'unicité, et on construit une mesure extérieure pour l'existence en posant \[
    \diff F^\star(A)=\inf \left\{ \sum (F(b_i)-F(a_i)), \qquad  A\subset \bigcup_{i\geq 1}]a_i, b_i] \right\} 
\] 
\end{proof}

\begin{defprop}
    On appelle mesure de Radon une mesure finie sur les compacts\index{mesure de Radon}\index{Radon (mesure)}. On a alors une bijection entre l'ensemble des mesures de Radon et l'ensemble des fonctions croissantes continues à droite qui s'annulent en $0$ donnée par  \[
        \mu \longmapsto \left( x \longmapsto \begin{cases}
                \mu(]0,x]) & \text{ si } x\geq 0\\
                -\mu(]x, 0])&\text{ sinon }
        \end{cases}
         \right)
    \] 
    de réciproque $F \longmapsto \diff F$
\end{defprop}
