\ifsolo
    ~

    \vspace{1cm}

    \begin{center}
        \textbf{\LARGE Espace \texorpdfstring{$\L^p$}{L^p}} \\[1em]
    \end{center}
    \tableofcontents
\else
    \chapter{Espace \texorpdfstring{$\L^p$}{L^p}}

    \minitoc
\fi
\thispagestyle{empty}

\begin{dfn}
    Soit $(X, \mathcal  A, \mu)$ un espace mesuré. On note, pour $f:X\longrightarrow \R$ mesurable et pour $p>1$, \[
        \|f\|_p= \left( \int_X |f|^p\diff \mu\right)^{\sfrac1p}
    \]
    et \[
        \|f\|_{\infty}=\inf \left\{ M\geq 0, \quad  |f|\leq M \;\mu-\text{pp} \right\} 
    \]
    On note pour $p \in  [1, +\infty]$, $\mathcal  L^p(X, \mathcal A, \mu)$ l'ensemble des fonctions mesurables $f:X \longrightarrow \R$ telles que $\|f\|_p<+\infty$. Sur cet ensemble, on définit $\sim$ la relation d'équivalence d'égalité $\mu$-presque partout et on appelle $\L^p(X, \mathcal A, \mu)=\mathcal  L^p(X, \mathcal  A, \mu) / \sim$.
\end{dfn}

\begin{rem}
On travaillera indifféremment avec des fonctions ou des classes d'équivalences avec la même notation.
\end{rem}

\section{Rappel sur les fonctions convexes}

\begin{dfn}
Si $I$ est un intervalle de  $ \R$, une application $\varphi : I \longrightarrow  \R$ est convexe si l'une des deux propriétés équivalentes suivantes est vérifiée: \[
    \forall  x, y \in  I, \forall t \in  [0,1], \varphi((1-t)x+ty)\leq (1-t)\varphi(x)+t\varphi(y)
\]
\[
    \forall  x,y,z \in  I, x<y<z \implies \frac{\varphi(y)-\varphi(x)}{y-x}\leq \frac{\varphi(z)-\varphi(x)}{z-x}\leq \frac{\varphi(z)-\varphi(y)}{z-y}
\]
\end{dfn}

\begin{rem}
Si $\varphi$ est  $\mathcal  C^1$, alors elle est convexe si et seulement si $\varphi'$ est croissante.
\end{rem}

\begin{prop}
Si $\varphi$ est convexe alors  \[
    \varphi(x)= \sup \left\{ ax+b, \quad  a,b \in  \R, ay+b\leq \varphi(x), \forall  y \in  I \right\} 
\] 
\end{prop}

\begin{prop}[Inégalité de Jensen\index{Jensen (inégalité)}]
    Soit $(X, \mathcal  A, \mu)$ un espace mesuré tel que $\mu(X)=1$, et $f:X\longrightarrow I$ intégrable avec $I$ un intervalle réel. Soit $\varphi$ convexe telle que  $\varphi\circ f$ est intégrable. Alors,  \[
        \varphi \left( \int_X f\diff \mu \right) \leq  \int_X \phi\circ f\diff \mu
    \] 
\end{prop}

\begin{proof}
On pose \[
y=\int_X f\diff \mu \leq \int_X \sup_Xf\diff\mu=\sup_X f\leq \sup I
\] 
car $f$ est à valeurs dans  $I$. On a aussi $\inf I\leq y$ donc $y \in  \bar{I} $. Si par exemple $y=\sup f$, alors  \[
\int_X f\diff\mu =\int_X \sup f\diff \mu
\] 
donc \[
    \int_X(\sup f-f)\diff \mu=0
\] 
et $f=\sup_X f$  $\mu$-pp et en particulier $y \in I$. De même, si $y=\inf_Xf$ alors $y \in  I$.

On a alors  \[
    \varphi(y)=\left\{ ay+b, \quad  a, b \in  \R, ax+b \leq \varphi(x), \forall  x \in  I \right\} 
\]
Soient $a, b \in  \R$ tels que $\forall  x \in  I, ax+b\leq \varphi(x)$. Alors \[
    \forall  x \in  I, af(x)+b\leq \varphi\circ f(x)
\] 
donc \[
    a\int_Xf\diff \mu+b\leq \int_X\varphi\circ f\diff \mu.
\] 
On passe au sup pour $a, b$ et cela conclut.
\end{proof}

\section{Inégalités de Hölder et Minkowski}

\begin{dfn}
    On dit que deux réels $p, q\in  [1, +\infty]$ sont des exposants conjugués si $\sfrac1p+\sfrac1q=1$.\index{exposants conjugués}
\end{dfn}

\begin{thm}[Inégalité de Hölder]
    \index{Hölder (inégalité)} Soient $p, q$ des exposants conjugués, et  $f,g:X\longrightarrow \R$ mesurables. Alors \[
    \|fg\|_1 \leq \|f\|_p \|g\|_q
    \] 
\end{thm}

\begin{rem}
\begin{itemize}
    \item Toutes ces quantités sont définies, éventuellement infinies
    \item On retrouve Cauchy-Schwarz avec $p=q=2$
    \item Il existe une version analogue pour les fonctions à valeurs complexes: les résultats obtenus sont toujours valables en considérant que les valeurs absolues sont des modules.
\end{itemize}
\end{rem}

\begin{proof}
    On peut supposer que $f $ et  $g$ ne sont pas nulles  $\mu$-presque partout, et que  $\|f\|_p, \|g\|_q<+\infty$ (sinon, il n'y a rien à montrer). Par convexité de l'exponentielle, \[
        \forall  x , y \in  \R, \forall  t \in  [0, 1], \qquad  \exp((1-t)y+tx)\leq (1-t)\exp(y)+t\exp(x)
    \] 
    En notant $u=\exp(tx)>0$ et $v=\exp((1-t)y)>0$ on obtient \[
        \forall  u, v>0, \forall  t \in  ]0, 1[, \qquad  uv\leq tu^{\sfrac1t}+(1-t)v^{\sfrac1{1-t}}
    \] 
    ce qui reste vrai si $uv=0$. En posant, pour  $x \in X$, \[
        u=\frac{|f(x)|}{\|f\|_p} \qquad  \text{ et  }\qquad  v= \frac{|g(x)|}{\|g\|_q}
    \]
    et $t=\sfrac1p$, on obtient  \[
        \frac{|f(x)g(x)|}{\|f\|_p \|g\|_q}\leq \frac{1}{p}\left(\frac{|f(x)|}{\|f\|_{p}}\right)^{p}+\frac{1}{q}\left(\frac{|g(x)|}{\|g\|_{q}}\right)^q
    \] 
    En intégrant, on trouve \[
    \frac{\|fg\|_1}{\|f\|_p \|g\|_q} \leq \frac1p+\frac1q=1
    \]
    d'où le résultat au moins si $p,q \in  ]1, +\infty[$. Sinon, on a simplement $|f(x)g(x)|\leq \|g\|_\infty |f(x)|$ $\mu$-presque partout donc en intégrant on retrouve le bon résultat.
\end{proof}

\begin{thm}[Inégalité de Minkowski\index{Minkowski (inégalité)}]
    Soient $f, g:X \longrightarrow \R$ (ou $\C$) deux fonctions mesurables et $p \in  [1, +\infty]$. Alors \[
    \|f+g\|_p\leq \|f\|_p+\|g\|_p
    \] 
\end{thm}

\begin{proof}
    Il n'y a rien à montrer si $\|f\|_p$ ou $\|g\|_p$ vaut $+\infty$, ou si  $\|f+g\|_p$=0. On suppose donc que ça n'est pas le cas. Dans le cas où $p=+\infty$, on a  $|f|\leq \|f\|_{\infty}$ $\mu$-presque partout donc  $|f+g|\leq \|f\|_\infty+\|g\|_{\infty}$ $\mu$-p.p. et donc  $\|f+g\|_\infty\leq \|f\|_\infty+\|g\|_\infty$. On suppose maintenant $1\leq p<+\infty$.  La convexité de $x \longmapsto |x|^p$ donne \[
        |f+g|^p\leq 2^{p-1}(\|f\|^p+ \|g\|^p)
    \] 
    donc \[
        \|f+g\|_p^p\leq 2^{p-1}(\|f\|_p^p+\|g\|_p^p)<+\infty
    \] 
    Puis, \[
    |f+g|^p=|f+g||f+g|^{p-1} \leq |f||f+g|^{p-1}+|g||f+g|^{p-1}
    \]
    En intégrant, l'inégalité de Hölder permet d'écrire, avec $\sfrac1q=1-\sfrac1p$\[
    \|f+g\|_p^p \leq \|f\|_p\|\,|f+g|^{p-1}\,\|_q + \|g\|_p\| \,|f+g|^{p-1}\,\|_q
    \]
    Or \[
    \| |f+g|^{p-1}\|_q = \left( \int_X |f+g|^{q(p-1)} d\mu\right)^{\sfrac1q} =  \|f+g\|_p^{\sfrac{p}{q}} \in ]0,+\infty[
    \] 
    On peut diviser par cette quantité et on obtient l'inégalité souhaitée.
\end{proof}

\section{L'espace de Banach \texorpdfstring{$\mathcal  L^p(X, \mathcal  A, \mu)$}{L^p(X,A,mu)}}

\begin{rem}
Vu l'inégalité de Minkowski, $\|\;\|_p$ définit une norme.
\end{rem}

\begin{thm}[Riesz]
    L'espace $(\mathcal L^p(X, \mathcal  A, \mu), \|\;\|_p)$ est un espace de Banach
\end{thm}

\begin{proof}
    Soit $(f_n)_{n\geq 1}$ une suite de Cauchy dans cet espace. On peut choisir une extraction $(n_k)$ telle que  \[
        \|f_{n_{k+1}}-f_{n_k}\|_p\leq \frac1{2^k}
    \]
    Si $p=+\infty$ alors  $A_k= \left\{ x \in  X, \quad  |f_{n_{k+1}}(x)-f_{n_k}(x)|>\sfrac1{2^k} \right\} $ est de mesure nulle. Puis, sur $(\bigcup_{k\geq 1}A_k)^c$, $(f_{n_k})$ converge uniformément vers \[f\defeq f_{n_1}+\sum_{k\geq 1}(f_{n_{k+1}}-f_{n_k})\] et on a bien  $f \in  \mathcal  L^\infty(X, \mathcal A, \mu)$.

    Si $p<+\infty$, on note $g_k=f_{n_{k+1}}-f_{n_k}$, et considérons:
    \begin{align*} \int_X\left( \sum_{k\geq1}|g_k|\right)^p\diff\mu &=\lim_{n\to +\infty}\int_X\left( \sum_{k=1}^n|g_k|\right)^p\diff\mu \\&= \lim_{n\to +\infty}\||g_1|+\dots +|g_n|\|_p^p \\&\leq \lim_{n\to +\infty}\left( \sum_{k=1}^n \|g_k\|_p\right)^p<+\infty \end{align*}
donc $\mu$-presque partout, $\sum_{k\geq 1}|g_k|<+\infty$.
En dehors d'un mesurable $A$ de mesure nulle, \[ f\defeq f_{n_1}+\sum_{k\geq 1}(f_{n_{k+1}}-f_{n_k})\]
est absolument convergente. Donc $\forall x\notin A, f_{n_k}(x) \xrightarrow[k\to+\infty ]{} f(x)$. Il reste à montrer que $(f_{n_{k}})$ converge vers $f$ dans  $ \mathcal  L^p$. Le lemme de Fatou donne \[
\int_X|f|^p\diff\mu =\int_X \liminf_{k\to \infty}|f_{n_k}|^p\diff\mu \leq \liminf_{k\to \infty}\int_X|f_{n_k}|^p\diff\mu <+\infty
\] 
donc $f \in  \mathcal  L^p$. Pour les mêmes raisons, \begin{align*}
\int_X|f_n-f|^p\diff\mu &= \int \liminf_{k\to\infty}|f_n-f_{n_k}|^p\diff\mu \\
&\leq \liminf_{k\to \infty}\int_X |f_n-f_{n_k}|^p\diff\mu \\
&= \liminf_{k\to\infty}\|f_n-f_{n_k}\|_p^p
\end{align*}
Si $\epsilon>0$, il existe $n_0$ tel que cette dernière limite est majorée par  $\epsilon$ pour tous $n\geq n_0$. Finalement, \[
\|f_n-f\|_p \xrightarrow[n\to+\infty]{}0
\] 
\end{proof}

\begin{rem}
Dans la démonstration, on a vu les deux résultats suivants
\end{rem}

\begin{thm}
    Si $(f_n)_n$ est une suite de $\mathcal  L^p$ qui converge vers $f \in  \mathcal  L^p$, alors il existe une sous-suite $(f_{n_{k}})$ qui converge presque partout vers $f$
\end{thm}

\begin{prop}
    Si $(f_n)$ est bornée dans  $\mathcal  L^p$ et converge presque partout vers une limite $f$, alors  $f \in \mathcal  L^p$
\end{prop}

\begin{rem}
Il n'est pas vrai en général que $f_n \longrightarrow f$ dans $\mathcal  L^p$.
\end{rem}

\begin{exo}
    Si $\mu$ est finie,  $(f_n)$ converge  $\mu$-presque partout vers une limite  $f$ et  $(f_n)$ est bornée dans  $ \mathcal  L^r$, alors $f_n$ converge vers  $f$ dans  $\mathcal  L^p$ pour tous $p \in  [1, r[$\footnotemark
\end{exo}

\footnotetext{Penser à l'inégalité de Hölder}

\section{Théorèmes de densité dans les espaces \texorpdfstring{$\mathcal L^p$}{L^p}}

\begin{dfn}
    On dit d'une mesure $\mu$ sur un espace  $(X, \mathcal  B(X))$ où $X$ est un espace topologique muni de sa tribu borélienne qu'elle est  \emph{extérieurement régulière}\index{régularité extérieure} si \[
        \forall  A \in  \mathcal  B(X), \quad  \mu(A)= \inf \left\{ \mu(\mathcal  U), \quad  A\subset \mathcal  U, \;\mathcal  U\text{ ouvert } \right\} 
    \] 
\end{dfn}

\begin{thm}
\begin{enumerate}
    \item Soit $(X, \mathcal  A, \mu)$ un espace mesuré et $p \in  [1, +\infty]$, alors les fonctions simples qui sont intégrables sont denses dans $\mathcal  L^p(X, \mathcal  A, \mu)$.
    \item Soit $(X,d)$ un espace métrique, et  $\mu$ une mesure extérieurement régulière sur  $(X, \mathcal B(X))$, et $p \in  [1, +\infty[$ alors l'espace des fonctions lipschitziennes bornées qui sont dans $\mathcal  L^p$ est dense dans $\mathcal  L^p$
    \item Soit $(X, d)$ un espace métrique séparable, localement compact\footnotemark et  $\mu$ une mesure de Radon sur  $(X, \mathcal  B(X))$. Alors l'espace des fonctions lipschitziennes à support compact est dense dans $\mathcal  L^p$, pour tous $p \in  [1, +\infty[$
\end{enumerate}
\end{thm}

\footnotetext{$\forall  x \in  X, \forall  V \in  \mathcal  V(x), \exists U \in  \mathcal  V(x), V\subseteq \bar{U} \text{ compact}$}

\begin{cor}
    Toute fonction de $\mathcal  L^p(\R^d, \mathcal  B(\R^d), \mu)$ avec $\mu$ une mesure de Radon est limite dans  $ \mathcal  L^p$ de: \begin{itemize}
        \item Des fonctions continues à support compact
        \item Des fonctions en escalier à support compact
    \end{itemize}
\end{cor}

\begin{proof}
\begin{enumerate}
    \item En écrivant $f\in \mathcal L^p$ sous la forme $f=f^+-f^-$, on a $f^+,f^-\in \mathcal L^p$. et donc il suffit de traiter le cas $f\geq 0$.
        Or on sait que dans ce cas, $f$ est limite croissante des fonctions simples \[\varphi_n\defeq 2^{-n}\lfloor 2^nf\rfloor  \leq f\]
        La croissance est même uniforme, puisque \[0\leq \|f-\varphi_n\|_{\infty}\leq  \sfrac{1}{2^n} \xrightarrow[n\to+\infty]{} 0\]
et pour $p\in [1;\infty[$, on a $|\varphi_n-f|^p \xrightarrow[n\to+\infty]{} 0$ et $\leq |f|^p$ qui est intégrable
donc par convergence dominée $\|\varphi_n-f\|_p \xrightarrow[n\to\infty]{} 0$
\end{enumerate}
\end{proof}


