\ifsolo
    ~

    \vspace{1cm}

    \begin{center}
        \textbf{\LARGE Mesures complexes} \\[1em]
    \end{center}
    \tableofcontents
\else
    \chapter{Mesures complexes}

    \minitoc
\fi
\thispagestyle{empty}

\section{Définitions et propriétés de base}

\begin{dfn}
    Soit $(X, \mathcal  A)$ un espace mesurable. Une mesure complexe\index{mesure complexe} sur $(X, \mathcal  A)$ est une fonction $\nu: \mathcal  A \longrightarrow  \C$ telle que \begin{itemize}
        \item $\nu(\emptyset)=0$
        \item Si  $(A_i)_{i \in  I}$ est une famille dénombrable d'éléments de $\mathcal  A$ deux à deux disjoints, \[
                \nu \left( \bigcup_{i \in  I} A_i\right) = \sum_{i \in  I}\nu(A_i)
        \] 
        où la convergence est supposée indépendante du choix d'ordre sur $I$
    \end{itemize}
\end{dfn}

\begin{rem}
On admet que l'indépendance par permutation de la convergence entraine la convergence absolue.
\end{rem}

\begin{ex}
    Soit $(X, \mathcal  A, \mu)$ un espace mesuré avec $\mu$ une mesure usuelle positive, et soit  $f \in  \L^1_{\C}(X, \mathcal  A, \mu)$. Alors \[
        \nu(\mathcal  A)=\int_A f\diff \mu
    \] 
    est une mesure complexe (notée $f\diff \mu$). Clairement, $\nu(\emptyset)=0$ et  \[
        \nu \left(\bigcup_{n \in  \N}A_n\right)=\int_{X}f\diff \mu \1_{\bigcup A_i}=\int_{X}\sum_{n \in  \N}\underbrace{f\1_{A_n}}_{\mathclap{\text{dominée par }|f|\text{ intégrable }}}\diff \mu = \sum_{n \in  N}\int_{A_n} f\diff \mu=\sum_{n \in  \N}\nu(A_n)
    \] 
\end{ex}

\begin{dfn}
    Soit $\nu$ une mesure complexe. La mesure de variation totale de  $\nu$ \index{variation totale (mesure)} est la fonction  $|\nu|:\mathcal  A \longrightarrow \R_+\cup \left\{ \infty \right\} $ définie par \[
        |\nu|(A) = \sup \left\{ \sum_{i \in  N}|\nu(A_i)|, \quad  (A_i)_{i \in  \N} \text{ deux à deux disjoints, d'union } A\right\} 
    \] 
\end{dfn}

\begin{prop}
Si $\nu$ est une mesure complexe, alors  $|\nu|$ est une mesure.
\end{prop}

\begin{proof}
    On a clairement $|\nu|(\emptyset)=0$. Soit  $(B_j)_{j \in  \N}$ des éléments de $\mathcal  A$ deux à deux disjoints.

    Si $|\nu|(B_j)=+\infty$ pour un  $j$ alors on peut repartitionner  $B_j$ en  $\bigcup_{i \in  \N} A_{j,i}$ avec $\sum_{i \in  \N}|\nu|(A_{j,i})$ arbitrairement grand, et \[
        \sum_{k\neq j}|\nu(B_k)| + \sum_{i \in  I}|\nu(A_{j,i})|
    \] 
    est arbitrairement grand donc $|\nu|(\bigcup B_j)=+\infty$. Sinon,  pour tout $j \in  \N$, $|\nu|(B_j)<+\infty$ et pour  $\epsilon>0$, il existe des partitions $(A_{j,i}) _{i \in  \N}$ des $B_j$ avec  \[
        \sum_{i \in  \N}|\nu(A_{j, i})| \geq |\nu|(B_j)-\frac\epsilon{2^{j+1}}
    \] 
    et $(A_{j,i})_{j,i \in  \N}$ est une partition dénombrable de $\bigcup_{j \in  \N}B_j$ avec \[
        \sum_{j, i \in  \N}|\nu(A_{j,i})|\geq \sum_{j \in  \N}|\nu|(B_j)-\epsilon
    \] 
    donc \[
        |\nu| \left(\bigcup_{j \in  \N}B_j\right)\geq \sum_{j \in  \N}|\nu|(B_j)
    \] 
    Puis si $(A_i)_{i \in  \N}$ est une partition de $\bigcup_{j \in  \N}B_j$ en éléments de $\mathcal  A$, alors \[
        \sum_{i \in  \N}|\nu(A_i)|=\sum_{i \in  \N}\left|\sum_{j \in  \N}\nu(A_i\cap B_j)\right|\leq \sum_{j \in  \N}|\nu|(B_j)
    \] 
    d'où le résultat en passant au sup
\end{proof}

\begin{rem}
    $\nu$ est la plus petite mesure positive vérifiant  $\forall  A \in  \mathcal  A, |\nu(A)|\leq |\nu|(A)$.
\end{rem}

\begin{thm}
La variation totale $|\nu|$ d'une mesure complexe $\nu$ est une mesure finie
\end{thm}

\begin{lmm}
Soit $z_0, \cdots , z_N \in  \C$ et $S=\sum_{i=0}^N|z_i|$. Alors, il existe $I\subset \llbracket 1,N \rrbracket  $ tel que \[
\left|\sum_{i \in  I}z_i\right| \geq \frac S6
\] 
\end{lmm}

\begin{proof}
On partitionne $ \C$ en quadrants $Q_i$. Il existe  $i \in  \llbracket 1,4 \rrbracket $ tel que \[
    \sum_{\substack{1\leq j\leq N\\ z_j \in  Q_i}}|z_j| \geq \frac S4
\]
Quitte à effectuer une rotation, on prend $i=1$. Alors, comme  $\forall  z \in  Q_1, \Re(z)\geq \frac1{\sqrt 2}|z|$, on a \[
    \left|\sum_{\substack{1\leq j\leq N\\ z_j \in  Q_i}}z_j\right|\geq \sum_{\substack{1\leq j\leq N\\ z_j \in  Q_i}} \Re(z_j) \geq  \frac1{\sqrt 2}\sum_{\substack{1\leq j\leq N\\ z_j \in  Q_i}}|z_j| \geq \frac S{4\sqrt 2} \geq \frac S6
\] 
\end{proof}

\begin{lmm}
    Si $|\nu|(A)=+\infty$, il existe  $B, C \in  \mathcal  A$ disjoints, $A=B\cup C$,  $|\nu|(B)=+\infty,|\nu(C)|\geq 1$
\end{lmm}

\begin{proof}
Comme $|\nu|(A)=\infty$, il existe une partition de $A$, $(A_i)_{i\in \N}$ avec $\sum_{i\in \N}|\nu(A_i)|> 6(1+|\nu(A)|)$.
Alors il existe $N$ tel que \[ \sum_{i=0}^N|\nu(A_i)|\geq 6(1+|\nu(A)|)\]
Par le lemme 1, il existe $I\subseteq \{0,\dots,N\}$ tel que \[ \Big| \sum_{i\in I}\nu(A_i)\Big| \geq 1+|\nu(A)| =\Big|\nu\Big(\underbrace{\bigcup_{i\in I}A_i}_{=G}\Big)\Big|\] notons $H=A\setminus G$.
Alors $|\nu(G)|\geq 1$, $|\nu(H)|=|\nu(A)-\nu(G)|\geq \big| |\nu(A)|-|\nu(G)|\big|\geq 1+|\nu(A)|-|\nu(A)|\geq 1$.
Donc $A=G\cup H$ disjoints.
$|\nu(G)|\geq 1 ,|\nu(H)|\geq 1$
mais aussi $\infty=|\nu|(A)=|\nu|(G)+|\nu|(H)$ donc $G$ ou $H$ est de variation totale infinie.
\end{proof}

\begin{proof}[Preuve du théorème]
Si $|\nu|(X)=+\infty$
En appliquant successivement le lemme 2, on construit une suite $(C_n)_{n\geq 0}$ d'ensembles deux à deux disjoints avec $|\nu(C_n)|\geq 1$
Alors $\nu(\bigcup_{n\geq 0}C_n)=\sum_{n\in\N}\nu(C_n)$ contradiction.
\end{proof}

\begin{rem}
    Une mesure complexe peut s'écrire $\Re(\nu)+i\Im(\nu)$ avec $\Re(\nu)$ et $\Im(\nu)$ des mesures complexes à valeurs dans $ \R$. On parle alors de mesures signées\index{mesure!signée}. Si $\xi$ est une telle mesure signée et $|\xi|$ est sa mesure de variation totale alors les mesures $\xi^+, \xi_-$ sont des mesures positives finies et $\xi=\xi^+-\xi^-$, $|\xi|=\xi^++\xi^-$. Cette décomposition s'appelle la décomposition de Jordan\index{Jordan (décomposition)} de $\xi$.
\end{rem}

\section{Théorème de Radon-Nikodyn}

\begin{dfn}
    Soit $\mu$ une mesure positive et  $\nu$ une mesure positive ou complexe. On dit que  $\nu$ est absolument continue par rapport à  $\mu$\index{absolue continuité (mesure)}\index{mesure!absolument continue} si  \[
        \forall  A \in  \mathcal  A, \mu(A)=0 \implies \nu(A)=0
    \] 
    On note alors $\nu \ll \mu$
\end{dfn}

\begin{rem}
Si $\nu \ll \mu$, alors  $|\nu|\ll\mu$
\end{rem}

\begin{dfn}
    Une mesure positive ou complexe est concentrée sur $A \in  \mathcal  A$\index{mesure!concentrée} \[
        \forall  B \in  \mathcal A, \quad  B\cap A=\emptyset \implies \nu(B)=0
    \] 
    Si $\mu$ est positive et  $\nu$ est positive ou complexe, on dit que  $\mu$ et  $\nu$ sont étrangères ou singulières \index{mesures étrangères} s'il existe  $A \in  \mathcal  A$ tel que $\mu$ est concentrée sur  $A$ et  $\nu$ est concentrée sur  $A^c$. On note  $\mu\bot\nu$
\end{dfn}

\begin{rem}
    Si $\mu, \nu$ sont positives, on a $\mu\bot \nu$  si et seulement si $ \exists  A \in  \mathcal  A, \nu(A)=0, \mu(A^c)=0$
\end{rem}

\begin{ex}
    Si $\mu$ est positive et  $f \in  \mathcal  L^1_{\C}(\mu)$, alors $\nu=f\diff \mu\ll\mu$
\end{ex}

\begin{thm}[Radon-Nikodyn\index{Radon-Nikodyn (théorème)}]
Si $\mu$ est une mesure  $\sigma$-finie et  $\nu$ est une mesure  $\sigma$-finie ou complexe telle que  $\nu\ll\mu$, alors il existe  $g\geq 0$ mesurable, unique à égalité $\mu$-presque partout, telle que  $\mu=g\diff \mu$
\end{thm}

\begin{thm}[Décomposition de Lebesgue]
    Soit $\mu$ une mesure $\sigma$-finie et  $\nu$ une mesure  $\sigma$-finie ou complexe, Alors, il existe un unique couple  $(\nu_a, \nu_s)$ de mesures  $\sigma$-finies ou complexes tel que  $\nu_a \ll \mu$,  $\nu_s\bot \mu$ et  $\nu=\nu_a+\nu_s$
\end{thm}

\begin{proof}[Preuve des théorèmes]
    Il y a quatre cas \begin{enumerate}
        \item 
            On suppose que $\mu,\nu$ sont deux mesures positives, finies, et que $\nu\leq \mu$:  $\forall A\in \mathcal{A},\nu(A)\leq \mu(A)$.
Pour $f\in \mathcal{L}^2(X,\mathcal{A},\mu)$, on note \[\varphi(f)= \int_Xf\diff\nu\] qui est bien défini car $f\in \mathcal{L}^2(X,\mathcal{A},\nu)$ car $\nu\leq \mu$.
Alors,
\begin{align*}
|\varphi(f)|&\leq \int_X |f|\diff\nu \leq \int_X 1\cdot |f|\diff\mu\\
&\underset{\text{C-S}}{\leq} \sqrt{\mu(X)} \left( \int_X |f|^2 \diff\mu\right)^{\sfrac12}\\
&\leq \sqrt{\mu(X)}\|f\|_2,
\end{align*}
donc $\varphi: \mathcal  L^2(X, \mathcal A, \mu)\longrightarrow \R$ est continue.
    Un théorème dû à Riesz montre que si $(H,\scalar{\;}{\;} )$ est un espace de Hilbert et si $\varphi:H\to \R$ est une application linéaire continue, il existe un unique $h\in H$
    tel que \[\forall x\in H ,\varphi(x)=\scalar{h}{x} .\]
Ainsi, il existe une unique application $g\in \mathcal{L}^2 (X,\mathcal{A},\mu)$ telle que $\forall f\in \mathcal{L}^2(X,\mathcal{A},\mu)$, \[\varphi(f)=\int_Xf\diff\nu=\int_Xfg\diff\mu = \scalar{y}{g}  _{\mathcal{L}^2(\mu)}\]
Puis, comme $\forall A\in \mathcal{A},\1_A\in \mathcal{L}^2(\mu)$ puisque $\mu$ est finie, on obtient
\[\forall A\in \mathcal A , \nu(A)=\int_Ag\diff\mu\]
Donc $\nu=g\diff\mu$.
Vérifions que $g\in [0,1]$ $\mu$ presque partout.
Pour cela, on se donne $\varepsilon >0$, et
\begin{align*}
\mu(\{g\geq 1+\varepsilon\})& \geq \nu(\{g\geq 1+\varepsilon\})\\
                            &= \int_{\{g\geq 1+\varepsilon\}}g\diff\mu \\&\geq (1+\varepsilon)\mu(\{g\geq 1+\varepsilon\})
\end{align*}
donc $g< 1+\varepsilon$ $\mu$-presque partout donc $g\leq 1$ $\mu$-presque partout.
De façon similaire, on a $g\geq 0$ $\mu$-presque partout.
Sans perdre de généralité, on peut supposer que $g\in [0,1]$ partout.
\item 
    Soient $\mu$ et $\nu$ des mesures positives finies.
On remarque que $\nu\leq \mu+\nu$ donc le cas précédent donne l'existence d'une fonction mesurable $g:X\to [0,1]$ telle que $\nu= g\diff(\mu+\nu)$, c'est à dire que pour toute fonction $f$ positive mesurable ou bornée, 
\[\int_X f\diff\nu= \int_X fg\diff(\mu+\nu)\]
et donc  \[\int_X f(1-g)\diff\nu =\int_Xfg \diff\mu.\]
Soit $g^{-1}(\left\{ 1 \right\} )$ et 
$f= \1_N$. On a \[0=\int_X \1_N(1-g)\diff\nu = \int_X \1_N g\diff\mu = \mu(N)\]
De même, quitte à remplacer $f$ par la fonction $f=\frac{\1_{N^c}}{1-g}$, 
    \begin{align*}
    \int_X \1_{N^c} \frac{{1-g}}{{1-g}}\diff\nu &= \int_Xf \frac{g}{1-g}\diff\mu \1_{N^c}\\
&= \int_X f h\diff\mu
    \end{align*} 
où $h=\frac{g}{1-g}\1_{N^c}$, 
donc $\1_{N^c}\diff\nu=h \diff\mu$. On peut donc écrire $\nu= \1_{N^c}\diff\nu +\1_N\diff\nu = h\diff\mu + \1_N \diff\nu = \nu_a + \nu_s$
où $\nu_a \ll \mu$ et $\nu_s \bot \mu$.

Cela donne le théorème de Radon-Nikodyn dans le cas $\mu,\nu$ positives fines. Montrons l'unicité. Supposons que $\nu= \nu_a+\nu_s = \nu_a' + \nu_s'$
Alors $\nu_a- \nu_a'= \nu_s'-\nu_s$, et il existe $N,N'\in \mathcal{A}$, avec $\mu(N)=\mu(N')=0$ et $\nu_s'((N')^c)=\nu_s (N^c)=0$.
Alors si $A\in \mathcal{A}$,
\begin{align*}
(\nu_s' -\nu_{s})(A) &= (\nu_s' -\nu_s)(A\cap (N\cup N' ))\\
&= (\nu_a- \nu_a' )(A\cap (N\cup N' ))\\
&=0 
\end{align*}
car $\mu(A\cap (N\cup N' ))\leq \mu(N\cup N' )=0$, donc $\nu_s=\nu_s'$ et $\nu_a=\nu_a'$.
De même, si $h\diff\mu=h'\diff\mu=\nu$ avec $h,h'\geq 0$, alors
\[ \int_{h' >h}h' \diff\mu = \nu(\{h' >h\})= \int_{h' >h}h\diff\mu\]
donc \[ \int_{h'>h}(h'-h)\diff\mu =0,\] d'où $\mu(\{h'>h\})=0$
et $h'\leq h$ $\mu$-presque partout, donc par symétrie des rôles $h=h'$
\item 
    Si $\mu$ et $\nu$ sont $\sigma$-finies,
il existe une partition mesurable $(X_n)_{n\geq 0}$ de $X$ avec $\mu(X_n)<+\infty$ et $\nu(X_n)<+\infty$.
On note \[
    \begin{cases}
    \mu_n=\1_{X_n}\diff\mu\\ \nu_n=\1_{X_n}\diff\nu
    \end{cases}\]
Par le cas précédent, $\nu_n = \nu_a^{(n)}+\nu_s^{(n)}=g_n \diff\mu_n +\nu_s^{(n)}$ avec $g_n\geq 0$ mesurable, $\nu_s^{(n)}\bot \mu_n$.
Comme $\mu_n$ est concentrée sur $X_n$, on peut supposer que $g_n=0$ en dehors de $X_n$.
On pose alors $g= \sum_{n\geq0} g_n$ et $\nu_s =\sum_{n\geq 0}\nu_s^{(n)}$
et $\nu=g\diff\mu+\nu_s$, avec $\nu_s=\bot \mu$. L'unicité se montre comme ci-dessus.
\item 
    Si $\mu$ est $\sigma$ finie et $\nu$ est complexe,
on écrit $\nu=\nu_1-\nu_2 +i(\nu_3-\nu_4)$ avec $\nu_1,\dots,\nu_4$ positives finies, et on applique la décomposition précédente à chaque terme.
    \end{enumerate}
\end{proof}

\begin{prop}
Soient $\mu$ une mesure positive et $\nu$ une mesure complexe. Il y a équivalence entre  \begin{itemize}
    \item $\nu\ll\mu$
    \item  $\forall  \epsilon>0, \exists  \delta>0, \forall  A \in  \mathcal  A, \mu(A)\leq \delta \implies |\nu(A)|\leq \epsilon$
\end{itemize}
\end{prop}

\begin{proof}~
\begin{itemize}
    \item $(\impliedby )$ Si $A \in  \mathcal  A$ est tel que $\mu(A)=0$, alors  $|\nu(A)|<\epsilon$ pour tout $ \epsilon>0$ donc $\nu(A)=0$
    \item  $(\implies )$ Procédons par l'absurde: \[
            \exists  \epsilon_0>0, \forall  n \in  \N, \exists  A_n \in  \mathcal  A, \mu(A_n)\leq \frac1{2^n}\text{ et } |\nu(A_n)| >\epsilon_0
    \] 
    On pose \[B_n=\bigcup_{k\geq n}A_k\] de sorte que $\mu(B_n)\leq \frac1{2^{n-1}}$ et \[
    B = \bigcap_{n\geq 0}B_n
    \] 
    de sorte que $\mu(B)=\lim_{n\to +\infty} \mu\left( B_n \right) =0$. Or, \[
        |\nu|(B)=\lim_{n\to +\infty}|\nu|(B_n)\geq \lim_{n\to +\infty} |\nu|(A_n)\geq \lim_{n\to +\infty} |\nu(A_n)|\geq \epsilon_0
    \] 
    donc $|\nu|(B)>0$, ce qui est absurde.
\end{itemize}
\end{proof}

\begin{thm}
    Soit $\nu$ une mesure complexe sur un ensemble mesurable $(X, \mathcal  A)$. Alors il existe une fonction mesurable $h : X \longrightarrow \C$, avec $|h|=1$, telle que $\nu=h\diff |\nu|$
\end{thm}

\begin{proof}
    $\nu\ll|\nu|$ donc par le théorème de Radon-Nikodyn, il existe  $h \in  \mathcal  L^1_{\C}(|\nu|)$. Il reste à vérifier que $\nu$-presque partout,  $|h|=1$. 
    Soit $A_r=\{x, \quad  |h(x)|<r\}= \{|h|<r\}$ où $r\in ]0,1[$. Soit $(B_n)_{n\geq 0}$ une partition mesurable de $A_r$.
    \begin{align*}
        \sum_{n\geq 0}|\nu(B_n)| &= \sum_{n\geq 0}\left| \int_{B_n}h\diff|\nu|\right|\\
&\leq \sum_{n\geq0}\int_{B_n}|h|\diff|\nu|\leq r|\nu|(A_r)
\end{align*}  
Donc en passant au sup sur $B\cap B_n$, $|\nu|(A_r)\leq r|\nu|(A_r)$.
Comme $r<1$, cela implique que $|\nu|(A_r)=0$ donc $|h|\geq 1$ $|\nu|$ presque partout.
On remarque alors que si $|\nu|(A)>0$, alors \[ \left| \frac{1}{|\nu|(A)}\int_Ah\diff\nu\right| = \frac{|\nu(A)|}{|\nu|(A)}\leq 1.\]
par propriété de moyenne,  $|h|\leq 1$ $|\nu|$-presque partout %(cf Rudin)
\end{proof}

\begin{cor}
    Soit $\mu$ une mesure  $\sigma$-finie et  $f \in  \mathcal  L^1(\mu)$. Si $\nu=f\diff\mu$ alors  $|\nu|=|f|\diff \mu$
\end{cor}

\begin{proof}
Par le théorème précédent, on a $f\diff \mu=\nu=h\diff |\nu|$ avec $|h|=1$ donc pour toute fonction $g$ mesurable bornée, \[
\int_X gf\diff\mu = \int_X gh\diff |\nu|
\] 
et donc avec $g= \bar{h}$, \[
\int_X \bar{h}f\diff\mu=\int_X |h|^2 \diff |\nu|=\int_X \diff \nu
\]
donc $|\nu|=\bar{h} f\diff\mu$, donc $\bar{h} f\geq 0$ $\mu$-presque partout et comme  $|h|=1$,  $\bar{h}f=|f|$ $\mu$-presque partout et donc  $|\nu|=|f|\diff \mu$.
\end{proof}

\begin{thm}[Hahn\index{Hahn (théorème)}]
    Soit $\xi$ une mesure signée, alors il existe  $A, B \in  \mathcal  A, A\cup B=X$ et $A\cap B=\emptyset$ tels que  $\forall  C \in  \mathcal  A, \xi_+(C)=\xi(C\cap A), \xi^-(C)=-\xi(C\cap B)$
\end{thm}

\begin{rem}
$\xi^+$ est concentrée sur  $A$,  $\xi^-$ est concentrée sur  $B$ donc  $\xi^+\bot \xi^-$.
\end{rem}

\begin{proof}
    Comme $\xi$ est une mesure complexe,  $\xi=h\diff |\xi|$ avec  $|h|=1$.  Les valeurs de $\xi$ sont réelles donc  $h$ est aussi à valeurs réelles presque partout (spdg partout). Puis, $A = h^{-1}(\left\{ 1 \right\} ), B = h^{-1}(\left\{ -1 \right\} )$ conviennent.
\end{proof}

\begin{prop}
La décomposition de Jordan est minimale au sens où si $\xi$ est une mesure signée de la forme  $\mu_1-\mu_2$ avec  $\mu_1, \mu_2$ des mesures positives, alors  $\xi_+\leq \mu_1, \xi^-\leq \mu_2$
\end{prop}

\begin{rem}
    On a noté $\nu = g\diff \mu$. On note souvent  $g=\frac{\diff \nu}{\diff \mu}$, et l'on dit que  $g$ est la dérivée de Radon-nokodyn de  $\nu$ par rapport à  $\mu$
\end{rem}

\section{Dualité \texorpdfstring{$\L^p$ -- $\L^q$}{L^p - L^q}}

Soit $\mu$ une mesure positive sur  $(X, \mathcal  A)$, $p \in  [1, +\infty]$, $q$ son exposant conjugué. Si  $g \in  \L^q_{\C}(X, \mathcal  A, \mu)$, on note \[
    \Phi_g:f \in  \L^p_{\C}(X, \mathcal  A, \mu)\longmapsto \int_X gf\diff\mu
\] 
et cette quantité est bien définie par l'inégalité de Hölder. L'application $\Phi_g$ est une forme linéaire sur  $\L^p_{\C}(\mu)$, et \[
    |\Phi_g(f)|\leq \|g\|_q \|f\|_p
\] 
donc $\Phi_g$ est continue de norme d'opérateur  $\nop\Phi_g\nop\leq \|g\|_q$

\begin{dfn}[Notation]
    On note $(\L^p(\mu))'$ le dual topologique de $\L^p(\mu)$, c'est à dire l'espace des formes linéaires continues sur $\L^p(\mu)$
\end{dfn}

\begin{thm}
Soit $\mu$ une mesure $\sigma$-finie et  $p \in  [1, \infty[$. L'application \[
\begin{array}{rrcl}
    \Phi:& \L^q(\mu) & \longrightarrow & (\L^p(\mu))' \\
    & g & \longmapsto & \displaystyle \Phi_g
\end{array}
\] 
est une isométrie surjective.
\end{thm}

\begin{proof}[Preuve dans le cas d'une mesure finie]
Supposons que $\mu$ est finie.  
    Soit $\phi \in (\L^p(\mu))'$. Si $A\in \mathcal{A}$ alors $\1_A\in \L^p(\mu)$ car $\mu$ est finie
    et $0\leq \1_A\leq 1$.
    On définit $\nu(A)= \phi(\1_A)$ de sorte que  $\nu$ est une mesure complexe.
    En effet, $\nu(\emptyset)=\phi(\1_\emptyset)= \phi(0)=0$ 
    et si $(A_i)_{i \in  I}$, avec $I$ dénombrable, \[\forall i,j\in I,A_i\cap A_j=\emptyset\] et \[\forall i\in I,A_i\in \mathcal{A}\]
    Alors,
    \[ \forall x\in X, \quad  \1_{\bigcup_{i\in A_i}} (x)= \lim_{k\to \infty}\sum_{j=0}^k \1_{A_{i_j}}(x)\] où $I=\{i_0,i_1,\dots \}$.
    La somme est dominée par  $1\in \L^p(\mu)$ donc par convergence dominée, \[\displaystyle\sum_{j=0}^k \1_{A_{i_j}}\xrightarrow[k\to +\infty]{\L^p(\mu)} \1_{\bigcup_{i\in I}A_i}.\]
    Puis,
        \[\sum_{j=0}^k\nu(A_{i_j}) = \phi\left(\sum_{j=0}^k \1_{A_{i_j}}\right)\xrightarrow[k\to \infty]{} \phi\left(\1_{\bigcup_{i\in I}A_i}\right)= \nu\left(\bigcup_{i\in I}A_i\right)\]
        donc \[\nu\left(\bigcup_{i\in I}A_i\right)= \sum_{i\in I}\nu(A_i)\] et cela ne dépend pas de l'ordre de sommation.
Si $A\in \mathcal{A},\mu(A)=0$, alors $\1_A=0$ dans $\L^p(\mu)$ donc $\nu(A) =\phi(\1_A) =\phi(0)=0$ donc $\nu\ll \mu$.
Comme $\mu$ est $\sigma$-finie, le théorème de Radon-Nikodyn garantit l'existence de $g\in \L^1(\mu)$ telle que $\nu=g\diff\mu$.
Ainsi, $\forall A\in \mathcal{A}, \phi(\1_A) =\nu(A)= \int_Ag\diff\mu$ donc pour toute fonction simple $f$, \[\phi(f)=\int_X fg\diff\mu\]
donc c'est aussi vrai pour $f$ mesurable bornée, puisque toute telle fonction est limite uniforme (donc dans $\L^p(\mu)$ car $\mu$ est finie) de fonctions simples.

Montrons que $g\in \L^q(\mu)$.
Si $p=1$, on a \[\forall A\in \mathcal{A}, \quad  |\phi(\1_A)|\leq \|\phi\|\|\1_A \| =\left| \int_A g\diff\mu \right|.\]
Par propriété de moyenne, $|g|\leq \|\phi\|$, $\mu$-presque partout 
(on prend $A= \{x, \quad  |g|\geq \|g\|_{\infty}+\epsilon\}$)
donc $g\in \L^{\infty}(\mu)$, et $\|g\|_\infty \leq \|\phi\|$.

Si $1<p<\infty$, on pose \[f_n = \1_{\{|g|\leq n\}} |g|^{q-1} \frac{\bar{g}}{|g|}\] de sorte que  $f_n$ est bornée (donc dans $\L^p(\mu)$), et 
\[ \phi(f_n) =\int_Xf_ng\diff\mu = \int_{\{|g|\leq n\}}|g|^{q-1}\frac{|g|^2}{|g|}\diff\mu =\int_{\{|g|\leq n\}}|g|^q\diff\mu\]
et $|\phi(f_n)|\leq \|\phi\|\|f_n\|_p$ et \[ \int_X |f_n|^p \diff\mu = \int_{\{|g|\leq n\}} |g|^{pq-p} \diff\mu.\]
Ainsi, \[ \int_{\{|g|\leq n\}} |g|^q \diff\mu \leq \|\phi\| \left( \int_{\{|g|\leq n\}} |g|^q \diff\mu \right)^{\frac1p},\]
donc \[ \left( \int_{\{|g|\leq n\}} |g|^q\right)^{\frac1q} \leq \|\phi\|.\]
En faisant $n\to \infty$ et en appliquant le théorème de Beppo-Levi, on obtient $\|g\|_q \leq \|\phi\|$ donc $g\in \L^q(\mu)$.
On conclut en étendant la formule \[\phi(f)=\int_X fg\diff\mu= \Phi_g(f)\] à toute fonction $f$ de $\L^p(\mu)$ en l'approchant par des fonctions 
mesurables bornées, et par continuité de $\Phi_g$ sur $\L^p(\mu)$, maintenant que l'on sait que $g\in \L^q(\mu)$. Cela donne la surjectivité de $\Phi$.

De plus, la preuve montre que $\|g\|_q \leq \|\phi\|=\|\Phi_g\|\leq \|g\|_q$ d'où la propriété d'isométrie.
\end{proof}

Lorsque $\mu$ est seulement  $\sigma$-finie, on utilise le lemme suivant.

\begin{lmm}
Il existe une fonction $w\in \L^1 (\mu)$ telle que $\forall x\in X, 0<w(x) <1$.
\end{lmm}

\begin{proof}[Preuve du lemme]
Soient $X_n\in \mathcal{A}$ tels que $\mu(X_n) <\infty$, et $\bigcup_{n\geq 1} X_n=X$.
Posons \[w_n(x)= \frac{1}{2^n}\cdot \frac{\1_{X_n}}{1+\mu(X_n)}\] et \[w=\sum_{n\geq 1} w_n.\]
Cette fonction convient.
\end{proof}

\begin{proof}[Preuve dans le cas général]
On observe que, la fonction $w$ du lemme étant fixée, si $\tilde{\mu}=w\diff\mu$ alors $\tilde{\mu}$ est finie, positive et
\[
    \begin{array}{rcl}
    \L^p(\tilde{\mu})& \to& \L^{p}(\mu) \\
    \tilde{f}&\mapsto &\tilde{f}w^{1/p}
    \end{array}\] est une isométrie surjective.
En effet,
\[ \|\tilde{f}w^{1/p}\|_{\L^p(\mu)}=\int_X\left| \tilde{f}w^{1/p}\right|^p\diff\mu =\int_X|\tilde{f}|^p w\diff\mu = \|\tilde{f}\|_{\L^p(\tilde{\mu})}\]
la surjectivité vient de ce que $w$ est strictement positive.
Si $\phi\in (\L^p(\mu))'$, on définit un élément $\tilde{\phi} \in (\L^p(\tilde{\mu}))'$
Comme $\tilde{\phi}(\tilde{f}) =\phi(\tilde{f}w^{1/p}), \tilde{f}\in  \L^p(\tilde{\mu})$,
on a $\|\tilde{\phi}\|=\|\phi\|$ et par le cas précédent il existe $\tilde{g}\in \L^q(\tilde{\mu})$ telle que
\begin{align*}
\tilde{\phi} (\tilde{f}) &= \int_X\tilde{f}\tilde{g} \diff\tilde{\mu}\\
    &= \int_X\tilde{f}\tilde{g} w\diff\mu\\
&= \int_X(\tilde{f}w^{1/p})(\tilde{g}w^{1/q}) \diff\mu
\end{align*}
Ainsi, si $ f\in \L^p(\mu)$, avec $\tilde{f}=fw^{-1/p}$, on obtient \[\phi(f)= \tilde{\phi}(fw^{-1/p}) =\int_X f(\tilde{g}w^{1/q})\diff\mu\]
On pose $g= \tilde{g}w^{1/q} \in \L^q(\mu)$. On conclut comme ci-dessus que $\|\phi\|=\|g\|_p$, par la même preuve.
\end{proof}

\begin{rem}
Si $p=+\infty$, le résultat est faux en général
\end{rem}

\begin{ex}
$l^\infty= l^\infty(\N)$. Soit $H=\{ \underline{x}\in l^\infty,\quad  \lim_{k\to\infty}x_k \text{ existe}\}$
sous espace fermé de $l^\infty$.
Posons pour $\underline{x}\in H$, $\phi(\underline{x}) = \lim \underline{x}$. C'est une forme linéaire sur $H$ 
et $|\phi(\underline{x})|\leq 1.\| \underline{x}\|_{\infty}$.
Par le théorème de Hahn-Banach, il existe un prolongement $\phi: l^\infty (\N)\to \C$ tel que $\|\phi\|\leq 1$
et $\forall \underline{x}\in H,\phi(\underline{x}) =\lim \underline{x}$ donc $\phi\in (l^\infty)'$.
On ne peut pas écrire
\[ \phi(\underline{x}) = \int_{\N}\underline{a}.\underline{x}\diff\#_{\N}= \sum_{k\geq 0}a_kx_k\]
avec $(a_k)\in l^1(\N)$ car si $\underline{x}= (0,\dots,0,1,0,\dots) \in H$ (vaut 1 au $n$ ième rang)
on a $\sum_{k\geq 0} a_k x_k = a_n$ et $\phi(\underline{x}) =0$ donc $a_n=0$, donc $\forall n, \phi=0$ absurde.
\end{ex}
