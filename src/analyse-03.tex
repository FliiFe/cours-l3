\ifsolo
    ~

    \vspace{1cm}

    \begin{center}
        \textbf{\LARGE Intégrales de Cauchy} \\[1em]
    \end{center}
    \tableofcontents
\else
    \chapter{Intégrales de Cauchy}

    \minitoc
\fi
\thispagestyle{empty}

\section{Intégration complexe (première version)}

\begin{dfn}
    Soit $X$ un espace topologique. Un chemin\index{chemin} de $X$ est une application continue $\gamma:[0,1]\longrightarrow X$ continue. On note $o(\gamma)=\gamma(0)$ son origine\index{chemin!origine} et $t(\gamma)=\gamma(1)$ son terme\index{chemin!terme}. Si $o(\gamma)=t(\gamma)$ alors on dit que $\gamma$ est un lacet\index{lacet}\index{chemin!lacet}.
\end{dfn}

\begin{dfn}
    Une subdivision d'un intervalle $[a, b]$ avec $a<b$ est une famille finie $\sigma=(\lambda_0, \cdots , \lambda_n)$ telle que \[a=\lambda_0<\lambda_1<\cdots <\lambda_n=b\]
    On dit que $\gamma:[a, b]\longrightarrow \C$ est $\mathcal C^1$ par morceaux s'il existe une subdivision $(\lambda_0, \cdots , \lambda_n)$ de $[a, b]$ telle que $\gamma_{|]\lambda_i, \lambda_{i+1}[}$ est $\mathcal  C^1$ prolongeable en $\lambda_i, \lambda_{i+1}$
\end{dfn}
