\ifsolo
    ~

    \vspace{1cm}

    \begin{center}
        \textbf{\LARGE Formes bilinéaires} \\[1em]
    \end{center}
    \tableofcontents
\else
    \chapter{Formes bilinéaires}

    \minitoc
\fi
\thispagestyle{empty}

Soit $k$ un corps, on considère des espaces vectoriels de dimension finie sur $k$.

\section{Matrice d'une forme bilinéaire}

 \begin{dfn}
     Soient $E$ et  $F$ des espaces vectoriels et  $\phi \in  \Bil(E, F)$. Soient $\mathcal  B_E$ et $ \mathcal  B_F$ des bases de $E$ et de  $F$. La matrice de  $\phi$ dans les bases  $\mathcal B_E, \mathcal  B_F$ est \[
         \mathcal  M_{\mathcal  B_E, \mathcal  B_F}(\phi)= (\phi(e_i, f_j))_{i,j}
     \] 

     On obtient aussi un isomorphisme\footnotemark entre $\Bil(E, F)$ et  $\mathcal  M_{m,n}(k)$ (avec $m=\dim E, n=\dim F$)
\end{dfn}

\footnotetext{non canonique}

\begin{rem}[Changement de base]
Soient $\mathcal  B_E', \mathcal  B_F'$ d'autres bases. Alors \[
    \mathcal  M_{\mathcal  B_E', \mathcal  B_F'}(\phi)=\transpose P_E \mathcal M_{\mathcal  B_E, \mathcal  B_F}(\phi)P_F
\] 
où $P_E, P_F$ sont les matrices de passage.
\end{rem}

\begin{dfn}
Le rang de $\phi$ est le rang de sa matrice. Il ne dépend pas de la base choisie.
\end{dfn}

\begin{rem}
Si $E=F$, alors  \[
    \mathcal  M_{\mathcal  B_E', \mathcal  B_E'}(\phi)=\transpose P_E \mathcal M_{\mathcal  B_E, \mathcal B_E}(\phi)P_E
\] 
Ces matrices sont dites congruentes\index{congruence (matrices)}\index{matrices congruentes}
\end{rem}


\section{Formes bilinéaires et dualité}

\begin{dfn}
    Soit $\phi \in  \Bil(E, F)$. On définit les applications linéaires \[
    \begin{array}{rrcl}
        L_\phi:& E & \longrightarrow &F^\star  \\
               & x & \longmapsto & \displaystyle (y \longmapsto \phi(x, y))\\
        R_\phi:& F & \longrightarrow &E^\star  \\
               & y & \longmapsto & \displaystyle (x \longmapsto \phi(x, y))
    \end{array}
    \] 
\end{dfn}

\begin{rem}
L'isomorphisme de bidualité permet de montrer que ces deux application sont transposées l'une de l'autre
\end{rem}

\begin{ex}
Si $\phi$ est le crochet de dualité \[
\begin{array}{rrcl}
    \langle \;\rangle:& E\times E^\star & \longrightarrow & k \\
                      & (x, \lambda) & \longmapsto & \displaystyle \lambda(x)
\end{array}
\] 
alors $L_\phi$ est l'isomorphisme canonique de bidualité et  $R_\phi$ est l'identité de  $E^\star$
\end{ex}

\begin{prop}
    Les applications $L:\phi \longmapsto L_\phi$ et $R:\phi \longmapsto R_\phi$ fournissent des isomorphismes canoniques \[
        \Bil(E, F)\cong \Hom(E, F^\star) \qquad \text{ et }\qquad \Bil(E, F)\cong \Hom(F, E^\star)
    \] 
\end{prop}

\begin{proof}
    Il y a égalité des dimensions, puis $L_\phi=0 \implies \forall  x \in E, L_\phi(x)=0 \implies \phi\equiv 0$
\end{proof}

\begin{prop}
    Soient $\mathcal  B_E, \mathcal  B_F$ des bases de $E, F$, et soit  $ \phi \in \Bil(E, F)$. Alors \[
        \mathcal  M_{\mathcal  B_E, \mathcal  B_F}(\phi)=\mathcal  M_{\mathcal  B_E, \mathcal  B_F^\star}(L_\phi)=\mathcal M_{\mathcal  B_F, \mathcal  B_E^\star}(R_\phi)
    \] 
\end{prop}

\begin{proof}
    Si $\mathcal  B_E=(e_1, \cdots , e_m)$ et $\mathcal  B_F=(f_1, \cdots , f_n)$ alors \[
        R_\phi(f_j)=\sum_{i=1}^m \phi(e_i, f_j)e_i^\star
    \] 
\end{proof}

\begin{cor}
\[
\rg \phi=\rg L_\phi=\rg R_\phi
\] 
\end{cor}

\section{Formes bilinéaires non dégénérées}

\begin{prop}
    Une forme bilinéaire est dite non-dégénérée\index{forme bilinéaire non-dégénérée} si et seulement si: \begin{enumerate}
        \item $\exists  \mathcal  B_E, \mathcal  B_F$ tels que $\mathcal  M_{\mathcal  B_E, \mathcal  B_F}(\phi)$ est inversible
        \item $\forall  \mathcal  B_E, \mathcal  B_F$, $\mathcal  M_{\mathcal  B_E, \mathcal  B_F}(\phi)$ est inversible
        \item $L_\phi$ est un isomorphisme
        \item  $R_\phi$ est un isomorphisme
    \end{enumerate}
\end{prop}

\begin{ex}
Le crochet de dualité est une forme bilinéaire non dégénérée car $R_{\langle\;\rangle}=\id_{E^\star}$ est un isomorphisme.
\end{ex}

\begin{rem}
Si $\dim E=\dim F$ alors c'est équivalent à l'injectivité de  $L_\phi$ ou  $R_\phi$
\end{rem}

\section{Orthogonalité}

\begin{dfn}
    Soit $\phi \in  \Bil(E, F)$ et $V$ un sev de $E$. On pose \[
        V^\top = V^{\top, \phi}= \left\{ y \in  F \suchthat \forall  x  \in  V, \phi(x, y)=0 \right\} 
    \] 
    De même, si $W$ est un sev de $F$, on pose \[
        W^\top = W^{\top, \phi}= \left\{ x \in  E \suchthat \forall  y  \in  W, \phi(x, y)=0 \right\} 
    \] 
    $V^\top$ et $W^\top$ sont des sev respectifs de $F, E$.
\end{dfn}

\begin{rem}
Si $\phi$ est le crochet de dualité, on retrouve la notion d'orthogonalité définie dans le chapitre correspondant.
\end{rem}


\begin{prop}
Si $V$ est un sev de  $E$, alors  \[
\dim V+\dim V^{\bot, \phi}\geq \dim F
\] 
\end{prop}

\begin{proof}
    On considère $L_\phi: V\longmapsto L_\phi(V)$. On a \[\dim V\geq \dim L_\phi(V)\geq \dim F-\dim (L_\phi(V))^\bot\]
    or $L_\phi(V)^\top=\left\{ y \in F, \quad \forall x \in  V, \phi(x, y)=0 \right\}  =V^\bot$.
\end{proof}

\begin{prop}
    Si $\phi$ est non dégénérée, alors pour tout sev  $V$ de  $E$, on a \[\dim V+\dim V^{\bot, \phi}=\dim F=\dim E\]
\end{prop}
\begin{proof}
    Puisque $L_\phi$ est un isomorphisme,  $\dim V=\dim L_\phi(V)$
\end{proof}

\begin{prop}
    Soit $\phi \in  \Bil(E, F)$. Alors \begin{itemize}
        \item Si  $V_1$ et $V_2$ sont des sev de $E$, $(V_1+V_2)^\bot=V_1^\bot\cap V_2^\bot$
    \end{itemize}
    Si  de plus $ phi$ n'est pas dégénérée alors \begin{itemize}
        \item Pou $V$ sev de $E$, $V ^{\bot\bot}=V$ (idem pour les sev de $F$)
        \item $(V_1\cap V_2)^\bot=V_1 \bot+V_2^\bot$
        \item  $V=\left\{ 0 \right\} \iff  V^\bot=F $  et $V=E \iff  V^\bot =\left\{ 0 \right\} $. Idem pour les sev de $F$.
    \end{itemize}
\end{prop}

\begin{proof}
Dans les cas où $\phi$ n'est pas dégénérée:  \begin{itemize}
    \item $V\subseteq V^{\bot\bot}$ et il y a égalité des dimensions
    \item $(V_1^\bot+V_2^\bot)^\bot=V_1^{\bot\bot}\cap V_2^{\bot\bot}=V_1\cap V_2$ et on passe à l'orthogonal
    \item Si $V = \left\{ 0 \right\} $ alors $V^\bot =F$ clair, et dans l'autre sens si  $V^\bot=F$ alors  $V^{\bot\bot}=V=F^\bot=\left\{ 0 \right\} $
\end{itemize}
\end{proof}

% TODO: exemple

\section{Restriction, passage au quotient}

\begin{prop}
    Soit $\phi\in \Bil (E, F)$. Si  $E'$ et  $F'$ sont des sev de  $E, F$ alors  $\phi'=\phi\left|_{E'\times F'}\right.$ est une forme bilinéaire. En général,  $\phi'$ n'hérite pas de la non-dégénérescence de  $\phi$
\end{prop}

% TODO: exemple

\begin{prop}
    Soit $\phi \in  \Bil(E, F)$. On pose $E'=E / \ker L_\phi$ et  $F'=F / \ker R_\phi$. On note  $\pi_E, \pi_F$ les applications canoniques associées. Il existe une forme bilinéaire $\phi' \in  \Bil(E', F')$ non dégénérée telle que $\phi'(\pi_E, \pi_F)=\phi$.
\end{prop}

\begin{proof}
    Soit $x \in  E$. L'application $L_\phi(x)$ est nulle sur  $\ker R_\phi$. Par théorème de passage au quotient,  $L_\phi(x)$ induit une forme linéaire sur  $F'$ qu'on note  $f(x):(F')^\star \longrightarrow k$. On peut ainsi définir $f:E \longmapsto (F')^\star$, et cette application est linéaire.

    De plus, $\ker f=\ker L_\phi= \left\{ x \in E, \quad  \forall  y \in  F, \phi(x, y)=0 \right\} $. Par théorème de passage au quotient pour $f$, on obtient  $\bar{f} : E' \longrightarrow (F')^\star$, et cette fonction est injective (on a quotienté par $\ker f$).

    Cette fonction $\bar{f}$ définit une forme bilinéaire $\phi' \in  \Bil(E, F)$ qui convient, et on sait déjà que $L_{\phi'}$ est injective. La même construction en partant de  $R_\phi$ donne lieu à la même forme bilinéaire, donc  $R_{\phi'}$ est aussi injective et  $\phi'$ n'est pas dégénérée.
\end{proof}

\section{Formes bilinéaires symétriques, antisymétriques et alternées}

On suppose désormais que $E=F$

 \begin{dfn}
     On dit que $\phi \in  \Bil(E)$ est \begin{itemize}
         \item Symétrique\index{forme bilinéaire!symétrique} si $\forall  x, y \in  E, \phi(x, y)=\phi(y, x)$ 
         \item Antisymétrique\index{forme bilinéaire!antisymétriques} si $\forall  x, y \in  E, \phi(x, y)=-\phi(y, x)$ 
         \item Alternée\index{forme bilinéaire!alternée} si $\forall  x \in  E, \phi(x, x)=0$
     \end{itemize}
\end{dfn}

\begin{lmm}
Si $\phi$ est alternée alors  $\phi$ est antisymétrique.
\end{lmm}

\begin{proof}
    $0=\phi(x + y, x+y)=\phi(x, y)+\phi(y, x)$
\end{proof}

\begin{rem}
    Si $2\neq 0$ alors les formes antisymétriques sont alternées car  $2\phi(x, x)=\phi(x, x)+\phi(x, x)=0$
\end{rem}

\begin{rem}
    On note $\Sym(E), \AS(E), \Alt(E)$ les sev de  $\Bil(E)$ correspondants aux formes bilinéaires symétriques, antisymétriques, alternées.
\end{rem}
