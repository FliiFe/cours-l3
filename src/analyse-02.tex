\ifsolo
    ~

    \vspace{1cm}

    \begin{center}
        \textbf{\LARGE Séries entières et fonctions analytiques} \\[1em]
    \end{center}
    \tableofcontents
\else
    \chapter{Séries entières et fonctions analytiques}

    \minitoc
\fi
\thispagestyle{empty}

On se place dans le corps $\K=\R$ ou $\C$

\section{Algèbre des séries entières}

\begin{dfn}
    On appelle \textbf{algèbre des séries formelles}\index{série formelle} à coefficients dans $\K$ (ou séries entières si $\K=\C$) l'ensemble des suites $\K^\N$ d'éléments de $\K$ où on note \[
    \sum_{n \in  \N}\alpha_n T^n
    \] 
    la suite $(\alpha_n)_{n \in  \N}$, muni des opérations \begin{itemize}
        \item \[
                \left( \sum_{n \in  \N}\alpha_nT^n \right) + \left( \sum_{n\in \N} \beta_n T^n \right)= \sum_{n \in  \N} (\alpha_n+\beta_n)T^n
        \] 
        \item \[
        \lambda \sum_{n \in  \N} \alpha_nT^n=\sum_{n \in  \N} \lambda \alpha_n T^n
        \] 
        \item \[
                \left( \sum_{n \in  \N} \alpha_nT^n \right) \left( \sum_{n \in  \N} \beta_nT^n \right)= \sum_{n \in  \N} \left( \sum_{p+q=n} \alpha_p\beta_q \right)T^n
        \] 
    \end{itemize}
    On note $\K [ [T]] $ cette algèbre
\end{dfn}

\begin{rem}
    L'algèbre $\K[T]$ est une sous-algèbre donnée par les séries formelles de terme général à support finis.
\end{rem}

\begin{dfn}[Série dérivée\index{série dérivée}]
    La dérivée d'une série formelle $S=\sum_{n \in \N} \alpha_{n} T^{n}$ est la série formelle
\[
S'=\sum_{n \in \N}(n+1) \alpha_{n+1} T^{n}
\]
\end{dfn}

\section{Rappels sur les rayons de convergence}

\begin{dfn}
Soit $S=\sum_{n \in \N} \alpha_{n} T^{n} \in \K[[T]]$ une série formelle. Son rayon de convergence est défini par
\[
R_{S}=\max \left\{t \in \R \suchthat \text { La série } \sum_{n \geqslant 0} |\alpha_{n} |t^{n} \text { converge }\right\} \in[0,+\infty]
\]
\end{dfn}

\begin{prop}
\begin{enumerate}
    \item Soit $z \in  \C$, $|z|<R_S$. Alors la série  $\sum _{n\geq 0}\alpha_nz^n$ converge absolument
    \item Soit $z \in  \C$, $|z|>R_S$. Alors la suite $(\alpha_z z^n)_n$ n'est pas bornée
\end{enumerate}
\end{prop}

\begin{prop}
\[
R_S= \left(\limsup_{n\to +\infty}(|\alpha_n|^{1/n})\right)^{-1}
\]
\end{prop}

\begin{prop}
\begin{itemize}
    \item $R_{S+T}\geq \min(R_S, R_{T})$ avec égalité si les deux rayons sont distincts
    \item $R_{ST}\geq \min(R_S, R_{T})$
    \item $R_{S'}=R_S$
\end{itemize}
\end{prop}

\section{Compléments sur les séries absolument convergentes}

\begin{rem}
Dans une série absolument convergente, on peut permuter les termes sans perturber la convergence ou la limite. C'est faux sinon.\footnotemark
\end{rem}

\footnotetext{Cf. Théorème de réarrangement de Steinitz}

\begin{prop}
    Soit $I$ un ensemble dénombrable et  $(u_i)_{i \in  I}$ une famille telle que \[
        \left\{ \sum_{j \in  J}|u_j| , \quad J \in \mathcal  P_f(I)\right\} 
    \] 
    est majoré. Alors il existe un unique $s \in  \C$ tel que \[
        \forall  \epsilon>0, \exists J \in \mathcal  P_f(I), \forall  K \in  \mathcal P_f(I), J\subset K \implies \left| s- \sum_{i \in  K} u_i \right|<\epsilon
    \] 
    On le note \[
    s=\sum_{i\in I} u_i
    \] 
\end{prop}

\section{Dérivabilité}

\begin{prop}
    Soit $S=\sum \alpha_nT^n \in  \K[ [ T]]$ de rayon de convergence $R>0$. Alors \[
    \begin{array}{rrcl}
        f:&\B_\K(0, R)  & \longrightarrow & \K \\
        & z & \longmapsto & \displaystyle \sum_{n\geq 0} \alpha_nz^n
    \end{array}
    \] 
    est dérivable et sa dérivée est donnée par l'application obtenue avec la série $S'$
\end{prop}

\begin{proof}
    Soit $a \in  \B_\K(0, R)$ et $R' \in  ]|a|, R[$. Alors pour $z \in  \B_\K(0, R')\setminus  \left\{ a \right\} $, \[
        \frac{f(z)-f(a)}{z-a}= \sum_{n\geq 0} \alpha_n \frac{z^n-a^n}{z-a}= \sum_{n\geq 0} \alpha_{n+1}\sum_{k=0}^{n} a^kz^{n-k} \tag{$\star$}
    \] 
    Or \[
        \sum_{k=0}^{n} a^kz^{n-k}\leq (n+1)R'^n
    \] 
    et la série $\sum \alpha_{n+1} (n+1)R'^n$ converge absolument donc $(\star)$ converge normalement.
    En particulier, pour tout $\epsilon>0$, il existe $N \in \N$ tel que pour tout $n \geq N$, on ait
\[
\forall z \in \B_{\K}\left(0, R'\right)-\{a\}, \quad\left|\frac{f(z)-f(a)}{z-a}-\sum_{n=0}^{N} \alpha_{n+1} \sum_{k=0}^{n} a^{k} z^{n-k}\right|<\epsilon
\]
et, pour $z=a$,
\[
\left|\sum_{n \geq 0}(n+1) \alpha_{n+1} a^{n}-\sum_{n=0}^{N}(n+1) \alpha_{n+1} a^{n}\right|<\epsilon
\]
Comme la somme $\sum_{n=0}^{N} \alpha_{n+1} \sum_{k=0}^{n} a^{k} z^{n-k}$ tend vers $\sum_{n=0}^{N}(n+1) \alpha_{n+1} a^{n}$ quand $z$ tend vers $a$, on obtient que le quotient $\frac{f(z)-f(a)}{z-a}$ tend vers $\sum_{n \geq 0}(n+1) \alpha_{n+1} a^{n}$ quand $z$ tend vers $a$.
\end{proof}

\begin{cor}
$f$ est infiniment dérivable.
\end{cor}

\section{Fonctions analytiques}

\begin{dfn}
    Soit $U$ un ouvert de  $\K$, $f:U \longrightarrow \K$ et $a \in  U$. On dit que $f$ est \textbf{analytique} \index{analytique} en $a$ s'il existe une série formelle $S \in  \K [ [T]]$ avec $R_S>0$ et  $r \in  ]0, R]$ tel que \[
        \forall  z \in  \B_\K(0, r), f(z)= \sum_{n=0}^{+\infty} \alpha_n (z-a)^n
    \] 
    On dit que $f$ est analytique sur  $U$ si elle l'est en tout point de  $U$.
\end{dfn}

\begin{rem}
Une fonction analytique est donc infiniment dérivable et ses coefficients sont donnés par le développement de Taylor.
\end{rem}

\begin{thm}
    Si $S=\sum \alpha_nT^n \in \K [[T]]$ avec $R_S>0$, soit $a \in  \K$, alors \[
        f: \B_\K(a, R_S) \longrightarrow \K, z \longmapsto \sum_{n\geq 0} \alpha_n(z-a)^n
    \] 
    est analytique sur $\B_\K(0, R)$
\end{thm}

\begin{proof}
    On se contente du cas $a=0$. On se donne  $b \in  \B_\K(0, R_S)$ et soit $r >0$ tel que $|b|+r<R_S$. Soit  $z \in  \B_\K(b, r)$. Alors
\begin{align*}
f(z) &=\sum_{n \geq 0} \alpha_{n} z^{n}=\sum_{n \geq 0} \alpha_{n}(z-b+b)^{n} \\
&=\sum_{n \geq 0} \sum_{p+q=n} \alpha_{n} \frac{n !}{p ! q !} b^{p}(z-b)^{q} \\
&=\sum_{n \geq 0} \sum_{p+q=n} \alpha_{p+q} \frac{(p+q) !}{p ! q !} b^{p}(z-b)^{q} \tag{$\star$}
\end{align*}
Mais la série formelle $\sum_{p \in \N}(p+q) \cdots(p+1) \alpha_{p+q} z^{p}$ est la série formelle $S^{(q)}$ obtenue en dérivant $q$ fois la série $S .$ Donc son rayon de convergence est également $R$. Cela nous permet de poser
\[
\beta_{q}=\frac{1}{q !} \sum_{p \geq 0} \frac{(p+q) !}{p !} \alpha_{p+q} b^{p}
\]
Les majorations
\[
\sum_{p+q=n}\left|\alpha_{n}\right| \frac{n !}{p ! q !}\left|b^{p}(z-b)^{q}\right|=\left|\alpha_{n}\right|(|b|+|z-b|)^{n} \leq\left|\alpha_{n}\right|(|b|+r)^{n}
\]
et $|b|+r<R$ impliquent que la somme ($\star$) converge absolument et on obtient l'égalité
\[
f(z)=\sum_{q \geq 0}\left(\sum_{p \geq 0} \frac{(p+q) !}{p ! q !} \alpha_{p+q} b^{p}\right)(z-b)^{p}=\sum_{q \geq 0} \beta_{q} z^{q}
\]
\end{proof}

\section{Principe des zéros isolés}

\begin{thm}[Principe des zéros isolés\index{zéros isolés}]
Soit $U$ un ouvert de  $\K$ et $f:U \longrightarrow \K$ une fonction analytique. Soit $a \in U$. Il y a deux cas \begin{itemize}
    \item Toutes les dérivées sont nulles en $a$, et dans ce cas il existe  $r>0$ tel que  $f(\B(a, r))= \left\{ 0 \right\} $ 
    \item Sinon, il existe $r>0$ tel que  $0 \not \in f(\B(a, r)\setminus \left\{ a \right\} )$
\end{itemize}
\end{thm}

\begin{proof}
   Par le paragraphe précédent, il existe $r_{0} \in \R_{+}^{*}$ tel que la série $\sum_{n \in \N} \frac{f^{(n)}(a)}{n !} T^{n}$ a un rayon de convergence $R>r_{0}$ et
\[
\forall z \in B\left(a, r_{0}\right), \quad f(z)=\sum_{n \geqslant 0} \frac{f^{(n)}(a)}{n !}(z-a)^{n}
\]
    Si $f^{(n)}(a)=0$ pour tout $n \in \N$, il en résulte que l'application $f$ est nulle sur $\B(a, r_{0})$. Dans le cas contraire, posons $\alpha_{n}=\frac{f^{(n)}(a)}{n !}$ pour $n \in \N$ et
\[
n_{0}=\min \left\{n \in \N \mid f^{(n)}(a) \neq 0\right\}
\]
On peut alors écrire
\[
\forall z \in \B\left(a, r_{0}\right), \quad f(z)=(z-a)^{n_{0}}\left(\alpha_{n_{0}}+g(z-a)\right)
\]
où $g(z)=\sum_{n \geqslant 1} \alpha_{n+n_{0}} z^{n}$ pour $z \in \B\left(0, r_{0}\right)$. L'application $g$ est continue et $g(0)=0 .$ Comme $\alpha_{n_{0}} \neq 0$, on peut choisir un nombre réel $\left.r \in\right] 0, r_{0}[$ tel que
\[
\forall z \in \B(0, r), \quad|g(z)|<\left|\alpha_{n_{0}}\right|
\]
Donc, pour $z \in B(a, r)$
\[
\begin{aligned}
|f(z)| &=|z-a|^{n_{0}}\left|\alpha_{n_{0}}\right|\left|1+\frac{1}{\alpha_{n_{0}}} g(z-a)\right| \\
& \geqslant|z-a|^{n_{0}}\left|\alpha_{n_{0}}\right|\left|1-\frac{1}{\left|\alpha_{n_{0}}\right|}\right| g(z-a)||
\end{aligned}
\]
Mais comme $\frac{1}{\left|\alpha_{n_{0}}\right|} g(z-a)<1$ pour $z \in \B(a, r)$, on obtient que $|f(z)|>0$ pour $z \in \B(a, r)\setminus \{a\}$
\end{proof}

\begin{cor}
Soit $U$ un ouvert connexe de $\K$ et soit $f$ une fonction non identiquement nulle analytique sur $U$, alors l'ensemble de ses zéros
\[
Z(f)=\{z \in \K \mid f(z)=0\}
\]
est une partie localement finie de $U$.
\end{cor}

\begin{cor}[Principe du prolongement analytique\index{prolongement analytique}]
    Soit $U$ un ouvert connexe de $\K$. Soient $f$ et $g$ des applications analytiques sur $U$. Si l'une des deux conditions suivantes est vérifiée \begin{enumerate}[label=(\roman*)]
\item $\exists a \in U, \quad \mathrm{DT}_{a}(f)=\mathrm{DT}_{a}(g)$
\item L'ensemble $\{z \in U \mid f(z)=g(z)\}$ n'est pas localement fini dans $U$
\end{enumerate}
alors $f=g$
\end{cor}
\begin{proof}
Tout d'abord l'assertion (i) implique (ii). En effet, si $\mathrm{DT}_{a}(f-g)=0$, alors il existe $r \in \R_{+}^{\star}$ tel que $f_{\mid \B(a, r)}=g_{\mid \B(a, r)}$.

Si l'assertion (ii) est vérifiée, alors $f-g$ est l'application constante nulle.
\end{proof}

\begin{cor}
    Soit $U$ un ouvert connexe de C. Soit $Z \subset U$ une partie qui n'est pas localement finie dans U. Soit $f: Z \longrightarrow \K$ une application. Alors il existe au plus une application analytique $\tilde{f}$ sur $U$ qui prolonge $f$ (i.e. $\tilde{f}_{\mid Z}=f$)
\end{cor}

\begin{cor}
Soit $U$ un ouvert connexe de $\K$, et $f,g$ des applications analytiques sur $U$. Si $f$ et $g$ sont non nulles alors $fg$ est non nulle.
\end{cor}

\section{Détermination principale du logarithme}

\begin{rem}[Rappel]
    La série $\sum \sfrac{T^n}{n!}$ a un rayon de convergence infini. L'application associée définit l'exponentielle complexe\index{exponentielle complexe}. \[
    \begin{array}{rrcl}
        \exp:& \C & \longrightarrow & \C \\
             & z & \longmapsto & \displaystyle \sum_{n\geq 0} \frac{z^n}{n!}
    \end{array}
    \] 
    Cette application a les propriétés suivantes pour tous $a, b, z, z' \in  \C$: \begin{itemize}
        \item $\exp(a+b)=\exp(a)\exp(b)$
        \item $\exp (\bar{z})=\bar{\exp (z)}$ 
        \item $|\exp(z)|=\exp(\Re(z))$
        \item $\exp(z)=\exp(z')\iff  \exists k \in  \Z, \quad , z-z'=2ik\pi$
    \end{itemize}
\end{rem}

\begin{prop}
L'exponentielle est holomorphe sur $ \C$ et $\exp'=\exp$.
\end{prop}

\begin{defprop}[Détermination principale du logarithme]
Il existe une unique application continue \[
    \Log: \C \setminus ]-\infty, 0] \longrightarrow \C
\] 
telle que \begin{enumerate}
    \item $\Log(1)=0$
    \item $\exp\circ\Log=\id_{\C\setminus ]-\infty, 0]}$
\end{enumerate}
On l'appelle la détermination principale du logarithme\index{logarithme!détermination principale}. Cette application est holomorphe de dérivée $z\longmapsto \sfrac1z$ et \[
    \forall  z \in  \B(0, 1), \quad  \Log(1-z)=-\sum_{n\geq 1}\frac{z^n}{n}
\] 
\end{defprop}

\begin{proof}
%TODO: cette preuve
\end{proof}
