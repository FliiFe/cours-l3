\ifsolo
    ~

    \vspace{1cm}

    \begin{center}
        \textbf{\LARGE Espaces métriques} \\[1em]
    \end{center}
    \tableofcontents
\else
    \chapter{Espaces métriques}

    \minitoc
\fi
\thispagestyle{empty}

\section{Distances}

\begin{dfn}
    Un espace métrique\index{espace métrique} $(E, d)$ est un espace vectoriel $E$ muni d'une distance\index{distance} $d:E\times E \longrightarrow  \R$ qui satisfait, pour tous $x,y,z \in  E$
    \begin{itemize}
        \item $d(x, y)\geq 0$ et $d(x, y)=0 \iff  x=y$
        \item $d(x, y)=d(y, x)$
        \item  $d(x, z)\leq d(x, y)+d(y, z)$
    \end{itemize}
\end{dfn}

\begin{ex}
\begin{itemize}
    \item $E=\R$, $d(x, y)=|x-y|$
    \item  $E$ un e.v.n,  $d(x, y)=\|x-y\|$ 
    \item $E$ un ensemble,  \[
            d(x, y)=
            \begin{dcases}
                0 & \text{ si } x=y \\
                1 & \text{ sinon }
            \end{dcases}
    \] 
    Cette distance est appelée distance discrète\index{distance!discrète}
    \item $(E_1, d_1), (E_2, d_2)$ des espaces métriques. On peut définir sur $E_1\times E_2$ \[
            d((x_1, x_2), (y_1, y_2))= \max (d_1(x_1, y_1), d_2(x_2, y_2))
    \]
\item $E=\R^2 $, Distance euclidienne: \[ d(x, y)=\sqrt{(x_1-x_2)^2 + (y_1-y_2)^2 } \]
    Distance "Manhattan": \[
        d(x, y)=|x_1-x_2|+|y_1-y_2|
    \] 
\item $(E, d_E)$,  $A\subset E$. La distance induite sur  $A$ par  $E$ est  $d_{E}\!\left|_{A}\right.$
    \item  $\Q\subset \R$. On se donne $p$ un nombre premier. La valeur absolue  $p$-adique vaut  \[
            \left| \frac{a}{b} \right|_p=p^{v_p(b)-v_p(a)}
    \] 
    et $|0|_p=0$. La distance  $d(a, b)=|a-b|_p$ est la distance $p$-adique\index{distance!p-adique}. Elle est dite ultramétrique\index{distance!ultramétrique} car elle satisfait  \[
        \forall  x, y, z \in  \Q, \quad d(x, y) \leq  \max (d(x, z), d(z, y))
    \] 
\end{itemize}
\end{ex}

\begin{exo}
Quels sont les $p>0$ tels que  \[
    d(x, y)=\sqrt[p]{(x_1-x_2)^p + (y_1-y_2)^p } 
\] 
définit une distance ?
\end{exo}

\begin{dfn}
    Soit $(E, d)$ un espace métrique. Si  $\{x_i\}^{i\geq 1}$ est une suite de $E$ et  $x \in  E$, alors on dit que $\{x_i\}^{i\geq 1}$ converge vers $x$ si pour tout  $\epsilon>0$, il existe
    $N\geq 1$ tel que $d(x_i, x)\leq \epsilon$ si $i\geq N$. Dans ce cas, on dit que $x$ est la limite de  $\{x_i\}^{i\geq 1}$
\end{dfn}

\begin{dfn}
    Une valeur d'adhérence\index{valeur d'adhérence} (ou point d'accumulation\index{point d'accumulation}) de $\{x_i\}_{i\geq 1}$ est un $x \in  E$ tel qu'il existe une extraction $\varphi$ telle que  $\{x_{\varphi(i)}\}_{i\geq  1}$ converge vers $x$
\end{dfn}

\begin{dfn}
    La boule\index{boule} ouverte de centre $a$ et de rayon  $r>0$ est  $\B(a, r)=\{x \in  E, d(x, a)<r\}$. La boule fermée de centre $a$ et de rayon  $r>0$ est  $\BF(a, r)=\{x \in  E, d(x, a)\leq r\}$. La sphère\index{sphère} de centre $a$ et de rayon  $r>0$ est $\BF(a, r)\setminus \B(a, r)$
\end{dfn}

% TODO: Exemples dans R^2 avec la distance euclidienne et la distance manhattan. Exemple aussi avec la distance issue de la norme infinie

\begin{exo}
$\Z$ muni de la distance $p$-adique. Montrer \[
    \BF(a, \sfrac{1}{p})= \left\{ b \in  \Z, \; b\equiv a \mod p \right\} 
\] 
\end{exo}

\begin{dfn}
    Un voisinage\index{voisinage} de $x \in  E$ est une partie de $E$ qui contient  $\B(x, r)$ pour un $r>0$
\end{dfn}

\begin{dfn}
    On dit qu'une partie $U$ de $E$ est ouverte\index{ouvert} si pour tout $x \in  U$, il existe $r>0$ tel q e $ \B(x, r) \subset U$. Une partie est fermée\index{fermé} si son complémentaire est ouvert.
\end{dfn}

\begin{prop}
    Une union quelconque d'ouverts est ouverte. Une intersection \emph{finie} d'ouverts est ouverte.
\end{prop}

\begin{rem}
Une intersection infinie d'ouverts n'est \emph{a priori} pas ouverte.
\end{rem}

\begin{cor}
Une intersection quelconque de fermés est fermée, et une union finie de fermés est fermée.
\end{cor}

\begin{exo}
    Si $d$ est ultramétrique,  $\B(a, r)=\B(x, r)$ pour tout $x \in  \B(a, r)$. Les boules ouvertes d'un espace ultramétrique sont fermées
\end{exo}

\begin{thm}[Caractérisation séquentielle des fermés\index{caractérisation des fermés}]
    Une partie $F\subset E$ est fermée  si et seulement si pour tout suite $\{x_i\}_{i\geq 1}$ de $F$ qui converge vers $x \in  E$, on a $x \in  F$
\end{thm}

\begin{proof}
    $\implies )$ On suppose $F$ fermé. On se donne une suite  $\{x_i\}$ de  $F$ qui converge vers  $x \in E$. Supposons que $x \in F^c$. Alors, il existe $r>0$ tel que  \[
        \B(x, r)\cap F=\emptyset
    \]
    En particulier, $\forall  i \geq 1, d(x, x_i)\geq r$ ce qui est absurde. Donc $x \in  F$.

    $\impliedby )$ Montrons que  $F^c$ est ouvert. On se donne  $x \in  F^c$. S'il n'existe pas $r>0$ tel que  $\B(x, r)\subset F^c$ alors \[
        \forall  n\geq 1, \exists x_n \in  \B\left(x, \frac{1}{n}\right)\cap F \qquad  \text{donc} \qquad  x_n\xrightarrow[n\to +\infty]{}x \in F
    \]
    absurde.\footnote{Démonstration constructive en exercice}
\end{proof}

\begin{dfn}
    Si $P \subset E$, alors l'adhérence\index{adhérence}  $\bar P$ de  $P$ est l'intersection des fermés de  $E $ qui contiennent $P$.
    C'est aussi\footnotemark{} l'ensemble des limites de suites de $P$ qui convergent dans  $E$.

    L'intérieur\index{intérieur} $\mathring P$ d'une partie $P$ est l'union des ouverts contenus dans $P$. La frontière\index{frontière} de $P$ est  $\partial P=\bar P\setminus \mathring P$.
\end{dfn}
\footnotetext{Les limites sont dans tous les sur-ensembles fermés. Toutes les boules ouvertes autour d'un point de l'adhérence rencontrent $P$}


\begin{dfn}
    On dit que $P$ est dense\index{densité} dans  $E$ si $\bar P=E$. On dit que $E$ est séparable\index{séparabilité}\index{espace séparable} s'il contient une partie dénombrable et dense.
\end{dfn}

\begin{ex}
    L'espace $E=\mathcal  C^0([0,1], \R)$ muni de la norme infinie est séparable (Approximation polynomiale de Weierstrass). $\R$ muni de la distance discrète n'est pas séparable.
\end{ex}

\begin{rem}
    Toute partie d'un espace séparable est séparable.\footnote{pas si facile}
\end{rem}

\begin{dfn}
    Un point $a \in  E$ est isolé\index{point isolé} s'il existe $r>0$ tel que  $\B(a, r)=\{a\}$. L'espace $E$ est discret\index{espace discret} si tous ses points sont isolés.
\end{dfn}

\begin{ex}
    \begin{itemize}
        \item 
 Un ensemble  $E$, muni de la distance discrète est un espace discret.
 \item Tous les points sauf $0$ sont discrets dans  \[
         \bar{ \left\{ \frac{1}{n}, \quad n>0 \right\}  }
 \] 
 \item $\Z$ avec sa distance usuelle est discret
 \item $\Z$ muni de la distance $p$-adique n'est pas discret, aucun point n'est isolé:  \[
         \forall  a \in  \Z, \qquad  a+p^n \xrightarrow[n\to +\infty]{}a \quad  \text{ et } \quad a+p^n \neq a
 \]
    \end{itemize}
\end{ex}

\begin{dfn}
    Deux distances sont (fortement) équivalentes\index{distances équivalentes} s'il existe $C, C'$ telles que  $Cd_2 \leq d_1 \leq  C'd_2$
\end{dfn}

\begin{exo}
Les ouverts sont identiques pour deux distances équivalentes.
\end{exo}

\section{Fonctions continues}

\begin{dfn}
On se donne deux espaces métriques $X$ et  $Y$. On dit que  $f:X \longrightarrow Y$ est continue en  $x \in X$ si \[
    \forall  \epsilon>0, \exists \delta >0, \forall  x' \in  X, d_X(x, x')\leq \delta \implies d_Y(f(x), f(x'))\leq  \epsilon
\] 
On dit que  $f$ est continue\index{continuité} si elle est continue pour tout  $x \in  X$.
\end{dfn}

\begin{exo}
    Toutes les fonctions $f:(\R, \text{discrète}) \longrightarrow (\R, \text{usuelle})$ sont continues.
\end{exo}

\begin{thm}
Soit $f:X \longrightarrow Y$ une fonction. Les propriétés suivantes sont équivalentes:  \begin{enumerate}
    \item $f$ est continue
    \item  Pour toute suite $\{x_n\}_{n\geq 1}$ qui converge vers $x$, la suite  $\{f(x_n)\}_{n\geq 1}$ converge vers $f(x)$
    \item Pour tout ouvert  $U$ de  $Y$,  $f^{-1}(U)$ est ouvert dans $X$.
    \item Pour tout fermé  $F$ de  $Y$,  $f^{-1}(F)$ est fermé dans $X$.
\end{enumerate}
\end{thm}

\begin{proof}
    $(3 \iff  4)$ par passage au complémentaire.

    $(1 \implies 2)$  $ \left\{ x_n \right\} $ qui converge vers $x$,  $\epsilon>0$,  $\delta>0$ associé en  $x$,  \[
        d_X(x_n, x)<\delta \implies  d_Y(f(x_n), f(x))<\epsilon
    \] 
    On en déduit facilement $f(x_n)\longrightarrow f(x)$

    $(2 \implies 3)$ Soit $U$ un ouvert de  $Y$ et  $x \in  f^{-1}(U)$. Montrons qu'il existe $r>0$ tel que  $\B(x, r)\subset f^{-1}(U)$. Si ce n'est pas le cas, alors pour tout $n\geq 1$, il existe $x_n \in  \B(x, \sfrac 1n)\cap (X\setminus f^{-1} (U))$. On a $x_n \to  x$ et donc $f(x_n) \to  f(x)$. Les $f(x_n)$ sont dans  $Y\setminus U$ fermé donc $f(x) \in  U^c$ absurde.

    $(3 \implies 1)$ Soit $x \in  X$, $\epsilon>0$. $f^{-1}(\B(f(x), \epsilon))$ est un ouvert de $X$ qui contient  $x$. Donc il existe  $\delta>0$ tel que  \[
        \B(x, \delta) \subset f^{-1}(\B(f(x), \epsilon))
    \] 
\end{proof}

\begin{dfn}
    On dit qu'une fonction $f: X \rightarrow Y$ est \textbf{uniformément continue}\index{continuité uniforme} si pour tout $\epsilon>0$, il existe $\delta>0$ tel que pour tous $x, x' \in X$ vérifiant $d_{X}\left(x, x'\right)<\delta$, on a $d_{Y}\left(f(x), f\left(x'\right)\right)<\epsilon$.

    On dit qu'une fonction continue $f: X \rightarrow Y$ est un \textbf{homéomorphisme}\index{homéomorphisme}  si $f$ est bijective, et si $f^{-1}$ est continue. On dit que $f$ est une \textbf{isométrie} \index{isométrie} si $d_{Y}\left(f(x), f\left(x'\right)\right)=d_{X}\left(x, x'\right)$ quels que soient $x, x' \in X$
\end{dfn}

% TODO: trou ici (jusqu'à complétude inclus)

\section{Complétude}

\begin{dfn}[Complétude]
    Si $(E, d)$ est un espace métrique, on dit qu'une suite $\left\{x_{n}\right\}_{n \geq 1}$ est de Cauchy si pour tout $\epsilon>0$, il existe $N \geq 1$ tel que si $m, n \geq N$, alors $d\left(x_{m}, x_{n}\right)<\epsilon$. Par exemple, si $\left\{x_{n}\right\}_{n \geq 1}$ converge, alors elle est de Cauchy. Une suite de Cauchy qui admet une valeur d'adhérence est convergente.

    On dit que $E$ est complet\index{espace complet} si toute suite de Cauchy\index{suite de Cauchy} admet une limite. 
    Un espace vectoriel normé complet est appelé espace de Banach\index{espace de Banach}\index{Banach!espace}
\end{dfn}

\begin{ex}
L'espace $\R^{n}$ est un espace métrique complet. Si $E$ est complet et si $F \subset E$, alors $F$ est fermé dans $E$ si et seulement s'il est complet.
\end{ex}

\begin{thm}
    Si $(X, d)$ est un espace métrique et si  $E$ est l'espace des fonctions continues et bornées sur  $X$ à valeurs dans $\R$, muni de la norme $\|f\|_X= \sup_{x \in  X} |f(x)|$, alors $E$ est complet.
\end{thm}

\begin{proof}
    Soit $\left\{ f_n \right\} _{n\geq 1}$ une suite de Cauchy de $E$. Montrons qu'il existe  $f \in  E$ telle que $f_n \longrightarrow f$. Soit $x \in  X$. La suite $\left\{ f_n(x) \right\} _{n\geq 1}$ est une suite de Cauchy de $\R$ qui admet donc une limite. On définit $f:X \longrightarrow  \R$ par $f(x)=\lim_nf_n(x)$.

    Montrons que  $f$ est bornée et continue et que  $f_n \longrightarrow f$. Soit $\epsilon>0$ et $N\geq 1$ tel que $\|f_n-f_m\|\leq \epsilon$ si $m,n\geq N$. On a \[
        f(x)-f_n(x)=\underbrace{f(x)-f_m(x)}_{|\,\cdot\,|<\epsilon\text{ APCR}}+\underbrace{f_m(x)-f_n(x)}_{|\,\cdot\,|<\epsilon \text{ APCR}}
    \] 
    On a donc $|f(x)-f_n(x)|<2\epsilon$ si $n\geq N'$ donc $f_n \longrightarrow f$ et $f$ est bornée. Puis la continuité de $f$ est héritée par convergence uniforme.
\end{proof}

\begin{rem}~
\begin{enumerate}
    \item On n'a pas supposé $X$ complet 
    \item Le même résultat est vrai avec des fonctions à valeurs dans un espace de Banach.
    \item On a montré que l'espace des fonctions bornées (sans hypothèse de continuité) est complet, et que l'espace des fonctions continues bornées est fermé donc complet.
\end{enumerate}
\end{rem}

\begin{thm}[Théorème du point fixe de Banach\index{point fixe de Banach (théorème)}\index{Banach!point fixe (théorème)}]
    Soit $(E, d)$ un espace métrique complet, et  $f:E\longrightarrow E$ une fonction telle qu'il existe $\lambda \in  [0, 1[$ vérifiant $d(f(x), f(y))\leq \lambda f(x, y)$. On dit que $f$ est  $ \lambda$-contractante\index{fonction contractante}. La fonction $f$ admet alors un et un seul point fixe dans  $E$.
\end{thm}

\begin{proof}
    On se donne $x_0\in E$ et $x_n=f(x_{n-1})$ pour $n>0$. On a  \[ d(x_{n+1}, x_n)\leq \lambda d(x_n, x_{n-1})\leq (\cdots ) \lambda^k d(x_{n-k+1}, x_{n-k}) \leq (\cdots ) \leq \lambda^n d(x_1, x_0)\]

    Si $m\geq 1$, alors $f(x_{n+m}, x_n)\leq d(x_{n+m}, x_{n+m-1})+\cdots +d(x_{n+1}, x_n)\leq \frac{\lambda^n}{1-\lambda}d(x_1, x_0)$. Donc, la suite $\left\{ x_n \right\} $ est de Cauchy. Soit $x$ sa limite. Alors  $f(x)=x$ en passant à la limite dans $f(x_n)=x_{n+1}$. On a donc trouvé un point fixe de $f$. 

    Puis, si $x, y$ sont des points fixes alors  \[
        d(x, y)=d(f(x), f(y))\leq \lambda d(x, y)
    \] 
    donc $d(x, y)=0$ et  $x=y$.
\end{proof}

Si $(E, d)$ est un espace métrique, alors on peut le compléter en y ajoutant les limites des suites de Cauchy de  $E$. Cela se formalise par le théorème suivant.

\begin{thm}[Espace complété\index{espace complété}]
    Si $(X, d)$ est un espace métrique alors il existe un espace métrique complet  $(\hat{X}, d)$ et une isométrie $i:X \longrightarrow  \hat{X}$, tels que $i(X)$ est dense dans $\hat{X}$.

    Cet espace est unique à isométrie près et satisfait la propriété suivante: si $Y$ est complet et si  $f:X \longrightarrow Y$ est uniformément continue, alors $f$ se prolonge de manière unique en  $f:\hat{X}\longrightarrow Y$ uniformément continue.
\end{thm}

\begin{proof}[Démonstration partielle]
Il y a au moins deux manières de construire le complété.
\begin{itemize}
    \item 
\textbf{Première méthode (naturelle)}. On pose $C= \left\{ s = \left\{ s_n \right\}_n \suchthat  \left\{ s_n \right\} \text{ est de Cauchy }   \right\} $, et on définit la relation d'équivalence \[
    s\sim t \iff  d(s_n, t_n) \xrightarrow[n\to+\infty]{}0
\]
On pose alors $\hat{C}=C / \sim$. Si $\bar{s}$ et $ \bar{t}$ sont dans $\hat{X}$ alors on pose $d(\bar{s}, \bar{t})=\lim_{n\to +\infty}d(s_n, t_n)$.

Exercice: vérifier que $d$ est bien définie, que c'est une distance de  $\hat{X}$, que $\hat{X}$ est complet pour cette distance et que $i:X \longrightarrow \hat{X}, x \longmapsto \bar{\left\{ x \right\}_{n\geq 1} }$ est une isométrie.

\item \textbf{Seconde méthode (moins naturelle)}. On a construit précédemment une isométrie $i:X \longrightarrow E$ où $E$ est l'espace des fonctions continues bornées sur  $X$ munie de la norme  $\|\;\|_X$. On a vu que cet espace est complet, il suffit de prendre $\hat{X}=\bar{i(X)}$
\end{itemize}

Montrons à présent que si $f: X \longrightarrow Y$ est uniformément continue alors elle se prolonge à $\hat{X}$.
Soit $x \in  \hat{X}$ la limite d'une suite $\left\{ x_n \right\} _{n\geq 1}$ de Cauchy de $X$ et  $\epsilon>0$. Il existe $\delta>0$ d'uniforme continuité de  $f$ et si $d(x_n, x_m)<\delta$ alors  $d(f(x_n), f(x_m))<\epsilon$. La suite $\left\{ f(x_n) \right\} _{n\geq 1}$ est donc de Cauchy. Comme $Y$ est complet, cette suite admet une limite dans  $Y$. On pose  $f(x)$ cette limite. La fonction est bien définie car si $x_n \longrightarrow  x$ et $x_n' \longrightarrow x$ alors $d(x_n, x_n')\longrightarrow 0$ et donc $d(f(x_n), f(x_n'))\longrightarrow 0$ par uniforme continuité. On a donc $\lim f(x_n)=\lim f(x_n')$. Exercice: montrer que  $f:\hat{X}\longrightarrow Y$ est uniformément continue

Finalement, montrons l'unicité.
Enfin, si $\hat{X}_{1}$ et $\hat{X}_{2}$ sont deux espaces munis d'isométries $i_{1,2}: X \longrightarrow  \hat{X}_{1,2}$ d'images denses, alors $i_{1}$ se prolonge en une isométrie $\hat{X}_{2} \longrightarrow \hat{X}_{1}$ dont l'image est dense et complète (car elle est isométrique à $\hat{X}_{2}$ ) ce qui fait que $i_{1}: \hat{X}_{2} \longrightarrow \hat{X}_{1}$ est surjective et donc bijective.
\end{proof}

\begin{rem}
La première construction de $\hat{X}$ permet de montrer que si $X$ est un evn, alors $\hat{X}$ est un espace de Banach, et que l'application $i: X \rightarrow \hat{X}$ est linéaire.
\end{rem}

\begin{ex}~
\begin{enumerate}
    \item $ \hat{\Q}\simeq\R$
    \item $E=\R[X]$ et $\|P\|_{[0,1]}=\sup_{[0,1]} |P|$, $ \hat{E}=\mathcal C^0([0, 1], \R)$
    \item $E=\mathcal  C^0([0, 1], \R)$ muni de sa norme $\|f\|_2=\sqrt{\int |f|^2 }$ se complète en $ \hat{E}=\mathcal  L^2([0, 1])$
    \item $(\Z, d_p)$ n'est pas complet: $x_n=1+p+\cdots +p^n$ est de Cauchy pour $d_p$ et ne converge pas dans  $\Z$ si $p\neq 0$ car  $(p-1)x_n \longrightarrow -1$ et $-\frac{1}{p-1} \not\in \Z$. Le complété de cet espace est $\Z_p$ l'ensemble des entiers $p$-adiques.
\end{enumerate}
\end{ex}

\begin{exo}
Pour $p\neq 2$ premier, montrer que  $ \sqrt{1+p^2 } \in  \Z_p $
\end{exo}

\section{Espaces topologiques}

\begin{dfn}[Espace topologique\index{espace topologique}]
    Soit $X$ un ensemble. Une \textbf{topologie} \index{topologie} sur  $X$ est un ensemble de parties de  $X$, qu'on appelle les ouverts de  $X$, qui satisfait: \begin{itemize}
        \item $\emptyset$ et  $X$ sont ouverts
        \item Une intersection finie d'ouverts est ouverte
        \item Une réunion quelconque d'ouverts est ouverte.
    \end{itemize}
    Un ensemble $X$ muni d'une topologie est appelé \textbf{espace topologique}.
\end{dfn}

\begin{ex}~
\begin{itemize}
    \item Topologie discrète: $ \mathcal P(X)$
    \item $ \left\{ \emptyset, X \right\} $ (topologie triviale)
    \item Si $(E, d)$ est un espace métrique alors l'ensemble des ouverts définis avec la distance est une topologie.
\end{itemize}
\end{ex}

\begin{dfn}[Base d'ouverts\index{base d'ouverts}]
Une base d'ouverts est un ensemble $\mathcal  B$ d'ouverts de $X$ tel que tout ouvert de  $X$ est une réunion d'éléments de  $\mathcal  B$.
\end{dfn}

\begin{ex}
Dans un espace métrique, l'ensemble des boules ouvertes est une base d'ouverts.
\end{ex}

\begin{dfn}
    Un espace topologique est dit métrisable\index{espace métrisable} s'il existe une distance qui induit cette topologie. Deux distances qui donnent la même topologie sont dites topologiquement équivalentes\index{equivalence topologique@équivalence topologique}\index{distances équivalentes!topologiquement}
\end{dfn}

\begin{exo}
    $X=]0, 1]$ muni de  $d_1(x, y)=|x-y|$ et  $d_2(x, y)=|\sfrac1x-\sfrac1y|$. Montrer que ce sont des distances, qu'elles ne sont pas équivalentes, mais qu'elles sont topologiquement équivalentes. Montrer que $(X, d_2)$ est complet mais pas $(X, d_1)$.
\end{exo}

\begin{dfn}
    Si $X$ et  $Y$ sont des espaces topologiques, et  $f:X \longrightarrow Y$ est une fonction, alors on dit que $f$ est continue\index{continuité} si  $f^{-1}(U)$ est ouvert pour tout ouvert $U$. De même, on peut définir la connexité et la connexité par arcs sur des espaces topologiques.
\end{dfn}

\begin{dfn}
On dit qu'un espace topologique $X$ est séparé\footnotemark si pour tous $x\neq y \in X$, il existe un voisinage de $x$ et un voisinage de $y$ qui ne se rencontrent pas.
\end{dfn}

\footnotetext{Attention: aucun rapport avec la notion de séparabilité}

\begin{ex}
Un espace métrisable est séparé
\end{ex}

\begin{ex}[Cas particulier de la topologie de Zariski]
On peut munir $\R$ de la topologie qui contient $\emptyset$ et les complémentaires de parties finies. Cet espace n'est pas séparé.
\end{ex}

\begin{dfn}[Topologie produit\index{topologie produit}]
Si $X$ et $Y$ sont deux espaces topologiques, alors la \textbf{topologie produit}  sur $X \times Y$ est la topologie qui a pour base d'ouverts les parties de $X \times Y$ de la forme $U \times V$ avec $U$ ouvert de $X$ et $V$ ouvert de $Y$.
\end{dfn}

\begin{rem}
La distance produit donne la topologie produit.
\end{rem}
