\ifsolo
    ~

    \vspace{1cm}

    \begin{center}
        \textbf{\LARGE $\sigma$-algèbres et mesures} \\[1em]
    \end{center}
    \tableofcontents
\else
    \chapter{$\sigma$-algèbres et mesures}

    \minitoc
\fi
\thispagestyle{empty}

\section{Définitions}

\begin{dfn}
    Si $X$ est un ensemble,  $\mathcal  A \subset \mathcal  P(X)$ est une algèbre\index{algèbre} sur $X$ si  \begin{itemize}
        \item $\emptyset \in  \mathcal  A$
        \item $\forall  A, B \in  \mathcal  A, A \cup B \in  \mathcal  A$
        \item $\forall  A \in  \mathcal  A, A^c \in \mathcal  A$
    \end{itemize}
\end{dfn}

\begin{rem}
C'est aussi stable par intersection finie
\end{rem}

\begin{ex}~
\begin{itemize}
    \item $\mathcal  A= \left\{ \emptyset, X \right\} $
    \item $\mathcal  A = \mathcal  P(X)$
    \item $\left\{ A \subset X, A \text{ fini ou }A^c \text{ fini } \right\} $
    \item $\left\{ A \subset X, A \text{ fini ou dénombrable ou }A^c \text{ fini ou dénombrable } \right\} $ 
    \item $B$ fixé,  $\left\{ A \supset B \text{ ou } A\cap B=\emptyset\right\} $
    \item Dans $\R$, $\left\{ \text{unions finies d'intervalles} \right\} $
\end{itemize}
\end{ex}

\begin{dfn}
    $A\subset \mathcal  P(X)$ est une $\sigma$-algèbre (ou tribu\index{tribu}) si et seulement si
    \begin{itemize}
        \item $\emptyset \in  \mathcal  A$
        \item $\displaystyle \forall  (A_i \in  \mathcal A, i \in  \N), \bigcup_{i \in  \N} A_i \in  \mathcal  A$
        \item $A^c \in  \mathcal  A, \forall  A \in  \mathcal  A$
    \end{itemize}
\end{dfn}

\begin{rem}
\begin{itemize}
    \item Les $\sigma$-algèbres sont des algèbres
    \item $X \in  A$
    \item Stable par intersection dénombrable
\end{itemize}
\end{rem}

\section{$\sigma$-algèbre  engendrée}

\begin{lmm}
Une intersection quelconque de $\sigma$-algèbres est une $\sigma$-algèbre.
\end{lmm}

\begin{dfn}
    La $\sigma$-algèbre engendrée par  $\mathcal  M \subset \mathcal  P(X)$ est \[
        \sigma(\mathcal  M)=\bigcap_{\substack{\mathcal A \text{ tribu }\\ \mathcal  M \subset \mathcal  A}}\mathcal  A
    \] 
\end{dfn}

\begin{dfn}
    Si $X$ a une topologie, on appelle  $\sigma$-algèbre borélienne  $B(X)$ la  $\sigma$-algèbre engendrée par les ouverts.
\end{dfn}
