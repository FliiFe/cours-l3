\ifsolo
    ~

    \vspace{1cm}

    \begin{center}
        \textbf{\LARGE Intégration} \\[1em]
    \end{center}
    \tableofcontents
\else
    \chapter{Intégration}

    \minitoc
\fi
\thispagestyle{empty}

\section{Fonctions simples}

\begin{dfn}
    Soit $(X, \mathcal  A, \mu)$ un espace mesuré. Une fonction simple\index{fonction simple} sur $X$ est une fonction  $X\longrightarrow [0, +\infty[$ mesurable qui prend un nombre fini de valeurs.
\end{dfn}

\begin{rem}
    Si on note $\alpha_i$ les valeurs et $A_i=f^{-1}(\left\{ \alpha_i \right\} )$, alors \[
        s=\sum_{i=1}^n \alpha_i \times \1_{A_i}
    \] 
    Cette écriture est unique si on impose que les $ \alpha_i$ soient deux à deux distincts et de préimage non vide.
\end{rem}

\begin{prop}
    Si $f:(X, \mathcal  A)\longrightarrow [0, +\infty]$ est mesurable alors il existe $(s_k)$ une suite de fonctions simples telle que  $s_k(x)\leq s_{k+1}(x)$ pour tout $x$
\end{prop}

\begin{prop}
\[
    \forall  i \in \llbracket 1, k2^k \rrbracket, A_i^k= f^{-1} \left( \left[ \frac{i-1}{2^k}, \frac{i}{2^k}  \right[ \right)
\] 
et $B^k=f^{-1}([k, +\infty]) \in  \mathcal  A$ car $f$ est mesurable. On pose \[
    s_k\defeq\sum_{i=1}^{k2^k} \frac{i-1}{2^k} \1_{A_i^k}+k\1_{B^k}
\] 
qui convient. La suite $(s_k)$ est bien croissante, et:  \begin{itemize}
    \item Si $f(x)=+\infty$ alors  $s_k(x)=k \longrightarrow +\infty$
    \item Si $f(x)<+\infty$ alors on prend $k>f(x)$ de sorte que  \[
            0\leq f(x)-s_k(x)<\frac{1}{2^k}
    \] 
\end{itemize}
\end{prop}

\begin{rem}
Si $f$ est bornée, alors la convergence est uniforme
\end{rem}

\section{Produit sur \texorpdfstring{$[0,+\infty]$}{[0,+inf]}}

L'addition est continue sur $[0, +\infty]$. On définit la multiplication comme la multiplication usuelle de  $\R$, puis si $a$ ou  $b$ vaut  $0$, on impose  $ab=0$, et sinon  c'est $+\infty$ qui prime. Cette multiplication n'est pas continue ($\frac{n}{n=1}$ ne tend pas vers $+\infty\times 0=0$)

\section{Intégrale de fonction simple}

\begin{dfn}
    Soit $(X, \mathcal  A, \mu)$ un espace mesuré, $s$ une fonction simple. On définit l'intégrale\index{intégrale} de  $s$ pour la mesure  $\mu$ sur  $E \in  \mathcal  A$, notée \[
        \int_E s\diff \mu=\sum_{i=1}^n \alpha_i \mu(A_i\cap E)
    \] 
    où  on écrit avec la condition d'unicité mentionnée \[
        s=\sum \alpha_i \1_{A_i}
    \] 
\end{dfn}

\begin{prop}
L'intégrale est additive
\end{prop}

\begin{prop}
\[
    s=\sum_{i=1}^n \alpha_i \1_{A_i} \qquad  t=\sum_{j=1}^m \beta_j \1_{B_j}
\] 
On note $C_{i,j}=A_i\cap B_j$. Puis \[
    s+t=\sum_{i, j} (\alpha_i+\beta_j)\1_{C_{i,j}}
\] 
Pour satisfaire la condition d'unicité, il reste à regrouper les ensembles $C_{i,j}$ et $C_{i',j'}$ si $\alpha_i+\beta_j=\alpha_{i'}+\beta_{j'}$ et se débarrasser des $C_{i,j}$ vides.
\end{prop}

\begin{cor}
    Si $s=\sum \alpha_i\1_{A_i}$ même sans condition d'unicité, alors \[
        \int_E s\diff \mu=\sum \alpha_i\mu(A_i\cap E)
    \] 
\end{cor}

\begin{prop}
\begin{itemize}
    \item L'intégrale est croissante
    \item $\displaystyle \int_E s\diff \mu=\int_X s\1_E\diff \mu$ 
    \item $c \in  \R_+^\star$, alors $\displaystyle \int_E cs\diff \mu=c\int_Es\diff \mu$
    \item  $\varphi:E \in  \mathcal  A \longmapsto \int_E s\diff \mu$ est une mesure sur $(X, \mathcal  A)$ (très important)
\end{itemize}
\end{prop}

\begin{proof}
\begin{itemize}
    \item On décompose sur les $A_i\cap B_j$ 
    \item Découle de la définition
    \item Idem
    \item On a bien $\varphi(\emptyset)=0$. Puis, si les  $E_k$ sont deux à deux distincts,  \[
            \varphi \left( \bigcup_{k \in  \N}E_k \right)=\sum_{i=1}^n \alpha_i \mu\left(A_i\cap \bigcup_{k \in  \N}E_k\right)=\sum_{i=1}^n \alpha_i \mu\left(\bigcup_{k \in  \N}A_i\cap E_k\right)=\sum_{i=1}^n \alpha_i \sum_{k \in  \N} \mu(A_i\cap E_k)
    \] 
    On peut intervertir les deux sommes car la somme est continue. Donc, \[
        \varphi \left( \bigcup_{k \in  \N}E_k \right)=\sum_{k \in  \N}\int_Es\diff \mu
    \] 
\end{itemize}
\end{proof}

\section{Intégration des fonctions à valeurs dans \texorpdfstring{$[0, +\infty]$ }{[0,+inf]}}

\begin{dfn}
    Si $(X, \mathcal  A, \mu)$ est un espace mesuré et $f:(X, \mathcal  A, \mu)\longrightarrow [0, +\infty]$ est mesurable, alors pour $E \in  \mathcal  A$, on appelle intégrale\index{intégrale} de $f$ sur $E$ pour $\mu$ notée \[
    \int_E f\diff \mu
    \]
    la borne supérieure \[
    \sup \left\{ \int_E s\diff \mu, \quad  s \text{ fonction simple } \leq f \right\} \in  [0, +\infty]
    \] 
\end{dfn}

\begin{thm}[Inégalité de Tchebychev\index{inégalité de Tchebychev}]
    Si $f$ est mesurable est positive, \[ \mu(f^{-1}([\alpha, +\infty])) \leq \frac{\int_Xf\diff \mu}{\alpha} \]
\end{thm}

\begin{proof}
    $\alpha \times \1_{f^{-1} ([\alpha, +\infty])}$ est simple et $\leq f$ donc \[
        \int_Xf\diff u\geq \mu(f^{-1} ([\alpha, +\infty]))\times \alpha
    \] 
\end{proof}

\begin{cor}
    Si $\int_Xf\diff \mu<+\infty$ alors  $\mu(f^{-1}(\left\{ +\infty \right\} ))=0$. Pour le voir, $ \alpha\longrightarrow +\infty$ donne \[
        \mu(f^{-1}(\left\{ +\infty \right\} ))\leq \mu(\left\{ f^{-1}([\alpha, +\infty]) \right\} )\leq \frac{\int_Xf\diff \mu}{\alpha} \xrightarrow[\alpha \to +\infty]{}0
    \]
\end{cor}

\begin{prop}
\begin{enumerate}
    \item Si $f\leq g$ alors $\int_E f\diff \mu\leq \int_Eg\diff \mu$
    \item Si $E\subset F$ alors  $\int_E f\diff \mu \leq  \int_Ff\diff\mu$
    \item $\int_Ef\diff \mu=\int_Xf\1_E\diff \mu$
    \item  $c \in  [0, +\infty]$ alors $\int_Ecf\diff\mu=c\int_Ef\diff\mu$
    \item  $f\equiv 0$ sur  $E$ alors  $\inf_Ef\diff \mu=0$
    \item  $\mu(E)=0 \implies \int_Ef\diff \mu=0$
\end{enumerate}
\end{prop}

\begin{proof}
\begin{enumerate}
    \item Sup sur un surensemble
    \item Clair
    \item $s$ simple et  $\leq f$. Alors $s\times \1_E\leq f\1_E$ et réciproque facile aussi.
    \item Si $c$ est fini, cf. fonctions simples. Pour le cas  $c=+\infty$, exercice (facile). 
    \item Clair
    \item Clair
\end{enumerate}
\end{proof}

\begin{thm}[Beppo Levi\index{Beppo-Levi}]
    Soit $f_k:(X, \mathcal  A, \mu)\longrightarrow [0, +\infty]$ des fonctions mesurables telles que $(f_k)$ est croissante. Soit  $f$ la limite simple des  $f_k$. Alors,  \[
        \forall  E \in  \mathcal  A, \inf_Ef\diff \mu=\lim_{k\to +\infty}\int_Ef_k\diff \mu
    \] 

\end{thm}

\begin{proof}
    En écrivant $\int_E\diff\mu=\int_X\1_E\diff\mu$, il suffit de le faire pour  $E=X$.  \[
        f_k\leq f_{k+1}\leq f \implies \left( \int_X f_k\diff\mu \right) \text{ est croissante  } \leq \int_Xf\diff\mu
    \] 
    Soit $\alpha=\lim\int_Xf_k\diff \mu\leq \int_X f\diff \mu$. Soit $s$ simple et  $\leq f$ et $c \in  ]0,1[$. On note \[
        E_n = \left\{ x / f_n(x)\geq cs(x) \right\} 
    \] 
    Les $E_n$ sont dans  $\mathcal  A$, forment une suite croissante et $\cup E_n=X$. Si  $f(x)=+\infty$ alors $x \in  E_n$ pour tout $n$ assez grand. Idem pour $f(x)<+\infty$.

    La fonction  $\varphi(E)=\int_E s\diff\mu$ est une mesure et  \[
        \varphi(E_n)=\underbrace{\int_{E_n}cs\diff\mu}_{\xrightarrow{}\int_Xcs\diff\mu}\leq \int_{E_n}f_n\diff\mu\leq \int_X f_n\diff\mu \xrightarrow[n\to +\infty]{\text{croissance}}\alpha
    \] 
    donc \[
    c\int_Xs\diff \mu\leq \alpha \implies \int_Xs\diff \mu\leq \alpha
    \]
    d'où en passant au sup sur $s$,  \[
    \int_Xf\diff\mu\leq \alpha
    \]
    On a donc \[
    \int_Xf\diff\mu = \alpha =\lim \int_X f_n\diff\mu
    \] 
\end{proof}

\begin{cor}
    Si $f$ et  $g$ sont à valeurs dans  $[0, +\infty]$ alors  \[
        \int_E(f+g)\diff \mu=\int_Ef\diff\mu+\int_E g\diff\mu
    \] 
    et \[
        \int_E \left(\sum_{k \in  \N} f_k\right)\diff\mu=\sum_{k \in  \N}\int_Ef_k\diff\mu
    \] 
\end{cor}

\begin{proof}
    On écrit $f=\lim s_k$ avec  $(s_k)$ croissante, idem $g=\lim t_k$.  Alors $s_k+t_k$ tend simplement vers  $f+g$ et Beppo-Levi donne:  \[
        \int_E(f+g)\diff\mu=\lim\int_E(s_k+t_k)\diff\mu=\lim \left(\int_Es_k\diff \mu+\int_Et_k\diff\mu\right)
    \] 
    et Beppo-Levi sur $(s_k)$ et sur  $(f_k)$ conclut. C'est idem pour les séries.
\end{proof}

\begin{lmm}[Lemme de Fatou\index{Fatou (lemme)}]
    Soient $f_k:(X, \mathcal  A, \mu)\longrightarrow [0, +\infty]$ des fonctions mesurables. \[
    \int_E \liminf_{k\to +\infty}f_k\leq \liminf_{k\to +\infty} \int_E f_k\diff\mu
    \] 
\end{lmm}

\begin{proof}
On écrit la limite inférieure comme une limite de $g_k$ et Beppo-Levi.
\end{proof}
