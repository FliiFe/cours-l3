\ifsolo
    ~

    \vspace{1cm}

    \begin{center}
        \textbf{\LARGE Groupes} \\[1em]
    \end{center}
    \tableofcontents
\else
    \chapter{Groupes}

    \minitoc
\fi
\thispagestyle{empty}

\section{Définitions}

\begin{dfn}
    Un groupe\index{groupe} est un triplet $(G, \star, e)$ où  $G$ est un ensemble,  $\star$ est une loi de composition interne associative,  $e$ est neutre pour  $\star$ et tous les éléments de  $G$ possèdent un inverse pour  $\star$.
\end{dfn}

\begin{rem}
\begin{itemize}
    \item L'élément neutre est unique
    \item L'inverse est unique (on note $a ^{-1}$ l'inverse de $a$)
    \item On dit que $G$ est abélien ou commutatif si $ \forall  a,b \in  G, a\star b=b\star a$
    \item On appelle ordre de $G$ son cardinal
\end{itemize}
\end{rem}

\begin{dfn}
    $H\subset G$ est un sous-groupe\index{sous-groupe} de  $G$ si  $e \in  H$ et la restriction de la loi de $G$ à  $H$ est une loi de groupe
\end{dfn}

\begin{rem}
C'est équivalent à \begin{itemize}
    \item $e \in  H$
    \item $\forall  x,y \in  H, xy \in H$
    \item $\forall  x \in  H, x^{-1} \in  H$
\end{itemize}
Ces deux dernières conditions peuvent encore s'écrire $\forall  x,y \in  H, xy^{-1}\in H$.
\end{rem}

\begin{prop}
Les sous-groupes de $\Z$ sont de la forme $n\Z$ pour $n \in  \N$.
\end{prop}

\begin{proof}
    Les $n\Z$ sont des sous-groupes. Si $H\subset \Z$ est un sous-groupe non trivial alors $n=\inf (H\cap \N^\star) $ est tel que $n\Z\subset H$ et par division euclidienne, $n\Z=H$ (si $x=qn+r$ est la division de  $x \in  H$ par $n$, alors $r\in H$ et $0\leq r<n$ donc $r=0$ et  $x \in  n\Z$)
\end{proof}

\begin{prop}
Une intersection de sous-groupes est un sous-groupe.
\end{prop}

\begin{dfn}
Si $X\subset G$, le sous-groupe engendré par  $X$, noté  $\langle X\rangle$ est l'intersection de tous les sous-groupes contenant  $X$.
\end{dfn}

\begin{prop}
    $\langle X\rangle$ est l'ensemble des éléments de  $G$ qui s'écrivent  $x_1^{n_1}\cdots x_p^{n_p}$, $p \in \N, n_i \in  \Z, x_i \in  X$.
\end{prop}

\begin{dfn}[Groupe monogène]
    Un groupe est monogène\index{monogène} s'il existe $x \in  G$ tel que $\langle x\rangle =G$. Si  $G$ est fini et monogène, on dira qu'il est cyclique\index{cyclique}.
\end{dfn}

\section{Morphismes de groupes}

\begin{dfn}
    Soient $G, G'$ deux groupes. Un morphisme\index{morphisme} de groupes $G \longrightarrow  G'$ est une application $f:G\longrightarrow G'$ qui vérifie $f(e)=e'$ et  $f(x\cdot y)=f(x)\cdot f(y)$ pour tous  $x,y \in  G$.
\end{dfn}

\begin{rem}
    Automatiquement, $f(x^{-1})=f(x)^{-1}$
\end{rem}

\begin{dfn}
    On appelle noyau\index{noyau} du morphisme $f$ l'ensemble  $f^{-1} (\left\{ e' \right\} )$, et image l'ensemble  $f(G)$.
\end{dfn}

\begin{prop}
L'image et le noyau d'un morphismes sont des sous-groupes
\end{prop}

\begin{prop}
$f$ est injective  si et seulement si $\ker f= \left\{ e \right\} $, surjective si et seulement si $\im f=G'$.
\end{prop}

\begin{dfn}
    Un morphisme bijectif est appelé isomorphisme\index{isomorphisme}
\end{dfn}

\begin{prop}
L'inverse d'un isomorphisme est un morphisme
\end{prop}

\begin{prop}
Si $G$ est un groupe monogène, alors  $G\cong \Z$ ou $G\cong {\Z} / {n\Z}$ pour un $n \in  \N^\star$.
\end{prop}

\begin{proof}
    Si  $G=\langle x\rangle$, on pose  $f(n)=x^n$. C'est un morphisme de groupes surjectif. Si  $f$ est injectif, alors  $G\cong \Z$. Sinon $\ker f=n\Z$ pour un $n \in  \N^\star$, et ${\Z} / {n\Z}\cong G$
\end{proof}

\section{Classe définie par un sous-groupe}

\begin{defprop}
Si $H\subset G$ est un sous-groupe, alors on note  $x\sim y \iff  \exists h \in  H, x=yh$. C'est une relation d'équivalence (facile) et on note $G/H$ l'ensemble des classes d'équivalence pour cette relation. Ce sont les classes à gauche La classe de $x$ s'écrit  $xH$. 

De même, on peut définir les classes à droite.
\end{defprop}

\begin{dfn}
Si $G / H$ est fini, on dit que  $H$ est d'indice fini dans  $G$ et on note  \[
    [G:H]=\# \sfrac GH
\] 
l'indice\index{indice} de $H$ dans  $G$. Sinon, on dit que  $H$ est d'indice infini.
\end{dfn}

\begin{prop}
Toutes les classes à gauche modulo $H$ sont en bijection avec $H$.
\end{prop}

\begin{proof}
    $f_a:aH\longrightarrow H, x\longmapsto a^{-1}x$ convient (bijectif car $f_a\circ f_{a ^{-1}}=\id$ et $f_{a^{-1}}\circ f_a=\id$).
\end{proof}

\begin{cor}
    Si $G$ est fini, alors  \[\#G=[G:H]\cdot \#H\]
\end{cor}

\begin{thm}[Lagrange]
Si $G$ est fini et  $H$ est un sous-groupe de  $G$ alors  $\#H$ divise $\#G$
\end{thm}

\begin{cor}
Si $x \in  G$ et $G$ est fini alors l'ordre de  $x$ divise $\#G$
\end{cor}

\section{Sous-groupe distingué}

\begin{dfn}
    $H$ est un sous-groupe distingué\index{sous-groupe distingué}\index{distingué (sous-groupe)} de  $G$  si et seulement si c'est un sous-groupe et $\forall  a \in G, aH=Ha$. On note dans ce cas $H\dist G$
\end{dfn}

\section{Exemples de sous-groupes distingués}

\begin{prop}
Si $f$ est un morphisme de groupes, alors  $\ker f$ est distingué
\end{prop}

\begin{dfn}[Centre]
    Le centre d'un groupe\index{centre (groupe)} $G$ est le sous-groupe  \[
            Z(G)= \left\{ x \in  G, \qquad  \forall  y \in  G, xy=yx \right\} 
    \]
\end{dfn}
\begin{prop}
    Le centre d'un groupe est un sous-groupe distingué
\end{prop}

\begin{dfn}[Commutateur]
    Soit $G$ un groupe. On définit le commutateur\index{commutateur} de  $x, y \in  G$ par \[
        [x, y]=xyx^{-1}y^{-1} \in  G
    \] 
\end{dfn}

\begin{rem}
    Dans un groupe $G$,  $x, y \in G$ commutent si et seulement si $[x, y]=e$
\end{rem}

\begin{dfn}[Groupe dérivé]
    Soit $G$ un groupe. On appelle groupe dérivé de  $G$ le sous-groupe $D(G)$ engendré par tous les commutateurs:  \[
        D(G) = \langle [x, y], \quad x,y \in  G\rangle
    \] 
\end{dfn}

\begin{defprop}
Pour $z \in  G$, l'application \[
\begin{array}{rrcl}
    f_z:& G & \longrightarrow & G \\
    & x & \longmapsto & \displaystyle zxz^{-1}
\end{array}
\] 
est un automorphisme, qu'on appellera automorphisme intérieur\index{automorphisme intérieur}
\end{defprop}

\begin{prop}
   Le groupe dérivé $D(G)$ est un sous-groupe distingué de  $G$
\end{prop}

\begin{proof}
    Soient $x, y, z\in  G$. Alors $f_z([x, y])=f_z(x)f_z(y)f_z(x)^{-1}f_z(y)^{-1}=[f_z(x), f_z(y)]$. Si $X$ est l'ensemble des commutateurs de  $G$, alors pour tout  $z \in  G$, $f_z(X)\subseteq D(G)$. Puis,  $D(G) = \langle X\rangle$ donc  $f_z(\langle X\rangle)=\langle f_z(X)\rangle\subset D(G)$.
\end{proof}

\begin{rem}
    $D(G) = \left\{ e \right\} $ si et seulement si $G$ est abélien
\end{rem}

\begin{dfn}[Normalisateur]
    Soit $G$ un groupe et  $H$ un sous-groupe de  $G$. On dit qu'un élément  $a \in  G$ normalise $H$ si  $aHa^{-1}=H$. On appelle normalisateur\index{normalisateur} de $H$ le groupe des éléments qui normalisent  $H$, qu'on note  $N_G(H)$
\end{dfn}

\begin{prop}
\begin{enumerate}
    \item $N_G(H)$ est un sous-groupe de  $G$ qui contient  $H$
    \item  $H\dist N_G(H)$
    \item Si  $N$ est un sous-groupe de  $G$ tel que  $N\dist H\subset G$ alors  $N\subset N_G(H)$
\end{enumerate}
\end{prop}

\begin{proof} ~
\begin{enumerate}
    \item Si $x, y \in  N_G(H)$ alors $xyH(xy)^{-1}=f_x\circ f_y(H)=H$ et $x^{-1} H x=x^{-1} (xHx^{-1} )x=H$. Puis si $h \in  H, hHh^{-1}=H$.
    \item Résulte de la définition.
    \item Tous les éléments de $N$ normalisent  $H$.
\end{enumerate}
\end{proof}

\section{Groupes simples}
\begin{dfn}
    Un groupe $G$ est dit simple\index{groupe simple}\index{simple (groupe)} si  $G\neq \left\{ e \right\} $ et les seuls sous-groupes distingués de $G$ sont  $\left\{ e \right\} $ et $G$.
\end{dfn}

\begin{ex}
\begin{itemize}
    \item Si $p$ est premier, alors  $G = \Z / p\Z$ est un groupe simple (Lagrange). En fait, $\Z / n\Z$ est simple si et seulement si $n$ est premier.
    \item  $\Z$ n'est pas simple car ses sous-groupes sont distingués (et il en existe un strict, $2\Z$)
    \item On verra plus tard des exemples de groupes simples non abéliens, finis et infinis.
\end{itemize}
\end{ex}

\section{Groupes quotient}

Soit $G$ un groupe, et  $H\dist G$.

 \begin{thmdef}
Il existe une unique structure de groupe sur $G / H$ telle que l'application canonique  $\pi : G \longrightarrow G / H$ est un morphisme de groupes. De plus, $\ker \pi=H$
\end{thmdef}

\begin{proof}
\begin{itemize}
    \item Existence. Soient $\alpha, \beta \in  G / H$ et $a, b \in G$ des représentants respectifs. Posons $ \alpha \beta = abH \in  G / H$. Vérifions que cela ne dépend pas du choix de $a$ et  $b$. Pour cela, on remarque que  $aH.bH=a(Hb).H=a(bH).H=abH.H=abH$ ($.$ désigne le produit ensembliste) donc $ \alpha \beta = \alpha . \beta$.

        C'est bien une loi de groupe car $H$ est  neutre, la loi est associative car le produit ensembliste l'est,  $ \alpha \beta=abH$ est bien une classe et $ a^{-1}H$ est l'inverse de $ \alpha=aH$.

        L'application $\pi$ est bien un morphisme de groupes car $\pi(ab)=abH=(aH)(bH)=\pi(a)\pi(b)$. Puis,  $\ker \pi = \pi^{-1}(H)=H$.
    \item L'unicité de la structure de groupe découle formellement de la surjectivité de $\pi$: la structure de $G$ est transférée sur  $G / H$ tout entier. Si $\star$ est une autre loi de groupe sur  $G / H$ vérifiant l'énoncé du théorème alors  $\pi(ab)=(aH)\star(bH)=abH$
\end{itemize}
\end{proof}


\begin{rem}
Pour $G$ abélien, tout sous-groupe  $H$ est distingué, donc  $G / H$ existe et hérite de la commutativité de  $G$
\end{rem}

\begin{prop}
Soit $G$ un groupe et  $H$ un sous-groupe de  $G$. Alors,  $H$ est distingué si et seulement si il existe  un groupe $G'$ et un morphisme de groupe  $f:G\longrightarrow G'$ tel que $H=\ker f$.
\end{prop}

\begin{proof}
Si $H$ est distingué, alors  $\pi$ convient. Si $H$ n'est pas distingué, alors il n'est le noyau d'aucun morphisme puisque les noyaux des morphismes sont toujours distingués.
\end{proof}

\begin{thm}[Passage au quotient]
Soit $f : G \longrightarrow  G'$ un morphisme de groupes. Soit $H$ un sous-groupe distingué de $G$ contenu dans $\ker f$. Alors il existe un unique morphisme de groupes $\bar{f} : G / H \longrightarrow G'$ tel que le diagramme suivant commute
\begin{center}
    \begin{tikzcd}[sep=large,every label/.append style={font=\normalsize}]
    G \arrow[r, "f"] \arrow[rd, "\pi", two heads] & G' \\
                                       & G/H \arrow[u, " \bar f", dashed]
\end{tikzcd}
\end{center}
\end{thm}

\begin{proof}
    Si $\pi(a)=\pi(b)$, on a  $a=bh$ pour un  $h \in  H$ donc $f(a)=f(b)f(h)=f(b)$ donc  $f$ passe au quotient ensemblistement, et il existe bien  $\bar{f}$ telle que $f=\bar{f} \circ \pi$. C'est bien un morphisme car $f$ est un morphisme et  $\pi$ aussi:  $\bar{f}(\alpha \beta)=f(ab)=f(a)f(b)=\bar{f}(\alpha)\bar{f}(\beta)$. Finalement, l'unicité est donnée par le passage au quotient ensembliste.
\end{proof}

\begin{cor}[Premier théorème d'isomorphisme]
Soit $f : G \longrightarrow G'$ un morphisme de groupe. Alors, $f$ induit un isomorphisme canonique  $\bar{f}:G/\ker f \longrightarrow \im f$
\end{cor}

\begin{cor}
Si $G$ est un groupe fini, et  $f:G\longrightarrow G'$ est un morphisme, alors $\#G=\#\ker f \cdot \#\im f$
\end{cor}

\section{Sous-groupes d'un quotient}

\begin{thm}
Soit $G$ un groupe et  $H\dist G$. Alors, il y a une bijection canonique entre l'ensemble des sous-groupes de  $G / H$ et l'ensemble des sous-groupes de  $G$ contenant  $H$, donnée par \[
    K \subset G / H \longmapsto \pi^{-1}(K)
\] 
d'inverse \[
    K' \subset G \longmapsto \pi(K')
\] 
\end{thm}

\begin{proof}
Ces applications sont inverses l'une de l'autre: \[
    \forall  K\subset G / H, \qquad  \pi(\pi^{-1}(K))=K
\] 
et si $K'$ est un sous-groupe contenant  $H$,  \[
    \pi^{-1}(\pi(K'))=K'
\] 
car $x \in \pi^{-1}(\pi(K'))$ s'écrit $x=yh$ avec  $y \in  K', h \in  H\subset K'$ donc $x \in  K'$ et  l'autre inclusion est claire.
\end{proof}

\begin{rem}
Cette bijection est croissante pour l'inclusion et elle préserve les sous-groupes distingués
\end{rem}

\section{Suite exacte de groupes}

\begin{dfn}
    Une suite exacte courte\index{suite exacte courte} est la donnée de trois groupes $G', G, G''$ et de deux morphismes de groupes  $\alpha:G'\longrightarrow G$ et $\beta:G \longrightarrow G''$ qui vérifient \begin{itemize}
        \item $\alpha$ est injectif
        \item $\im \alpha = \ker \beta$
        \item $\beta$ est surjectif.
    \end{itemize}
    On note \[
        1 \longrightarrow G' \overset{\alpha}\longrightarrow G\overset\beta\longrightarrow G''\longrightarrow 1
    \] 
\end{dfn}

\begin{ex}~
\begin{itemize}
    \item Suite exacte triviale: $G = G'\times G''$ \[
        1 \longrightarrow G' \overset{x\mapsto(x,1)}\longrightarrow G'\times G''\overset{(x,y)\mapsto y}\longrightarrow G''\longrightarrow 1
    \]
    \item \[
            0 \longrightarrow \Z \overset{\times n}\longrightarrow \Z \overset\pi\longrightarrow \Z / n\Z \longrightarrow 0
    \] 
    \item Plus généralement, pour tout groupe $G$, et tout $H\dist G$, la suite suivante est une suite exacte courte \[
            1 \longrightarrow H \overset{x\mapsto x}\longrightarrow G\overset\pi\longrightarrow G / H \longrightarrow 1
    \] 
\end{itemize}
\end{ex}

\begin{rem}
Si $G$ est un groupe simple, il n'existe pas de suite exacte courte avec  $G'$ et  $G''$ non triviaux.
\end{rem}
