\ifsolo
    ~

    \vspace{1cm}

    \begin{center}
        \textbf{\LARGE Intégration} \\[1em]
    \end{center}
    \tableofcontents
\else
    \chapter{Intégration}

    \minitoc
\fi
\thispagestyle{empty}

\section{Fonctions simples}

\begin{dfn}
    Soit $(X, \mathcal  A, \mu)$ un espace mesuré. Une fonction simple\index{fonction simple} sur $X$ est une fonction  $X\longrightarrow [0, +\infty[$ mesurable qui prend un nombre fini de valeurs.
\end{dfn}

\begin{rem}
    Si on note $\alpha_i$ les valeurs et $A_i=f^{-1}(\left\{ \alpha_i \right\} )$, alors \[
        s=\sum_{i=1}^n \alpha_i \times \1_{A_i}
    \] 
    Cette écriture est unique si on impose que les $ \alpha_i$ soient deux à deux distincts et de préimage non vide.
\end{rem}

\begin{prop}
    Si $f:(X, \mathcal  A)\longrightarrow [0, +\infty]$ est mesurable alors il existe $(s_k)$ une suite de fonctions simples telle que  $s_k(x)\leq s_{k+1}(x)$ pour tout $x$
\end{prop}

\begin{prop}
\[
    \forall  i \in \llbracket 1, k2^k \rrbracket, A_i^k= f^{-1} \left( \left[ \frac{i-1}{2^k}, \frac{i}{2^k}  \right[ \right)
\] 
et $B^k=f^{-1}([k, +\infty]) \in  \mathcal  A$ car $f$ est mesurable. On pose \[
    s_k\defeq\sum_{i=1}^{k2^k} \frac{i-1}{2^k} \1_{A_i^k}+k\1_{B^k}
\] 
qui convient. La suite $(s_k)$ est bien croissante, et:  \begin{itemize}
    \item Si $f(x)=+\infty$ alors  $s_k(x)=k \longrightarrow +\infty$
    \item Si $f(x)<+\infty$ alors on prend $k>f(x)$ de sorte que  \[
            0\leq f(x)-s_k(x)<\frac{1}{2^k}
    \] 
\end{itemize}
\end{prop}

\begin{rem}
Si $f$ est bornée, alors la convergence est uniforme
\end{rem}

\section{Produit sur \texorpdfstring{$[0,+\infty]$}{[0,+inf]}}

L'addition est continue sur $[0, +\infty]$. On définit la multiplication comme la multiplication usuelle de  $\R$, puis si $a$ ou  $b$ vaut  $0$, on impose  $ab=0$, et sinon  c'est $+\infty$ qui prime. Cette multiplication n'est pas continue ($\frac{n}{n=1}$ ne tend pas vers $+\infty\times 0=0$)

\section{Intégrale de fonction simple}

\begin{dfn}
    Soit $(X, \mathcal  A, \mu)$ un espace mesuré, $s$ une fonction simple. On définit l'intégrale\index{intégrale} de  $s$ pour la mesure  $\mu$ sur  $E \in  \mathcal  A$, notée \[
        \int_E s\diff \mu=\sum_{i=1}^n \alpha_i \mu(A_i\cap E)
    \] 
    où  on écrit avec la condition d'unicité mentionnée \[
        s=\sum \alpha_i \1_{A_i}
    \] 
\end{dfn}

\begin{prop}
L'intégrale est additive
\end{prop}

\begin{prop}
\[
    s=\sum_{i=1}^n \alpha_i \1_{A_i} \qquad  t=\sum_{j=1}^m \beta_j \1_{B_j}
\] 
On note $C_{i,j}=A_i\cap B_j$. Puis \[
    s+t=\sum_{i, j} (\alpha_i+\beta_j)\1_{C_{i,j}}
\] 
Pour satisfaire la condition d'unicité, il reste à regrouper les ensembles $C_{i,j}$ et $C_{i',j'}$ si $\alpha_i+\beta_j=\alpha_{i'}+\beta_{j'}$ et se débarrasser des $C_{i,j}$ vides.
\end{prop}

\begin{cor}
    Si $s=\sum \alpha_i\1_{A_i}$ même sans condition d'unicité, alors \[
        \int_E s\diff \mu=\sum \alpha_i\mu(A_i\cap E)
    \] 
\end{cor}

\begin{prop}
\begin{itemize}
    \item L'intégrale est croissante
    \item $\displaystyle \int_E s\diff \mu=\int_X s\1_E\diff \mu$ 
    \item $c \in  \R_+^\star$, alors $\displaystyle \int_E cs\diff \mu=c\int_Es\diff \mu$
    \item  $\varphi:E \in  \mathcal  A \longmapsto \int_E s\diff \mu$ est une mesure sur $(X, \mathcal  A)$ (très important)
\end{itemize}
\end{prop}

\begin{proof}
\begin{itemize}
    \item On décompose sur les $A_i\cap B_j$ 
    \item Découle de la définition
    \item Idem
    \item On a bien $\varphi(\emptyset)=0$. Puis, si les  $E_k$ sont deux à deux distincts,  \[
            \varphi \left( \bigcup_{k \in  \N}E_k \right)=\sum_{i=1}^n \alpha_i \mu\left(A_i\cap \bigcup_{k \in  \N}E_k\right)=\sum_{i=1}^n \alpha_i \mu\left(\bigcup_{k \in  \N}A_i\cap E_k\right)=\sum_{i=1}^n \alpha_i \sum_{k \in  \N} \mu(A_i\cap E_k)
    \] 
    On peut intervertir les deux sommes car la somme est continue. Donc, \[
        \varphi \left( \bigcup_{k \in  \N}E_k \right)=\sum_{k \in  \N}\int_Es\diff \mu
    \] 
\end{itemize}
\end{proof}

\section{Intégration des fonctions à valeurs dans \texorpdfstring{$[0, +\infty]$ }{[0,+inf]}}

\begin{dfn}
    Si $(X, \mathcal  A, \mu)$ est un espace mesuré et $f:(X, \mathcal  A, \mu)\longrightarrow [0, +\infty]$ est mesurable, alors pour $E \in  \mathcal  A$, on appelle intégrale\index{intégrale} de $f$ sur $E$ pour $\mu$ notée \[
    \int_E f\diff \mu
    \]
    la borne supérieure \[
    \sup \left\{ \int_E s\diff \mu, \quad  s \text{ fonction simple } \leq f \right\} \in  [0, +\infty]
    \] 
\end{dfn}

\begin{thm}[Inégalité de Tchebychev\index{inégalité de Tchebychev}]
    Si $f$ est mesurable est positive, \[ \mu(f^{-1}([\alpha, +\infty])) \leq \frac{\int_Xf\diff \mu}{\alpha} \]
\end{thm}

\begin{proof}
    $\alpha \times \1_{f^{-1} ([\alpha, +\infty])}$ est simple et $\leq f$ donc \[
        \int_Xf\diff u\geq \mu(f^{-1} ([\alpha, +\infty]))\times \alpha
    \] 
\end{proof}

\begin{cor}
    Si $\int_Xf\diff \mu<+\infty$ alors  $\mu(f^{-1}(\left\{ +\infty \right\} ))=0$. Pour le voir, $ \alpha\longrightarrow +\infty$ donne \[
        \mu(f^{-1}(\left\{ +\infty \right\} ))\leq \mu(\left\{ f^{-1}([\alpha, +\infty]) \right\} )\leq \frac{\int_Xf\diff \mu}{\alpha} \xrightarrow[\alpha \to +\infty]{}0
    \]
\end{cor}

\begin{prop}
\begin{enumerate}
    \item Si $f\leq g$ alors $\int_E f\diff \mu\leq \int_Eg\diff \mu$
    \item Si $E\subset F$ alors  $\int_E f\diff \mu \leq  \int_Ff\diff\mu$
    \item $\int_Ef\diff \mu=\int_Xf\1_E\diff \mu$
    \item  $c \in  [0, +\infty]$ alors $\int_Ecf\diff\mu=c\int_Ef\diff\mu$
    \item  $f\equiv 0$ sur  $E$ alors  $\inf_Ef\diff \mu=0$
    \item  $\mu(E)=0 \implies \int_Ef\diff \mu=0$
\end{enumerate}
\end{prop}

\begin{proof}
\begin{enumerate}
    \item Sup sur un surensemble
    \item Clair
    \item $s$ simple et  $\leq f$. Alors $s\times \1_E\leq f\1_E$ et réciproque facile aussi.
    \item Si $c$ est fini, cf. fonctions simples. Pour le cas  $c=+\infty$, exercice (facile). 
    \item Clair
    \item Clair
\end{enumerate}
\end{proof}

\begin{thm}[Beppo Levi\index{Beppo-Levi}]
    Soit $f_k:(X, \mathcal  A, \mu)\longrightarrow [0, +\infty]$ des fonctions mesurables telles que $(f_k)$ est croissante. Soit  $f$ la limite simple des  $f_k$. Alors,  \[
        \forall  E \in  \mathcal  A, \int_Ef\diff \mu=\lim_{k\to +\infty}\int_Ef_k\diff \mu
    \] 

\end{thm}

\begin{proof}
    En écrivant $\int_E\diff\mu=\int_X\1_E\diff\mu$, il suffit de le faire pour  $E=X$.  \[
        f_k\leq f_{k+1}\leq f \implies \left( \int_X f_k\diff\mu \right) \text{ est croissante  } \leq \int_Xf\diff\mu
    \] 
    Soit $\alpha=\lim\int_Xf_k\diff \mu\leq \int_X f\diff \mu$. Soit $s$ simple et  $\leq f$ et $c \in  ]0,1[$. On note \[
        E_n = \left\{ x / f_n(x)\geq cs(x) \right\} 
    \] 
    Les $E_n$ sont dans  $\mathcal  A$, forment une suite croissante et $\cup E_n=X$. Si  $f(x)=+\infty$ alors $x \in  E_n$ pour tout $n$ assez grand. Idem pour $f(x)<+\infty$.

    La fonction  $\varphi(E)=\int_E s\diff\mu$ est une mesure et  \[
        \varphi(E_n)=\underbrace{\int_{E_n}cs\diff\mu}_{\xrightarrow{}\int_Xcs\diff\mu}\leq \int_{E_n}f_n\diff\mu\leq \int_X f_n\diff\mu \xrightarrow[n\to +\infty]{\text{croissance}}\alpha
    \] 
    donc \[
    c\int_Xs\diff \mu\leq \alpha \implies \int_Xs\diff \mu\leq \alpha
    \]
    d'où en passant au sup sur $s$,  \[
    \int_Xf\diff\mu\leq \alpha
    \]
    On a donc \[
    \int_Xf\diff\mu = \alpha =\lim \int_X f_n\diff\mu
    \] 
\end{proof}

\begin{cor}
    Si $f$ et  $g$ sont à valeurs dans  $[0, +\infty]$ alors  \[
        \int_E(f+g)\diff \mu=\int_Ef\diff\mu+\int_E g\diff\mu
    \] 
    et \[
        \int_E \left(\sum_{k \in  \N} f_k\right)\diff\mu=\sum_{k \in  \N}\int_Ef_k\diff\mu
    \] 
\end{cor}

\begin{rem}
    La relation de Chasles découle de l'additivité. Si $f:(X, \mathcal  A) \longrightarrow [0, +\infty]$ et $E, F \in  \mathcal  A$ avec $E\cap F=\emptyset$ alors $\1_{E\sqcup F}=\1_E+\1_F$ donc \[
        \int_{E\sqcup F} f\diff \mu=\int_X f\1_{E\sqcup F}\diff \mu=\int_Ef\diff \mu+\int_Ff\diff\mu
    \]
\end{rem}

\begin{proof}
    On écrit $f=\lim s_k$ avec  $(s_k)$ croissante, idem $g=\lim t_k$.  Alors $s_k+t_k$ tend simplement vers  $f+g$ et Beppo-Levi donne:  \[
        \int_E(f+g)\diff\mu=\lim\int_E(s_k+t_k)\diff\mu=\lim \left(\int_Es_k\diff \mu+\int_Et_k\diff\mu\right)
    \] 
    et Beppo-Levi sur $(s_k)$ et sur  $(f_k)$ conclut. C'est idem pour les séries.
\end{proof}

\begin{lmm}[Lemme de Fatou\index{Fatou (lemme)}]
    Soient $f_k:(X, \mathcal  A, \mu)\longrightarrow [0, +\infty]$ des fonctions mesurables. \[
    \int_E \liminf_{k\to +\infty}f_k\leq \liminf_{k\to +\infty} \int_E f_k\diff\mu
    \] 
\end{lmm}


\begin{rem}
    Il n'y a pas toujours égalité. Dans $(\R, \mathcal B(\R), \lambda)$, on prend $f_k=\1_{[k, +\infty[}$. Alors $\liminf_{k\to +\infty} f_k=0$ mais \[
    \int_X f_k\diff\lambda=+\infty
    \] 
\end{rem}

\begin{proof}
% On écrit la limite inférieure comme une limite de $g_k$ et Beppo-Levi.
    On pose $g_n=\inf_{k\geq n}f_k$, c'est une suite croissante de fonctions mesurables positives et qui converge vers $\liminf f_k$. Beppo-Levi donne \[\inf_X\liminf_{k\to +\infty} f_k\diff \mu=\lim_{n\to +\infty}\int_Xg_n\diff\mu\leq \int_Xf_n\diff\mu\]
\end{proof}

\section{Intégration des fonctions à valeurs dans \texorpdfstring{$\C$}{C}}

\begin{dfn}
    Une fonction mesurable $f:(X, \mathcal  A)\longrightarrow \C$ est dite intégrable\index{intégrable}\index{fonction intégrable} si \[\int_X|f|\diff\mu<+\infty\]
    Dans ce cas, si on écrit $f=u+iv$ puis on définit  $u^+, u^-, v^+, v^-$ avec des notations évidentes, ces quatre fonctions sont mesurables et  $\leq |f|$ donc elles sont intégrables. On appelle alors intégrale de $f$ pour la mesure  $\mu$ sur  $E$ le complexe \[\int_Ef\diff\mu=\int_Eu^+\diff\mu-\int_Eu^-\diff\mu+i\int_Ev^+\diff\mu-i\int_Eu^-\diff\mu\]
\end{dfn}

\begin{rem}[Terminologie]
    Une fonction mesurable positive admet toujours une intégrale mais n'est pas toujours intégrable (elle l'est si et seulement si cette intégrale est finie).
\end{rem}

\begin{rem}
Il découle de la définition: \begin{itemize}
    \item Si $f$ est à valeurs réelles alors \[\int_Xf\diff\mu=\int_Xf^+\diff\mu-\int_Xf^-\diff\mu\] 
    \item Si $f$ est à valeurs complexes, \[\int_X f\diff\mu=\int_X\Re(f)\diff\mu+i\int_X\Im(f)\diff\mu\]
\end{itemize}
\end{rem}

\begin{lmm}
En ne considérant que des fonctions intégrables et des ensembles mesurables: \begin{itemize}
    \item $\displaystyle f\leq g \implies \int_Ef\diff\mu\leq \int_Eg\diff\mu$
    \item $\displaystyle \int_Ef\diff\mu=\int_Xd\1-E\diff\mu$
    \item  $\mu(E)=0 \implies \displaystyle \int_Ef\diff\mu=0$
    \item $f_{|E}\equiv 0 \implies  \displaystyle \int_Ef\diff\mu=0$
\end{itemize}
\end{lmm}

\begin{proof}~
\begin{itemize}
    \item Si $f\leq g$ alors $f^+\leq g^+$ et $f^- \geq g^-$ et il reste à exploiter la croissance de l'intégrale sur les fonctions positives.
    \item Pour le reste, on décompose $f$ en  $\Re(f)^+, \Re(f)^-, \Im(f)^+, \Im(f)^-$.
\end{itemize}
\end{proof}

\begin{thm}
    Soient $f, g : (X, \mathcal  A)\longrightarrow \C$ intégrables, $\alpha, \beta \in  \C$. Alors $ \alpha f+\beta g$ est intégrable et \[\int_X(\alpha f + \beta g)\diff\mu=\alpha\int_Xf\diff\mu+\beta\int_Xg\diff\mu\]
\end{thm}


\begin{thm}
    Soit $f:(X, \mathcal  A) \longrightarrow  \C$ intégrable alors \[
    \left|\int_X f\diff\mu \right|\leq \int_X |f|\diff\mu.
    \] 
\end{thm}

\begin{proof}
    Soit $ \alpha \in  \C, |\alpha|=1$ tel que \[ \alpha\int_Xf\diff\mu=\left| \int_Xf\diff\mu \right|.\]
    Alors \begin{align*}
        \left| \int_Xf\diff\mu \right|=\int_X\Re(\alpha f)\diff\mu+i\underbrace{\int_X\Im(\alpha f)\diff\mu}_{=0}&=\int_X\Re(\alpha f)^+\diff\mu-\int_X \Re(\alpha f)^-\diff\mu\\ &\leq \int_X(\Re(\alpha f)^++\Re(\alpha f)^-)\diff\mu)\\&=\int_X|\Re(\alpha f)|\diff \mu \\
                                                                                                                 &\leq \int_X|\alpha f|\diff\mu=\int_X |f|\diff\mu
    \end{align*}
\end{proof}


\begin{thm}[Convergence dominée de Lebesgue\index{Convergence dominée de Lebesgue (théorème)}]
    Soit $(f_n:(X, \mathcal  A)\longrightarrow \C)$ une suite de fonctions mesurables qui converge simplement vers $f$. On suppose qu'il existe une fonction  $g:(X, \mathcal  A)\longrightarrow [0, +\infty]$ mesurable et telle que \begin{itemize}
        \item $|f_n|\leq g$ pour $n\geq 0$
        \item $g$ est intégrable.
    \end{itemize}
    Alors, chaque $f_n$ est intégrable et \[\int_Xf_n\diff\mu \xrightarrow[n\to+\infty]{}\int_Xf\diff\mu\]
\end{thm}

\begin{proof}
    L'intégrabilité ne pose pas de problème. Remarquons  $|f_n-f|\leq 2g$ donc $h_n=2g-|f_n-f|$ est positive. On applique le lemme de Fatou à  $(h_n)$. \[
        \int_X\underbrace{\liminf_{n\to +\infty}2g-|f_n-f|}_{=2g}\diff\mu\leq \liminf_{n\to +\infty}\int_X2g-|f_n-f|\diff\mu
    \]
    donc \[
        2\int_Xg\diff\mu\leq 2\int_Xg\diff\mu -\limsup_{n\to +\infty}\int_X |f_n-f|\diff\mu
    \] 
    d'où \[
        \limsup_{n\to +\infty} \int_X\underbrace{|f_n-f|}_{\geq 0}\diff\mu \leq 0
    \] 
    donc \[\int_X|f_n-f|\diff\mu \xrightarrow[n\to+\infty]{}0\]
    D'où finalement \[
        \left| \int_Xf_n\diff\mu-\int_Xf\diff\mu \right|\leq \int_X |f_n-f|\diff\mu \xrightarrow[n\to+\infty]{}0
    \] 
\end{proof}

\section{Presque partout}

\begin{dfn}
    On dit d'une propriété qu'elle est vraie presque partout\index{presque partout} ou pour presque tout $x$ dans  $(X, \mathcal  A, \mu)$ mesuré si la propriété est vraie pour tout $x \in  A$ avec $\mu(A^c)=0$
\end{dfn}

\begin{rem}
Pour la mesure, deux fonctions égales presque partout ont les mêmes propriétés, et on les confondra souvent. De même, on hésitera pas à considérer des fonctions définies presque partout.
\end{rem}

\begin{lmm}
    Soit $(X, \mathcal  A, \mu)$ un espace mesuré, si  $f, g$ positives ou intégrables et $E$ un ensemble mesurable,  \begin{itemize}
        \item $f=g$ p.p  $\implies \int_Ef\diff\mu=\int_Eg\diff\mu$
        \item $f\geq g$ p.p $\implies \int_Xf\diff\mu\geq \int_G g\diff\mu$
        \item si $f\geq 0$, $\int_Xf\diff\mu<+\infty \implies f \text{ finie p.p.}$
        \item si $f\geq 0$, $\int_Xf\diff\mu=0\implies f=0$ p.p.
        \item $|\int_X f\diff\mu|=\int_X|f|\diff\mu \implies \exists  \alpha \in \C, \alpha f=|f|$ p.p.
    \end{itemize}
\end{lmm}

\section{Intégrales à paramètres}

\begin{thm}[Continuité sous le signe $\int$\index{continuité sous le signe somme@continuité sous le signe $\int$}]
    Soient $(X, \mathcal  A, \mu)$ un espace mesuré et $T$ un espace métrique. Soit  $f:T\times X \longrightarrow \C$ telle que
        $\forall  t \in  T, x \longmapsto f(t, x)$ est mesurable. Soit $t_0 \in  T$, et supposons que \begin{enumerate}
            \item Pour $\mu$-presque tout  $x$,  $t\longmapsto f(t, x)$ est continue en $t_0$
            \item Il existe  $g:(X, \mathcal A, \mu)\longrightarrow [0, +\infty]$ intégrable telle que $\forall  t \in  T, $ pour $\mu-$presque tout  $x \in  X$, $|f(t, x)|\leq g(x)$
        \end{enumerate}
        Alors la fonction \[F(t)=\int_Xf(t, x)\diff \mu(x)\] est définie en tout $t \in  T$ et $F$ est continue en  $t_0$.
\end{thm}

\begin{proof}
    Par caractérisation séquentielle de la limite. Soit $t_n \longrightarrow t_0$ et posons pour tout $n\geq 1, f_n=f(t_n, \cdot)$ et $f_0=f(t_0, \cdot)$. La suite  $(f_n)$ converge  $\mu$-presque partout vers  $f_{0}$ et est dominée par $g$ donc par le théorème de convergence dominée de Lebesgue,  \[
        F(t_n)=\int_Xf_n(x)\diff\mu(x) \xrightarrow[n\to+\infty]{}\int_X f_{0}(x)\diff \mu(x)=\int_X f(t_0, x)\diff \mu(x)=F(t_0)
    \] 
\end{proof}

\begin{thm}[Dérivation sous le signe $\int$\index{dérivation sous le signe somme@dérivation sous le signe $\int$}]
    On note $(X, \mathcal  A, \mu)$ un espace mesuré, $I$ un intervalle ouvert de  $\R$. Soit $f:I\times X \longrightarrow \R$ une fonction mesurable telle que $\forall  t \in I, x \longmapsto f(t, x)$ est mesurable. Supposons qu'il existe un ensemble $A \in  \mathcal  A$ avec $\mu(A^c)=0$ tel que  \begin{enumerate}[label=(\alph*)]
        \item Il existe $t_0 \in  I$ tel que $x\longmapsto f(t_0, x)$ est intégrable
        \item Pour tout $x \in  A$, $t \longmapsto f(t, x)$ est dérivable sur $I$.
        \item Il existe  $g:(X, \mathcal  A,  \mu)\longrightarrow [0, +\infty]$ intégrable telle que \[
                \forall  x \in  A, \forall  t \in  I, \left| \frac{\partial f}{\partial t}(t, x) \right|\leq g(x)
        \] 
        Alors la fonction \[
            F(t)=\int_Xf(t, x)\diff\mu(x)
        \] 
        est bien définie sur $I$ et dérivable sur  $I$ avec  \[
            \forall  t \in  I, F'(t)=\int_X \frac{\partial f}{\partial t}(t, x)\diff\mu(x)
        \] 
    \end{enumerate}
\end{thm}

\begin{proof}
Si $t \in  I$ alors \[
    \frac{\partial f}{\partial t}(t, x)\1_A(x)=\lim_{n\to +\infty} n(f(t+\sfrac1n, x)-f(t, x))\1_A(x)
\] 
donc $\sfrac{\partial f}{\partial x}(t, x)\1_A(x)$ est mesurable et même par (c) intégrable. Puis si $x \in  A$ et $(t, t') \in  I^2 $, par le TAF on a \[
    \left| \frac{f(t, x)-f(t', x)}{t-t'} \right|= \left| \frac{\partial f}{\partial t}(c, x) \right|\leq g(x)
\]
donc si $t'=t_0$ alors \[
    |f(t, x)|\leq \underbrace{|f(t_0, x)|+|t-t_0|g(x)}_{\text{intégrable}}
\] 
donc $F$ est bien définie. Soient  $t_0 \in  I$ et $t_n \xrightarrow[n\to+\infty]{}t_0$ avec $t_n \neq  t_0$. \[
    \frac{F(t_n)-F(t_0)}{t_n-t_0}=\int_X\underbrace{\frac{f(t_n, x)-f(t_0,x)}{t_n-t_0}}_{\xrightarrow[n\to+\infty]{}\frac{\partial f}{\partial t}(t_0, x)}\diff\mu(x)
\]
pour tout $x \in  A$ et par (c), on déduit alors par convergence dominée de Lebesgue que $F(t_0)$ existe et a bien la bonne valeur.
\end{proof}

\begin{rem}
Dans le premier résultat, le "presque partout" dépend du $t_0$ où l'on veut montrer la continuité, mais pas dans le second.
\end{rem}

\section{Lien avec l'intégrale de Riemann pour les fonctions continues par morceaux}

Si $f:[a, b] \longrightarrow \C$ est continue (par morceaux), alors cette fonction est mesurable pour $(\R, \mathcal  B(\R))$ et bornée, donc intégrable au sens de Lebesgue. En particulier, elle est intégrable pour la mesure de Lebesgue, de sorte que \[
\int_{[a, b]}|f|\diff\lambda <\infty
\]
et $f \in  \mathcal  L^1(\R, \mathcal  B(\R), \lambda)$. Comme $f$ est limite uniforme sur  $[a, b]$ de fonctions  $f_n$ en escalier (donc simples), on a  \[
    \int_{[a, b]}f_n\diff\lambda \xrightarrow[n\to+\infty]{} \int_{[a, b]}f\diff\lambda
\] 
par convergence dominée. On note \[
    \int_a^bf\diff\lambda \defeq\sgn(b-a)\int_{[a, b]}f\diff\lambda
\] 
et cette intégrale coïncide avec l'intégrale de Riemann de $f$ (par l'approximation vue).

\begin{prop}[Théorème fondamental de l'analyse]
Si $f:I \longrightarrow  \C$ est continue sur l'intervalle ouvert $I$ et  $x_o \in  I$ alors \[
\begin{array}{rrcl}
    F:& I & \longrightarrow & \C \\
      & x & \longmapsto & \displaystyle \int_{x_0}^x f(t)\diff t
\end{array}
\] 
est continue, dérivable de dérivée $f$.
\end{prop}

\begin{proof}
On fixe $x \in  I$, $h$ tel que  $x+h \in  I$ et \[
    \frac{F(x+h)-F(x)}{h}-f(x)=\frac{1}{h}\int_{x}^{x+h}(f(t)-f(x))\diff t
\] 
dont le module est majoré par \[
    \frac{1}{h} \int_x^{x+h}|f(t)-f(x)|\diff t
\]
Soit $\epsilon>0$ et $\delta$ le module de continuité associé pour  $f$. Alors pour  $|h|<\delta$ le majorant est borné par  $\epsilon$.
\end{proof}

\begin{cor}
    Si $g$ est de classe  $\mathcal  C^1$ sur l'intervalle ouvert $I$ et  $a<b$ sont dans  $I$ alors  \[
        g(b)-g(a)=\int_a^b g'(t)\diff t
    \]
\end{cor}

\begin{proof}
On prend $f=g'$ dans la proposition précédente de sorte que $F$ et  $g$ sont deux primitives de  $g'$ donc  $F=g+K$.
\end{proof}

\begin{rem}[Intégrale impropre]
    Si $f:]a, b[ \longrightarrow \C$ est mesurable (par exemple continue) intégrable, on a que $\int_{]a, b[}f\diff\lambda$ est bien définie et vaut \[
        \lim_{\substack{\alpha_n\to a,\beta_n\to b}}\int_{[\alpha_n, \beta_n]}f\diff\lambda
    \]
\end{rem}

On voit que la notion coïncide avec les intégrales impropres au sens de Riemann.

 \begin{rem}
Attention à distinguer cette notion de la notion d'intégrale semi-convergente comme \[
    \int_0^{+\infty} \frac{\sin x}{x}\diff x
\] 
qui n'est pas une intégrale de Lebesgue\footnotemark
\end{rem}

\footnotetext{Pour définir l'intégrale de Lebesgue, on a supposé la convergence des parties positives et négatives}
