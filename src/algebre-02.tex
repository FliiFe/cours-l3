\ifsolo
    ~

    \vspace{1cm}

    \begin{center}
        \textbf{\LARGE Formes linéaires et dualité} \\[1em]
    \end{center}
    \tableofcontents
\else
    \chapter{Formes linéaires et dualité}

    \minitoc
\fi
\thispagestyle{empty}

On se fixe un corps $k$. On travaille avec des espaces vectoriels sur  $k$.

\section{Formes linéaires}

 \begin{dfn}
     Soit $E$ un espace vectoriel. Une forme linéaire\index{forme linéaire} sur $E$ est un élément de $ \mathcal  L(E, k)$. On note $E^*$ l'espace dual\index{espace dual} $ \mathcal  L(E, k)$
\end{dfn}

\begin{prop}
    \begin{enumerate}
        \item Le noyau d'une forme linéaire est un hyperplan\footnotemark de $E$.
        \item Tout hyperplan de $E$ est le noyau d'une forme linéaire.
        \item Deux formes qui ont le même noyau sont colinéaires
    \end{enumerate}
\end{prop}

\footnotetext{sev de $E$ tq  $\forall  x \in  E \setminus H  $ alors $E = H \oplus kx$}

\begin{proof}
\begin{enumerate}
    \item Soit $\mu$ une forme linéaire non nulle,  $H=\ker \mu$. Soit  $x \in  E \setminus H$, alors $\mu(x)\neq 0$. Soit $y \in  E$, alors \[
            y- \frac{\mu(y)}{\mu(x)}x \in  H
    \] 
    donc $E=kx+H$
    \item Soit  $H$ un hyperplan,  $x \in  E \setminus  H$ de sorte que $E=H\oplus kx$. Tout élément de  $E$ s'écrit uniquement  $y=z+\lambda x$,  $z \in  H, \lambda \in  k$. En posant $\mu(y)=\lambda$, on a bien construit une forme linéaire de noyau  $H$. 
    \item Si $H=\ker \mu=\ker \mu'$ alors pour  $x \in  E \setminus  H, E=H\oplus kx$. Alors \[
            \mu = \frac{\mu(x)}{\mu'(x)}\mu'
    \]
\end{enumerate}
\end{proof}

Dans toute la suite, on supposera $E$ de dimension finie.

\begin{thm}
    Soit $(e_1, \cdots , e_n)$ une base de $E$. L'application  \[
    \begin{array}{rrcl}
        \phi:& E^* & \longrightarrow & k^n \\
             & \mu & \longmapsto & \displaystyle (\mu(e_1), \cdots , \mu(e_n))
    \end{array}
    \] 
    est un isomorphisme.
\end{thm}

\begin{cor}
$\dim E^*=\dim E$
\end{cor}

\begin{cor}[Base duale\index{base duale}]
    Soit $(e_1, \cdots , e_n)$ une base de $E$. La base duale de $E^*$ associée  $(e_1^*, \cdots , e_n^*)$ est définie par \[
        e_i^*(e_j)=\delta_{i,j}
    \] 
    On les appelle aussi "formes coordonnées".
\end{cor}

\begin{rem}
Si $\mu \in  E^*$, alors \[
    \mu = \sum_{i=1}^{n} \mu(e_i)e_i^*
\] 
\end{rem}

\begin{rem}[Notation]
    Si $(E_i)_{i \in  I}$ est une famille d'espaces vectoriels, alors $\prod_{i \in  I}E_i$ est l'espace produit (l'ensemble des $(x_i)_{i \in  I}$). Puis \[
        \bigoplus_{i \in  I}E_i \subseteq \prod_{i \in  I} E_i
    \] 
    est le sous-espace formé des $(x_i)_{i \in  I}$ à support fini.
\end{rem}

% TODO: prop ?
Soient $E_1, \cdots , E_n$ des espaces vectoriels de dimension finie. Si  $\displaystyle E = \bigoplus_{i=1}^nE_i$ alors $E^*\simeq \displaystyle \bigoplus_{i=1}^n E^*$ de la façon suivante  \[
\begin{array}{rrcl}
    & E^* & \longrightarrow & \displaystyle\bigoplus_{i=1}^nE_i^* \\
    & \mu & \longmapsto & \displaystyle (\mu\left|_{E_i}\right.)_{i \in  \llbracket 1, n \rrbracket }
\end{array}
\] 

\section{Bidualité}

\begin{thm}
L'application suivante est un isomorphisme. \[
\begin{array}{rrcl}
    \Phi:& E & \longrightarrow & E^{**} \\
         & x & \longmapsto & \displaystyle (\mu \longmapsto \mu(x))
\end{array}
\] 
\end{thm}

\begin{proof}
    On vérifie l'injectivité. Si $x \in  E \setminus  \left\{ 0 \right\} $ alors, on note $H$ un supplémentaire de  $kx$ de sorte que  $H$ est un hyperplan donc c'est le noyau d'une forme linéaire  $\mu$, et  $\Phi(x)(\mu)\neq 0$ donc  $\Phi(x) \neq 0$ et $\ker \Phi = \left\{ 0 \right\} $. L'égalité des dimensions conclut.
\end{proof}

\begin{prop}
    Soit $\mathcal  B = (e_1, \cdots , e_n)$ une base de $E$. On note  $\mathcal  B^*$ la base duale associée, et $\mathcal  B^{**}$ la base duale associée à $\mathcal  B^{*}$. Alors \[
        \Phi(e_i^*)=e_i^{**}
    \] 
\end{prop}

\begin{proof}
La base duale $\mathcal  B^{**}$ est caractérisée par \[
    e_i^{**}(e_i^*)=\delta_{i,j}
\] 
Par ailleurs, \[
    e_j^*(e_i)=\delta_{i,j}=\Phi(e_i)(e_j^*)
\] 
donc $\Phi(e_i)=e_i^{**}$
\end{proof}

\begin{cor}
Soit $\mathcal  B'$ une base de $E^{**}$. Alors il existe $ \mathcal  B$ une base de $E$ dont la base duale est  $\mathcal  B'$.
\end{cor}

\section{Transposition}

\begin{dfn}
    Soit $u \in  \mathcal  L(E, F)$. On définit $\transpose u \in  \mathcal  L(F^*, E^*)$ par \[
        \forall  \mu \in  F^*, \quad  \transpose u(\mu)=\mu\circ u
    \]
    On l'appelle transposée\index{transposée} de $u$
\end{dfn}

\begin{rem}
    La transposition renverse l'ordre des compositions. \[
        E \xrightarrow{u}F \xrightarrow{v}G
    \]
    On a $\transpose(v\circ u)=\transpose u\circ \transpose v$
\end{rem}

\begin{prop}
    Soit $\mathcal  B_E=(e_1, \cdots , e_n)$ une base de $E$,  $\mathcal  B_F=(f_1, \cdots , f_m)$ une base de $F$. Si  \[
        A=\mathcal  M_{\mathcal B_E, \mathcal  B_F}(u)
    \] 
    alors \[
        \transpose A = \mathcal  M_{\mathcal  B_F^*, \mathcal  B_E^*}(\transpose u)
    \] 
\end{prop}

\begin{proof}
    On suppose $A=(a_{i,j})_{\substack{1\leq i \leq m \\ 1\leq j\leq m}}$, c'est à dire \[
        \mu(e_j)= \sum_{i=1}^{m} a_{i,j}f_i
    \] 
    Alors \begin{align*}
        \transpose u(f_j^*)&=f_j^*\circ u \\
                           &= \sum_{i=1}^{n} (f_j^*\circ u)(e_i)e_i^* \\
                           &= \sum_{i=1}^{n} f_j^*(u(e_i))e_i^* \\
                           &= \sum_{i=1}^{n} f_j^*\left( \sum_{\ell=0}^{m} a_{\ell,j}f_{\ell}\right)e_i^* \\
                           &= \sum_{i=1}^{n} a_{j,i}e_i^*
    \end{align*}
\end{proof}

\begin{cor}
La transposition est linéaire et bijective.
\end{cor}

\begin{cor}
    $u \in  \mathcal  L(E, F)$, $\transpose \transpose u \in  \mathcal  L(E^{**}, F^{**})\simeq \mathcal  L(E, F)$ canoniquement, et pour cette identification, $\transpose \transpose u=u$
\end{cor}

\section{Orthogonalité}

\begin{dfn}
    Soit $X \subset E$. On note\index{espace orthogonal} \[
    X^\bot = \left\{ \mu \in  E^* \suchthat \mu(X)= \left\{ 0 \right\}  \right\} 
\] 
\end{dfn}

\begin{rem}
\begin{itemize}
    \item $X \subset Y \subset E \implies X^\bot \subset Y^\bot \subset E^*$
    \item $X^\bot$ est un sev de  $E^*$
    \item  $X^\bot =\Vect(X)^\bot$
\end{itemize}
\end{rem}

\begin{prop}
    Soient $F, G$ deux sev de  $E$. Alors  $(F+G)^\bot =F^\bot \cap G^\bot$
\end{prop}

\begin{proof}
    Si $\forall  x \in  F\cup G, \mu(x)=0$ alors $\forall  x \in  F+G, \mu(x)=0$ donc $F^\bot \cap G^\bot \subset (F+G)^\bot$.

    Puis  $(F+G)\bot \subset F^\bot$ et  $\subset G^\bot$ d'où l'autre inclusion.
\end{proof}

\begin{prop}
    Soit $F$ un sev de  $E$. Alors  $F^\bot$ s'identifie canoniquement à  $(E / F)^*$
\end{prop}

\begin{proof}
    $\pi:E \longrightarrow E / F$ la projection canonique. Alors $\transpose \pi:(E / F)^* \longrightarrow  E^*$ est injective car si $\mu \in  (E / F)^* \setminus \left\{ 0 \right\} $ alors $\mu \circ \pi\neq 0$ (surjectivité de $\pi$) donc $\transpose \pi (\mu)\neq  0$. L'application $\transpose \pi$ identifie donc  $(E / F)^*$ à un sev de  $E^*$.

    Puis, $\mu \in  \im(\transpose \pi)$ se factorise en $\mu=\mu'\circ \pi$ donc $F=\ker \pi \subset \ker \mu$ donc $\im(\transpose \pi)\subset F^\bot$. Même principe pour l'autre inclusion.
\end{proof}

\begin{cor}
    $\dim F^\bot=\dim (E / F)^* = \dim E / F = \dim E-  \dim F$ d'où  \[
    \dim F+\dim F\bot = \dim E
    \] 
\end{cor}

\begin{dfn}
Soit $G\subset E^*$ un sev. Alors  on note\[
    G^\top = \left\{ x \in  E, \mu(x)=0, \forall  \mu \in  G \right\} 
\] 
\end{dfn}

\begin{prop}
$G^\bot\subset E ^{**}$ s'identifie à $G^\top \subset E$ par  $\Phi_E$
\end{prop}

\begin{cor}
$\dim G^\top+\dim G=\dim E$
\end{cor}

\begin{prop}
Soit $F\subset E$ un sev. Alors  $F^{\bot\top}=F$
\end{prop}

\begin{proof}
    Clairement $F \subset F ^{\bot \top}$. Par ailleurs, $\dim F^{\bot\top}+\dim F^\top=\dim E^*$ et $\dim F+\dim F^\top=\dim E$ donc $\dim F=\dim F^{\bot\top}$
\end{proof}

\begin{prop}
    Soient $F, G$ deux sev de  $E$. Alors  $(F\cap G)^\bot=F^\bot+G^\bot$
\end{prop}

\begin{proof}
    $(F^\bot+G^\bot)^\top=F^{\bot\top}\cap G^{\bot\top}=F\cap G$ donc $(F\bot+G^\bot)=(F^\bot+G^\bot)^{\top\bot}=(F\cap G)^\bot$
\end{proof}

\begin{thm}
L'application $F \longrightarrow F^\bot$ établit une bijection décroissante entre les sev de $E$ et ceux de $E^*$. Sa réciproque est  $G\longmapsto G^\top$. Les sev de dimension $d$ correspondent aux sev de dimension  $\dim E-d$
\end{thm}

\section{Orthogonalité et transposition}

\begin{thm}
    Pour $u \in  \mathcal  L(E, F)$, \begin{itemize}
        \item $(\ker u)^\bot=\im(\transpose u)$
        \item  $(\im u)^\bot = \ker (\transpose u)$
    \end{itemize}
\end{thm}

\begin{proof}
    On note $(f_1, \cdots , f_m)$ une base de $F$,  $(f_1^*, \cdots , f_m^*)$ la base duale associée. \[
        \mu_i=\transpose u(f_i^*) \in  E^*
    \] 
    Si $x \in E$ alors \[
        u(x)=\sum_{i=1}^{m} \mu_i(x)f_i
    \] 
    donc $\ker u=\cap \ker \mu_i=\cap (k\mu_i)^\top=\Vect(\mu_1, \cdots , \mu_m)^\top=\im(\transpose u)^\top$ 

    Pour l'autre résultat, on applique le premier à $\transpose u$
\end{proof}

\begin{cor}
$\rg u=\rg \transpose u$
\end{cor}

\begin{proof}
\begin{align*}
    \dim \im u &= \dim E - \dim \im(u)^\bot \\&= \dim E - \dim (\ker \transpose u)\\ &= \dim E - (\dim E - \rg \transpose u)=\rg\transpose u
\end{align*}
\end{proof}

\section{Application concrète}

Soit $F\subset k^n$ un sev donné par une famille génératrice  $(f_1, \cdots , f_m)$. On veut décrire $F$ par des équations\footnote{cela revient à donner une famille génératrice de l'orthogonal}.

On note $A$ la matrice ayant pour colonnes  $(f_1, \cdots , f_m)$. C'est la matrice de  \[
\begin{array}{rrcl}
    u:& k^m  & \longrightarrow & k^n \\
    & e_i & \longmapsto & \displaystyle f_i
\end{array}
\] 
$F = \im u=\ker(\transpose u)^\bot$ donc  $F^\bot-\ker\transpose u$. Il reste à calculer le noyau (pivot de gauss)
