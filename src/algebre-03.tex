\ifsolo
    ~

    \vspace{1cm}

    \begin{center}
        \textbf{\LARGE Produit tensoriel} \\[1em]
    \end{center}
    \tableofcontents
\else
    \chapter{Produit tensoriel}

    \minitoc
\fi
\thispagestyle{empty}

Tous les espaces vectoriels seront de dimension finie sur un corps $k$ fixé

\section{Applications multilinéaires}

\begin{dfn}
    Soient $E_1, \cdots , E_n$ des espaces vectoriels. Soit $F$ un espace vectoriel. Une application\index{application multilinéaire}  $f:E_1\times \cdots \times E_n \longrightarrow F$ est dite $n$-linéaire si elle est linéaire par rapport à chacune des variables: \[
    \forall  i \in  \llbracket 1, n \rrbracket , \forall  (x_j \in E_j)_{j \in  \llbracket 1, n \rrbracket \setminus \left\{ i \right\} }, x_i \longmapsto f(x_1, \cdots , x_n) \text{ linéaire }
\] 
Une application $2$-linéaire sera dite bilinéaire, et si $F=k$, on appellera $f$ forme  $n$-linéaire
\end{dfn}

\begin{ex}
\begin{itemize}
    \item Si $A$ est une  $k$-algèbre alors $(a, b) \in  A^2 \longmapsto ab \in  A$ est bilinéaire.
    \item Si $E$ est un espace vectoriel alors l'application $(u, v) \in  \mathrm{End}(E)^2 \longmapsto  v\circ u$ est bilinéaire.
    \item Si $E$ est un espace vectoriel de dimension $n$ et  $\mathcal  B$ est une base de $E$ alors $(x_1, \cdots , x_n) \in  E^n \longmapsto \det_{\mathcal  B}(x_1, \cdots , x_n)$ est une forme $n$-linéaire
    \item Si  $E$ est un espace vectoriel,  $(x, \lambda) \in  E\times E^\star \longmapsto \lambda(x) \in  k$ est une forme bilinéaire appelée \textbf{crochet de dualité}\index{crochet de dualité}
    \item Le produit scalaire euclidien sur $\R^n$ est une forme bilinéaire
\end{itemize}
\end{ex}

\begin{dfn}[Notation]
    L'ensemble des applications $n$-linéaires $E_1\times \cdots \times E_n \longrightarrow F$ est noté $n-\mathrm{Lin}(E_1, \cdots , E_n, F)$. Pour $n=2$, on note  $\mathrm{Bil}(E_1, E_2, F)$. Ce sont des espaces vectoriels.
\end{dfn}

\begin{prop}
    Pour $i \in  \llbracket 1, n \rrbracket $, soit $(e_j^{(i)})_{j \in  J_i}$ une base de $E_i$. Alors l'espace vectoriel $n-\mathrm{Lin}(E_1, \cdots , E_n, F)$ est isomorphe à $F^{J_1\times\cdots \times J_n}$. En particulier, cet espace est de dimension $(\dim F)\times \prod \dim E_i$
\end{prop}

\begin{proof}
On définit \[
\begin{array}{rrcl}
    \phi:& n-\mathrm{Lin}(E_1, \cdots , E_n, F) & \longrightarrow & F^{J_1\times\cdots \times J_n} \\
         & f & \longmapsto & \displaystyle (f(e_{j_1}^{(1)}, \cdots , e_{j_n}^{(n)}))_{j_1, \cdots , j_n \in  J_1\times \cdots \times J_n}
\end{array}
\]
Cette fonction est clairement injective ($f$ est déterminée par sa valeur sur une base). Puis, elle est surjective: on se donne  $a_{(j_1, \cdots , j_n)}$ pour chaque $(j_1, \cdots , j_n) \in  J_1\times \cdots \times J_n$. Soit $x \in  E_1\times \cdots \times E_n$. On pose \[
    x_i=\sum_{j \in  J_i}\lambda_j^{(i)}e_j^{(i)}, \quad  \lambda_j^{(i)} \in  k
\] 
Puis on définit \[
    f(x_1, \cdots , x_n)=\sum_{(j_1, \cdots , j_n)} \lambda_{j_1}^{(1)} \cdots \lambda_{j_n}^{(n)} a_{(j_1, \cdots , j_n)} \in  F
\] 
\end{proof}

\section{Définition du produit tensoriel}

Soient $E$ et  $F$ des espaces vectoriels 

\begin{thm}
Il existe un espace vectoriel noté $E \tens F$ et une application bilinéaire  $b:E\times F \longrightarrow E\tens F$ qui vérifie la propriété universelle suivante: pour tout espace vectoriel $G$ et toute application bilinéaire  $\phi:E\times F\longrightarrow G$, il existe une unique application linéaire $f:E\tens F\longrightarrow G$ telle que $\phi=f\circ b$.

 \begin{center}
\begin{tikzcd}[sep=large,every label/.append style={font=\normalsize}]
    E\times F \arrow[r, "\phi"] \arrow[dr,"b"] & G \\
                                                & E\tens F \arrow[u, "f", dashed]
\end{tikzcd}
\end{center}
\end{thm}

\begin{proof}~
\begin{itemize}
    \item Existence: Soit $(e_1, \cdots , e_m)$ une base de $E$ et  $(f_1, \cdots , f_n)$ une base de $F$. On définit  $E\tens F$ comme l'espace vectoriel de base indexée par  $\llbracket 1,m \rrbracket \times \llbracket 1,n \rrbracket $. On note $e_i\oplus f_j$ l'élément de la base correspondant au couple  $(i, j)$ 
    
        Soient $(x, y) \in  E\times F$, et $(a_i)_i, (b_i)_i$ les coordonnées dans les bases choisies. On pose \[
            b(x, y)= \sum_{(i, j)} a_ib_j(e_i\tens f_j) \in  E\tens F
        \] 
        C'est une application bilinéaire. Montrons la propriété universelle. Soit $\phi \in  \mathrm{Bil}(E, F, G)$. On prend $f$ l'unique application linéaire envoyant  $e_i\tens f_j$ sur  $\phi(e_i, f_j)$. Montrons que  $\phi=f\circ b$. Les deux applications  $\phi$ et  $f\circ b$ sont bilinéaires. Par la proposition ci-dessus, il suffit de montrer qu'elles coïncident sur  les $(e_i, f_j)$. On a  \[
            \phi(e_i, f_j)=f(e_i\tens f_j)=f\circ b(e_i, f_j)
        \] 
    \item Unicité: Soit $(T, b')$ un autre couple vérifiant  (PU). Alors en appliquant (PU) sur $E\tens F$ et avec la symétrie, on obtient  $g:T\longrightarrow E\tens F$. On va montrer que $f$ et  $g$ sont inverses l'une de l'autre.
 \begin{center}
\begin{tikzcd}[sep=large,every label/.append style={font=\normalsize}]
    E\times F \arrow[r, "b'"] \arrow[dr,"b"] & T\arrow[d,"g",shift left] \\
                                                & E\tens F \arrow[u,"f",shift left] % TODO: fix.
\end{tikzcd}
\end{center}
On a $g\circ f: E\tens F \longrightarrow E\tens F$ et l'unicité dans la (PU) pour $(E\tens F, b)$ donne 
 \begin{center}
\begin{tikzcd}[sep=large,every label/.append style={font=\normalsize}]
    E\times F \arrow[r, "b"] \arrow[dr,"b"] & E\tens F \\
                                                & E\tens F \arrow[u, "\id"]
\end{tikzcd}
\end{center}
et $g\circ f$ est aussi une factorisation
 \begin{center}
\begin{tikzcd}[sep=large,every label/.append style={font=\normalsize}]
    E\times F \arrow[r, "b"] \arrow[dr,"b"] & E\tens F \\
                                                & E\tens F \arrow[u, "g\circ f"]
\end{tikzcd}
\end{center}
Donc $g\circ f=\id_{E\tens F}$ et par symétrie $f\circ g=\id_T$
\end{itemize}
\end{proof}

\begin{prop}
    On a $\dim(E\tens F)=\dim(E)\times \dim(F)$
\end{prop}

\begin{proof}
Par construction
\end{proof}

\begin{rem}
    Si on applique (PU) à $G=k$, on obtient un isomorphisme canonique  $\mathrm {Bil}(E, F, k) \cong (E\tens F)^\star $. Plus généralement, $\mathrm {Bil}(E, F, G)\cong \mathrm {Hom}(E\tens F, G)$
\end{rem}

\section{Tenseurs}

\begin{dfn}
    Pour $(x, y) \in  E\times F$. On pose \[
        x\tens y=b(x, y) \in  E\tens F
    \] 
    Les éléments $x\tens y$ sont appelés tenseurs simples\index{tenseurs simples}.
\end{dfn}

\begin{rem}
En général, il existe des éléments de $E\tens F$ qui ne sont pas des tenseurs simples.
\end{rem}

\begin{exo}
Si $\dim F=\dim E=2$ alors l'élément  $e_1\tens f_1+e_2\tens f_2$ n'est pas un tenseur simple.
\end{exo}

\begin{prop}
    Si $(e_1, \cdots , e_m)$ est une base de $E$,  $(f_1, \cdots , f_n)$ une base de $F$,  et $(x, y) \in  E\times F$ de coefficients respectifs $(a_i)_i, (b_j)_j$. Alors  \[
        x\tens y= \sum_{i=1}^{m} \sum_{j=1}^{n} a_ib_j(e_i\tens f_j)
    \] 
\end{prop}

\begin{proof}
    C'est la bilinéarité de $(x, y)\longmapsto x\tens y$
\end{proof}

\begin{prop}
Tout élément de $E\tens F$ est somme de tenseurs simples.
\end{prop}

\begin{proof}
    On décompose sur la base des $(e_i\tens f_j)$ et  $\alpha (x\tens y)=(\alpha x)\tens y$.
\end{proof}

\begin{rem}
L'écriture d'un tenseur comme somme de tenseurs simples n'est pas unique.
\end{rem}

\begin{prop}
    Soit $A=(e_1, \cdots , e_p)$ une famille de $E$ et  $B=(f_1, \cdots , f_q)$ une famille de $F$. Posons  $C=(e_i\tens f_j)_{\substack{1\leq i\leq p\\1\leq j\leq q}}$. Si $A$ et  $B$ sont libres (resp. génératrices, resp. des bases), alors  $C$ est libre (resp. génératrice, resp. une base).
\end{prop}

\begin{proof}
    Supposons que $A$ et  $B$ sont génératrices. Alors on sait que tout tenseur est une CL de tenseurs simples et les tenseurs simples sont dans  $\Vect(C)$.

    Supposons que $A$ et  $B$ sont des bases. Alors  $C$ est génératrice est  $\#C=\#A\times \#B$ donc  $C$ est une base.

    Si  $A$ et  $B$ sont libres, alors on complète en bases  $A'$ et  $B'$ et  $C\subset C'$ avec  $C'$ une base.
\end{proof}

\section{Quelques isomorphismes}

\begin{prop}
Soient $E, F$ des espaces vectoriels. Il y a un isomorphisme canonique \[
    E^\star \tens F\cong \Hom(E, F)
\] 
\end{prop}

\begin{proof}
    On va utiliser la (PU) pour une application bilinéaire bien choisie. On pose \[
    \begin{array}{rrcl}
        \phi:& E^* \times F& \longrightarrow & \Hom(E, F) \\
             & (\lambda, y) & \longmapsto & \displaystyle (x \longmapsto \lambda(x)y)
    \end{array}
    \] 
    Donc il existe une unique application linéaire $\psi : E^\star \tens F \longrightarrow \Hom(E, F)$ telle que $\psi(\lambda\tens y)=\phi(\lambda, y)$ pour tout  $\lambda \in  E^\star, y \in  F$. Puis, \[
        \dim(E^\star\tens F)=\dim (E)\times \dim (F)=\dim\Hom(E, F)
    \] 
    Il va suffire de montrer la surjectivité. Soit $(f_1, \cdots , f_n)$ une base de $F$ et  $g \in \Hom(E, F)$. Posons $g_i=f_i^\star\circ g$ où $(f_i^\star)$ est la base duale de  $(f_1, \cdots , f_n)$. Les $g_i$ sont des formes linéaires et \[
        g=\sum_{i=1}^n \phi(g_i, f_i)=\sum_{i=1}^n\psi(g_i\tens f_i)=\psi\left(\sum_{i=1}^n g_i\oplus f_i\right)
    \] 
\end{proof}

\begin{prop}
Pour tout espace vectoriel $E$, il y a un isomorphisme canonique  \[
E\tens k\cong E
\]
\end{prop}

\begin{proof}
    On définit $\varphi:E \longrightarrow E\tens k, x\longmapsto x\tens 1$. Dans l'autre sens, on considère $h:E\times k \longrightarrow E, (x, a)\longmapsto ax$, qu'on factorise avec (PU) qui donne $\psi:E\tens k \longrightarrow E, x\tens a \longmapsto ax$. Montrons que $\psi\circ \varphi=\id$ et  $\varphi\circ \psi=\id$. Soit  $x \in  E$. Alors \[
        \psi\circ \varphi(x)=\psi(x\tens 1)=1.x=x
    \]
    Soit $t \in  E\tens k$. Alors  c'est une somme de tenseurs simples. La linéarité permet de se contenter d'une vérification sur les tenseurs simples. \[
        \varphi\circ \psi(x\tens a)=\varphi(ax)=ax\tens a=a(x\tens 1)=x\tens a
    \]
\end{proof}

\begin{prop}
Pour tous espaces vectoriels $E_1, E_2, F$, on a un isomorphisme canonique \[
    (E_1\oplus E_2)\tens F\cong (E_1\tens F)\oplus(E_2\tens F)
\] 
\end{prop}

\begin{prop}
 \[
E\tens F\cong F\tens E
\] 
\end{prop}

\begin{prop}
\[
    (E\tens F)^\star \cong E^\star \tens F^\star
\] 
\end{prop}

\begin{prop}
    \[
        (E\tens F)\tens G \cong E\tens(F\tens G)
    \] 
\end{prop}
