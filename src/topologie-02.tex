\ifsolo
    ~

    \vspace{1cm}

    \begin{center}
        \textbf{\LARGE Espaces compacts} \\[1em]
    \end{center}
    \tableofcontents
\else
    \chapter{Espaces compacts}

    \minitoc
\fi
\thispagestyle{empty}

% TODO: gras pour les déf

\section{Compacité}

\begin{dfn}
    On dit qu'un espace métrique $(E, d)$ est compact\index{compact} si toute suite  $\left\{ x_n \right\} _{n\geq 1}$ admet une valeur d'adhérence. On dit que $(E, d)$ est précompact\index{précompact} si pour tout  $r>0$,  $E$ est l'union d'un nombre fini de boules de rayon  $r$.
\end{dfn}

\begin{thm}
    Un espace métrique $(E, d)$ est compact  si et seulement s'il est complet et précompact.
\end{thm}

\begin{proof}
    Supposons que toute suite de $E$ admet une valeur d'adhérence.  i $ \left\{ x_n \right\}_{n\geq 1} $ est de Cauchy, et $x$ en est une valeur d'adhérence, alors $x$ en est la limite. $E$ est donc complet. Soit $r>0$ et $x_1 \in  E$. On construit par récurrenc $x _{k+1} \in  E \setminus \B(x_1,r)\cup \cdots \cup \B(x_k, r)$. Si ce n'est plus possible à l'étape $k$, c'est qu'on a recouvert $E$ avec des boules de rayon $r$. Sinon, $ \left\{ x_n \right\}_{n\geq 1} $ n'admet par de valeur d'adhérence car la différence entre deux termes est toujours $ \geq r$, ce qui est absurde.

    Montrons la réciproque. Soit $ \left\{ x_n \right\}_{n\geq 1} $ une suite de $E$.  'espace $E$ est l'union d'un nombre fini de boules de rayon $1$ et l'une d'elle, notée $ \B_1$, contient donc une infinité de termes de la suite. De même, $E$ est l'union d'un nombre fini de boules de rayon $\sfrac12$ de sorte que l'une d'entre elle, notée  $\B_2$ est telle que  $\B_1\cap \B_2$ contient une infinité de termes. En itérant, on construit une sous-suite de Cauchy de $\left\{ x_n \right\} $, et comme $E$ est complet, cette sous-suite converge vers une valeur d'adhérence de la suite  $\left\{ x_n \right\} $.
\end{proof}

\begin{rem}
    On en déduit facilement que les compacts de $\R^n$ sont les fermés bornés. En général, les parties compactes d'un espace complet sont les fermés bornés qui satisfont en plus une condition d'uniformité (on le verra par exemple avec le théorème d'Ascoli)
\end{rem}

\begin{rem}
\begin{itemize}
    \item Un fermé inclus dans un compact est compact
    \item Le produit cartésien de compacts est compact
    \item Un compact est séparable
    \item L'image d'un compact par une application continue est un compact
\end{itemize}
\end{rem}

\begin{thm}[Heine\index{Heine (théorème)}]
Si $X$ est compact, alors toute fonction  $f:X \longrightarrow Y$ continue est aussi uniformément continue.
\end{thm}

\begin{proof}
    Si $\epsilon>0$, on veut montrer qu'il existe $\delta >0$ de continuité de  $f$ valable pour tout  $x$. Si ce n'est pas le cas, alors pour tout $n\geq 1$, il existe $x_n,x_n' \in  X$ tels que $d(x_n, x_n')<\sfrac1n $ mais  $d(f(x_n, x_n'))>\epsilon$. Si $x$ est une valeur d'adhérence de  $\left\{ x_n \right\} $ alors c'est aussi une valeur d'adhérence de $\left\{ x_n' \right\} $ et on trouve $d(f(x), f(x))>\epsilon$
\end{proof}

\begin{thm}[Poincaré\index{Poincaré (théorème)}]
Si $X$ est un compact et si  $f: X\longrightarrow Y$ est une bijection continue, alors $f$ est un homéomorphisme.
\end{thm}

\begin{proof}
    Il s'agit de montrer que si $F$ est un fermé de  $X$, alors  $f(F)$ est fermé dans  $Y$, ce qui est vrai car $F$ est compact donc  $f(F)$ aussi.
\end{proof}

\begin{dfn}
    Soit $E$ un  $\R-$espace vectoriel. On dit que deux normes $N_1, N_2$ sont équivalentes\index{normes équivalentes} s'il existe $C>0$ tel que  $N_1<CN_2$ et  $N_2<CN_1$.
\end{dfn}

\begin{rem}
    Si $N_1$ et  $N_2$ sont des normes équivalentes sur $E$ alors l'application identité  $(E, N_1)\longrightarrow (E, N_2)$ est un homéomorphisme, et si de plus $E$ est complet pour l'une des normes, il l'est aussi pour l'autre.
\end{rem}

\begin{thm}
Si $E$ est un  $\R$-espace vectoriel de dimension finie, alors toutes les normes sont équivalentes sur $E$.
\end{thm}

\begin{proof}
    Si $\left\{ e_1, \cdots , e_n \right\} $ est une base de $E$, on pose  \[
        \left\|\sum_{i=1}^nx_ie_i\right\|_\infty=\sup_{1\leq i\leq n} |x_i|
    \]
    On a \[
    \left\|\sum_{i=1}^nx_ie_i\right\|\leq \sum_{i=1}^n |x_i| \|e_i\|\leq C_1 \|x\|_\infty
    \] 
    avec $C_1=\sum \|e_i\|$. Puis, notons $B=\partial \B^\infty(0, 1)$. C'est un compact et $\|\;\cdot\;\|^{-1}:b\longrightarrow \R$ est continue, donc admet un maximum $C_2$ sur $B$ ce qui donne $\|x\|_\infty\leq C_2\|x\|$, donc toutes les normes sont équivalentes à la norme infinie.
\end{proof}

\begin{rem}
Les $\R$-ev de dimension finie sont complets pour toutes les normes. Si $E$ est un evn et si  $F$ est un sev de  $E$ de dimension finie alors la norme de  $E$ induit une norme sur  $F$ pour laquelle  $F$ est complet et donc  $F$ est fermé dans  $E$.
\end{rem}

\begin{lmm}
    Si $F$ est un sev de dimension finie d'un evn  $E$, et si  $y \in  E$, alors il existe $z \in  F$ tel que $\|y-z\|=d(y,F)$
\end{lmm}

\begin{proof}
    $d(y, F)\leq  \|y\|$ et comme $F$ est fermé dans  $E$,  $F\cap \B(y, \|y\|)$ est compact et $x\longmapsto  \|x-y\|$ admet un minimum $z \in F$ sur ce compact.
\end{proof}

\begin{thm}[Riesz\index{Riesz (théorème)}]
    Si $E$ est un evn, alors  $\bar{\B}(0,1)$ est compacte si et seulement si $E$ est de dimension finie.
\end{thm}

\begin{proof}
    On a déjà vu q e $B= \bar\B(0, 1)$ est compacte si $E$ est de dimension finie. Il reste à voir la réciproque. On suppose $B$ compacte. Il existe $x_1, \cdots , x_n$ dans $B$ tels que \[
        B\subset \B(x_1, \sfrac12)\cup \cdots \cup \B(x_n, \sfrac 12)
    \] 
    Soit $F=\Vect(x_1, \cdots , x_n)$ et $y \in  E$. On va montrer que $y \in F$. Par le lemme, $d(y, F)=d(y, z)$ pour un  $z \in  F$, avec $\|y-z\|>0$ si $y \not \in F$ et par construction des $x_i$, il existe  $i$ tel que  $\sfrac{y-z}{\|y-z\|} \in  \B(x_i, \sfrac12)$. On a alors \[
        \|y-\underbrace{(z+ \|y-z\|x_i)}_{\in  F}\|< \frac{\|y-z\|}2
    \]
    ce qui est absurde par minimalité de $z$. Donc  $y \in  F$ pour tout $y \in  E$ et $E=F$ est de dimension finie.
\end{proof}

\begin{dfn}
    Si $(E, d)$ est un espace métrique, alors un \textbf{recouvrement ouvert} \index{recouvrement ouvert} de  $E$ est un ensemble  $\left\{ U_i \right\}_{i \in  I} $ d'ouverts de $E$ tels que  $E$ est inclus dans l'union  des $U_i$. Un \textbf{sous-recouvrement}  de $\left\{ U_i \right\} _{i \in I}$ est un ensemble $\left\{ U_i \right\} _{i \in  J}$ avec $J\subset I$ qui recouvre toujours  $E$. On dit qu'il est fini si  $J$ est fini. On dit que $E$ a la  \textbf{propriété de l'intersection finie} si et seulement si pour toute famille de fermés $\left\{ F_i \right\} _{i \in  I}$ telle que l'intersection d'un nombre fini d'entre eux est non vide, l'intersection de tous les fermés est elle même non vide. En passant aux complémentaires, on voit que $E$ a la propriété de l'intersection finie  si et seulement si tout recouvrement ouvert de $E$ admet un sous-recouvrement fini.
\end{dfn}

\begin{thm}[Borel-Lebesgue\index{Borel-Lebesgue}]
    Si $(E, d)$ est un espace métrique, alors  $E$ est compact  si et seulement si tout recouvrement ouvert de $E$ admet un sous-recouvrement fini
\end{thm}

\begin{proof}
    Supposons que $E$ est compact, et soit  $\left\{ U_i \right\} _{i \in  I}$ un recouvrement ouvert de $E$. Montrons qu'il existe  $r>0$ (un \emph{nombre de Lebesgue} du recouvrement) ayant la propriété que pour tout $x \in  E$, il existe $ i \in  I$ tel que $\B(x, r)\subset U_i$. Si ce n'est pas le cas, alors pour tout $n$ il existe  $x_n \in  E$ tel que $\B(x_n, \sfrac1n)$ n'est contenu dans aucun des $U_i$. Si  $x$ est une valeur d'adhérence de la suite  $\left\{ x_n \right\} $ alors il existe $\epsilon>0$ et $i \in  I$ tels que $\B(x, \epsilon)\subset U_i$ et alors $\B(x_n, \sfrac1n)\subset U_i$ dès que $d(x_n, x)+\sfrac1n<\epsilon$, ce qui est absurde. Comme $E$ est compact, il est précompact et donc s'écrit comme une réunion finie de boules de rayon  $r$, chacune étant contenu dans un ouvert du recouvrement, ce qui conclut.

    Montrons maintenant que si $E$ a la propriété de l'intersection finie, alors il est compact. Soit  $\left\{ x_n \right\} _{n\geq 1}$ une suite de $E$ et  $k\geq 1$. On note $F_k$ l'adhérence de  $\left\{ x_n \right\} _{n\geq k}$. L'intersection d'un nombre fini de $F_k$ est non vide, et il en est donc de même pour  \[
    \dbigcap_{k\geq 1} F_k
    \] 
    or cet ensemble est précisément l'ensemble des valeurs d'adhérence de $\left\{ x_n \right\}_{n\geq 1} $
\end{proof}
