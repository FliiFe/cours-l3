\ifsolo
    ~

    \vspace{1cm}

    \begin{center}
        \textbf{\LARGE $\sigma$-algèbres et mesures} \\[1em]
    \end{center}
    \tableofcontents
\else
    \chapter{\texorpdfstring{$\sigma$}{sigma}-algèbres et mesures}

    \minitoc
\fi
\thispagestyle{empty}

\section{Définitions}

\begin{dfn}
    Si $X$ est un ensemble,  $\mathcal  A \subset \mathcal  P(X)$ est une algèbre\index{algèbre} sur $X$ si  \begin{itemize}
        \item $\emptyset \in  \mathcal  A$
        \item $\forall  A, B \in  \mathcal  A, A \cup B \in  \mathcal  A$
        \item $\forall  A \in  \mathcal  A, A^c \in \mathcal  A$
    \end{itemize}
\end{dfn}

\begin{rem}
C'est aussi stable par intersection finie
\end{rem}

\begin{ex}~
\begin{itemize}
    \item $\mathcal  A= \left\{ \emptyset, X \right\} $
    \item $\mathcal  A = \mathcal  P(X)$
    \item $\left\{ A \subset X, A \text{ fini ou }A^c \text{ fini } \right\} $
    \item $\left\{ A \subset X, A \text{ fini ou dénombrable ou }A^c \text{ fini ou dénombrable } \right\} $ 
    \item $B$ fixé,  $\left\{ A \supset B \text{ ou } A\cap B=\emptyset\right\} $
    \item Dans $\R$, $\left\{ \text{unions finies d'intervalles} \right\} $
\end{itemize}
\end{ex}

\begin{dfn}
    $A\subset \mathcal  P(X)$ est une $\sigma$-algèbre\index{sigma-algèbre@$\sigma$-algèbre} (ou tribu\index{tribu}) si et seulement si
    \begin{itemize}
        \item $\emptyset \in  \mathcal  A$
        \item $\displaystyle \forall  (A_i \in  \mathcal A, i \in  \N), \bigcup_{i \in  \N} A_i \in  \mathcal  A$
        \item $A^c \in  \mathcal  A, \forall  A \in  \mathcal  A$
    \end{itemize}
\end{dfn}

\begin{rem}
\begin{itemize}
    \item Les $\sigma$-algèbres sont des algèbres
    \item $X \in  A$
    \item Stable par intersection dénombrable
\end{itemize}
\end{rem}

\begin{dfn}[Espace mesurable]
    Un couple $(X, \mathcal  A)$ où $\mathcal  A$ est une $\sigma$-algèbre est appelé \textbf{espace mesurable}\index{espace mesurable}
\end{dfn}

\section{\texorpdfstring{$\sigma$}{sigma}-algèbre  engendrée}

\begin{lmm}
Une intersection quelconque de $\sigma$-algèbres est une $\sigma$-algèbre.
\end{lmm}

\begin{dfn}
    La $\sigma$-algèbre engendrée par  $\mathcal  M \subset \mathcal  P(X)$ est \[
        \sigma(\mathcal  M)=\bigcap_{\substack{\mathcal A \text{ tribu }\\ \mathcal  M \subset \mathcal  A}}\mathcal  A
    \] 
\end{dfn}

\begin{dfn}[Boréliens]
    Si $X$ a une topologie, on appelle  $\sigma$-algèbre borélienne (ou des boréliens\index{boréliens})  $B(X)$ la  $\sigma$-algèbre engendrée par les ouverts.
\end{dfn}

\begin{prop}
    Les parties suivantes de $\mathcal  P(\R)$ engendrent les boréliens de $\R$.
    \begin{itemize}
        \item Les ouverts de $\R$
        \item Les fermés de $\R$
        \item Les intervalles ouverts bornés (les ouverts sont des unions d'intervalles ouverts d'extrémités rationnelles)
        \item Les intervalles ouverts d'extrémités rationnelles
        \item Les $\{]a, b]\}$ ou  $\{[a, b[\}$ (par exemple $]a, c[=\bigcup ]a, r_n]$ avec  $r_n \in \Q \longrightarrow c$ croissante)
        \item $\{]a, +\infty[\}$
    \end{itemize}
\end{prop}

\begin{prop}
Dans $\R^n$, les ensembles de parties suivants engendrent les boréliens:
\begin{itemize}
    \item Les ouverts, les fermés
    \item Les boules ouvertes
    \item Les boules ouvertes de centre à coordonnées rationnelles et de rayon rationnel
    \item Les produits cartésiens d'intervalles ouverts bornés, fermés bornés, ...
    \item Les $\left\{ \R^{i-1}\times ]a, +\infty[ \times \R^{n-i} \right\} $
\end{itemize}
\end{prop}

\begin{exo}
    On munit $[0, +\infty]$ de la distance \[
        d(x, y)= \left| \arctan x-\arctan y \right|
    \] 
    Donner des parties qui engendrent les borélien de l'espace métrique $([0, +\infty], d)$
\end{exo}

\section{Mesures}

\begin{dfn}[Mesure\index{mesure}]
    Si $(X, \mathcal  A)$ est un espace mesurable, une mesure sur $(X, \mathcal  A)$ est une application $\mu: \mathcal  A \longrightarrow [0, +\infty]$ telle que \begin{enumerate}[label=(\emph{\roman*})]
        \item $\mu(\emptyset)=0$
        \item Si $(A_i)_{i \in  \N} \in  \mathcal  A^{\N}$ est une suite d'ensembles deux à deux disjoints, alors \[
                \mu \left(\bigcup_{i \in  \N}A_i\right)= \sum_{i \in  \N} \mu(A_i)
        \] 
    \end{enumerate}
    Cette seconde propriété s'appelle $\sigma$-additivité\index{sigma-additivité@$\sigma$-additivité}. On dit alors que $(X, \mathcal  A, \mu)$ est un espace mesuré\index{espace mesuré}
\end{dfn}

\begin{ex}
\begin{itemize}
    \item $(X, \mathcal  P(X), \mathrm{Card})$ est un espace mesuré
    \item $(X, \mathcal  P(X), \delta_a)$ où $\delta_a$ est la mesure de Dirac\index{mesure de Dirac}  \[
            \delta_a(A)=\1_A(a)
    \] 
    est un espace mesuré
\end{itemize}
\end{ex}

\begin{thm}[Théorème fondamental\index{théorème fondamental}]
    Il existe une unique mesure, appelée mesure de Lebesgue\index{mesure de Lebesgue}, sur $B(\R^n)$ telle que: \begin{itemize}
        \item Sur $\R$, $\mu([a, b])=b-a$ pour tous  $a<b$
        \item Sur $\R^n$, \[
                \mu \left( \prod_{i=1}^{n} [a_i, b_i] \right)= \prod_{i=1}^{n} (b_i-a_i)
        \] 
    \end{itemize}
\end{thm}

\begin{proof}
On admet ce théorème pour l'instant.
\end{proof}

\begin{prop}
    Soit $(X, \mathcal  A, \mu)$ un espace mesuré. Alors \begin{enumerate}[label=(\alph*)]
        \item $\mu$ est finiment additive
        \item $\mu$ est croissante
        \item $\mu$ est  $\sigma$-sous-additive
    \end{enumerate}
\end{prop}

\begin{prop}
\begin{enumerate}
    \item $A_i \in  \mathcal  A$ une suite croissante. Alors $\mu(\cup A_i)=\lim \mu(A_i)$ (cette limite existe)
    \item  $A_i \in  \mathcal  A$ une suite décroissante telle que $\mu(A_0)<+\infty$. Alors $\mu(\cap A_i)=\lim \mu(A_i)$
\end{enumerate}
\end{prop}

\begin{proof}
\begin{enumerate}
    \item $B_i=A_i\setminus A_{i-1}$. \[
            \mu \left(\bigcup_{i \in  \N}B_i\right)= \sum_{i \in  \N} \mu(B_i)=\lim_{n\to +\infty} \sum_{i=0}^{n} \mu(B_i)=\lim_{n \to  +\infty} \mu(A_n)
    \] 
    \item Même principe avec $B=A_i \setminus A_{i+1}$
\end{enumerate}
\end{proof}

\begin{dfn}
\begin{itemize}
    \item $\mu$ est finie\index{mesure finie} si  $\mu(X)<+\infty$
    \item  $\mu$ est une mesure de probabilité\index{mesure de probabilité} si  $\mu(X)=1$ 
    \item $\mu$ est $\sigma$-finie \index{sigma-finie@$\sigma$-finie} s'il existe des $X_k \in  \mathcal  A$ qui recouvrent $\mathcal  X$ et tels que $\mu(X_k)<+\infty$
\end{itemize}
\end{dfn}

\begin{rem}
La mesure de Lebesgue est $\sigma$-finie
\end{rem}

\section{Fonctions mesurables}

\begin{dfn}[Fonction mesurable]
    $(X, \mathcal  A), (Y, \mathcal  B)$ des espaces mesurables. Une fonction $f:X \longrightarrow Y$ est mesurable\index{fonction mesurable}  si et seulement si $\forall  B \in  \mathcal  B, f^{-1}(B) \in  \mathcal A$
\end{dfn}

\begin{rem}[Cas particulier important]
Lorsque $Y$ a une topologie, la tribu choisie par défaut sera la tribu des boréliens
\end{rem}

\begin{prop}
    Dans les conditions de la définition,  i $ \mathcal  B=\sigma(\mathcal  M)$ alors $f$ est mesurable si et seulement si \[
        \forall  B \in  \mathcal  M, f^{-1}(B) \in  \mathcal  A
    \] 
\end{prop}

\begin{proof}
    Sens direct immédiat. Dans l'autre sens, on pose $\mathcal C= \left\{  C \in  Y \suchthat f^{-1}(C) \in  \mathcal  A \right\} $. C'est une tribu qui contient $\mathcal  M$.
\end{proof}

\begin{cor}
    \begin{itemize}
        \item $f: (X, \mathcal  A)\longrightarrow Y$ topologique, dans ce cas mesurable équivaut à  $\forall  O \text{ ouvert }, f^{-1}(O) \in  \mathcal  A$
        \item $(f_k)_k$ mesurables,  $(X, \mathcal  A)\longrightarrow [0, +\infty]$. Les fonctions $\sup f_k$ et  $\inf f_k$ sont mesurables.
    \end{itemize}
\end{cor}

\begin{proof}
    Pour le 2, on prend $\mathcal  M = \left\{ ]a, +\infty] \right\} $. \[
        \sup f_k^{-1}(]a, +\infty])= \left\{  x \suchthat f_k(x)>a \right\} = \bigcup_k f_k^{-1}(]a, +\infty])
    \] 
\end{proof}

\begin{prop}
    On se donne $f:(X, \mathcal  A) \longrightarrow (Y, \mathcal B)$ et $g:(Y, \mathcal  B)\longrightarrow (Z, \mathcal  C)$ mesurables, alors $g\circ f$ est mesurable.
    Si  $Y$ a une topologie, $Z$ aussi et $g$ continue alors  $g\circ f $ est mesurable
\end{prop}

\begin{cor}
    Si $f:(X, \mathcal  A)\longrightarrow \C$ est mesurable, alors $|f|, \Re f, \Im f$ aussi.
\end{cor}

\begin{prop}
    Si $X$ et  $Y$ sont deux espaces topologiques à bases dénombrables d'ouverts et  $f:(X, \mathcal  A)\longrightarrow Y$, $g:(X, \mathcal  A)\longrightarrow Z$ mesurables, alors $(f, g):(X, \mathcal  A)\longrightarrow Y\times Z$ est mesurable.
\end{prop}

\begin{cor}
    Si $f,g:(X, \mathcal  A)\longrightarrow \R\text{ ou }\C$ sont mesurables alors $f+g, f\times g, f-g, \frac{f}{g}$ (si possible) sont mesurables
\end{cor}

\section{Limites de fonctions mesurables}

\begin{thm}
    Si $(f_k)_k$ mesurables $(X, \mathcal A)\longrightarrow Y$ qui convergent simplement vers $f$, avec $Y$ un espace \textbf{métrique}, alors  $f$ est mesurable.
\end{thm}

\begin{proof}
    \todo{à récupérer sur le portail des études}
\end{proof}
