\usepackage{amsmath}
\usepackage{bbm}
\usepackage{bm}
\usepackage{gauss}
\renewcommand\colswapfromlabel[1]{#1}
\renewcommand\colswaptolabel[1]{#1}
\renewcommand\rowswapfromlabel[1]{#1}
\renewcommand\rowswaptolabel[1]{#1}
\usepackage[cal=rsfso]{mathalpha}
\usepackage{mathtools}
\usepackage{esvect}
\usepackage{array}
\usepackage{stackengine}
\stackMath
\usepackage{centernot}
\usepackage{stmaryrd}
\usepackage{amssymb}
\usepackage{xfrac}
\usepackage{tikz-cd} 
% strikeout as \sout
\usepackage[normalem]{ulem}

% opérateurs usuels
\NewDocumentCommand\genmrm{O{#2}m}{%
    \expandafter\DeclareDocumentCommand\csname#1\endcsname{}{%
        \mathop{\mathrm{#2}}\nolimits%
    }%
}
\genmrm{det}
\genmrm{ker}
\genmrm{Ker}
\genmrm{dom}
\genmrm{Vect}
\genmrm{Sp}
\genmrm[Img]{Im}
\genmrm{id}
\genmrm{im}
\genmrm{rg}
\genmrm{ord}
\genmrm{Tr}
\genmrm{Com}
\genmrm{Cov}
\genmrm{Conv}
\genmrm{diag}
\genmrm{Hom}
\genmrm{sh}
\genmrm{ch}
\genmrm{Aut}

% continu par morceaux
\NewDocumentCommand\CPM{O{0}}{\mathcal{C}^{#1}_{{P\!M}}}

% d droit pour les différentielles (intégrale, calcul diff)
\newcommand\diff{\mathop{}\!\mathrm d}

% norme d'opérateur
\newcommand\nop{\|\hspace{-1pt}|}

% :=
\newcommand\defeq{\;\underset{\text{def}}=\;}

% fonction indicatrice
\newcommand\1{\mathbbm{1}}

\renewcommand\Im{\mathfrak{Im}}
\renewcommand\Re{\mathfrak{Re}}
% intérieur topologique
\renewcommand\mathring[1]{\overset\circ{#1}}

\renewcommand\epsilon\varepsilon
\renewcommand\emptyset\varnothing

\renewcommand\geq\geqslant
\renewcommand\leq\leqslant

\newcommand\floor[1]{\left\lfloor#1\right\rfloor}
\newcommand\ceil[1]{\left\lceil#1\right\rceil}
% Boule
\DeclareDocumentCommand\B{}{\mathbf B}
% Probabilité, Espérance
\DeclareDocumentCommand\P{}{\mathbf P}
\DeclareDocumentCommand\E{}{\mathbf E}
% usual sets
\DeclareDocumentCommand\R{}{\mathbf R}
\DeclareDocumentCommand\N{}{\mathbf N}
\DeclareDocumentCommand\Z{}{\mathbf Z}
\DeclareDocumentCommand\Q{}{\mathbf Q}
\DeclareDocumentCommand\C{}{\mathbf C}
% fields
\DeclareDocumentCommand\K{}{\mathbf K}
\DeclareDocumentCommand\F{}{\mathbf F}
% logo d'indépendance
\newcommand\indep{\protect\mathpalette{\protect\indepT}{\perp}}
\def\indepT#1#2{\mathrel{\rlap{$#1#2$}\mkern2mu{#1#2}}}
% suit la loi de proba ...
\newcommand\suit{\hookrightarrow}
% vv from esvect
\renewcommand\vec\vv
\renewcommand\bar[1]{\overline{#1}}
% if inside \left...\right use \middle
\newcommand*{\suchthat}{\;\ifnum\currentgrouptype=16 \middle\fi|\;}
% scalar product
\NewDocumentCommand\scalar{mmO{\langle}O{\rangle}}{\ensuremath{\left#3 \, #1 \;\middle|\; #2 \,\right#4}}
\NewDocumentCommand\pscalar{mm}{\scalar{#1}{#2}[(][)]}

\newcommand{\hookdoubleheadrightarrow}{%
  \hookrightarrow\mathrel{\mspace{-15mu}}\rightarrow
}

% \xnrightarrow pour "ne tend pas vers"
\makeatletter
\newcommand*{\xnrightarrow}[2][]{%
  \ext@arrow 0359\nrightarrowfill@{#1}{#2}%
}
\newcommand*{\nrightarrowfill@}{%
    \narrowfill@\relbar\relbar\rightarrow{\not \relbar}
}
\newcommand*{\narrowfill@}[5]{%
  $\m@th\thickmuskip0mu\medmuskip\thickmuskip\thinmuskip\thickmuskip
  \relax#5#1\mkern-8mu%
  \cleaders\hbox{$#5\mkern-2mu#2\mkern-2mu$}\hfill
  \mkern-5mu %
  #4%
  \mkern-5mu %
  \cleaders\hbox{$#5\mkern-2mu#2\mkern-2mu$}\hfill
  \mkern-7mu#3$%
}
\makeatother

% underbrace pour une partie d'une matrice
\newcommand\undermat[2]{%
  \makebox[0pt][l]{$\smash{\underbrace{\phantom{%
    \begin{matrix}#2\end{matrix}}}_{\text{$#1$}}}$}#2}

% Matrice Jr = diag(Ir, 0)
\newcommand{\Jr}{\underbrace{%
\addstackgap[16pt]{%
\left(\begin{array}{ccc|ccc}%
        1        &    0   &  \cdots  & \cdots & \cdots &    0   \\
        0        & \ddots &  \ddots  &        &        & \vdots \\
     \vdots      & \ddots &     1    & \ddots &        & \vdots \\
  \hline \vdots  &        &  \ddots  &  0     & \ddots & \vdots \\
     \vdots      &        &          & \ddots & \ddots &    0   \\
\undermat{r}{0   & \cdots & \cdots } & \cdots &    0   &    0   \\
\end{array}
\right)}
}_{J_r}}
