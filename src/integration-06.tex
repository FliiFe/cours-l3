\ifsolo
    ~

    \vspace{1cm}

    \begin{center}
        \textbf{\LARGE Mesures complexes} \\[1em]
    \end{center}
    \tableofcontents
\else
    \chapter{Mesures complexes}

    \minitoc
\fi
\thispagestyle{empty}

\section{Définitions et propriétés de base}

\begin{dfn}
    Soit $(X, \mathcal  A)$ un espace mesurable. Une mesure complexe\index{mesure complexe} sur $(X, \mathcal  A)$ est une fonction $\nu: \mathcal  A \longrightarrow  \C$ telle que \begin{itemize}
        \item $\nu(\emptyset)=0$
        \item Si  $(A_i)_{i \in  I}$ est une famille dénombrable d'éléments de $\mathcal  A$ deux à deux disjoints, \[
                \nu \left( \bigcup_{i \in  I} A_i\right) = \sum_{i \in  I}\nu(A_i)
        \] 
        où la convergence est supposée indépendante du choix d'ordre sur $I$
    \end{itemize}
\end{dfn}

\begin{rem}
On admet que l'indépendance par permutation de la convergence entraine la convergence absolue.
\end{rem}

\begin{ex}
    Soit $(X, \mathcal  A, \mu)$ un espace mesuré avec $\mu$ une mesure usuelle positive, et soit  $f \in  \L^1_{\C}(X, \mathcal  A, \mu)$. Alors \[
        \nu(\mathcal  A)=\int_A f\diff \mu
    \] 
    est une mesure complexe (notée $f\diff \mu$). Clairement, $\nu(\emptyset)=0$ et  \[
        \nu \left(\bigcup_{n \in  \N}A_n\right)=\int_{X}f\diff \mu \1_{\bigcup A_i}=\int_{X}\sum_{n \in  \N}\underbrace{f\1_{A_n}}_{\mathclap{\text{dominée par }|f|\text{ intégrable }}}\diff \mu = \sum_{n \in  N}\int_{A_n} f\diff \mu=\sum_{n \in  \N}\nu(A_n)
    \] 
\end{ex}

\begin{dfn}
    Soit $\nu$ une mesure complexe. La mesure de variation totale de  $\nu$ \index{variation totale (mesure)} est la fonction  $|\nu|:\mathcal  A \longrightarrow \R_+\cup \left\{ \infty \right\} $ définie par \[
        |\nu|(A) = \sup \left\{ \sum_{i \in  N}|\nu(A_i)|, \quad  (A_i)_{i \in  \N} \text{ deux à deux disjoints, d'union } A\right\} 
    \] 
\end{dfn}

\begin{prop}
Si $\nu$ est une mesure complexe, alors  $|\nu|$ est une mesure.
\end{prop}

\begin{proof}

\end{proof}
