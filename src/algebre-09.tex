\ifsolo
    ~

    \vspace{1cm}

    \begin{center}
        \textbf{\LARGE Théorie des caractères} \\[1em]
    \end{center}
    \tableofcontents
\else
    \chapter{Théorie des caractères}

    \minitoc
\fi
\thispagestyle{empty}

\section{Définitions}

\begin{dfn}[Caractère d'une représentation]
    Soit $G$ un groupe fini et  $(V, \rho)$ une représentation linéaire de  $G$. Le caractère\index{caractère} de  $V$ est la fonction  $\chi_V: G \longrightarrow  \C$ défini par \[
        \forall  g \in  G, \qquad  \chi_V(g)=\tr(\rho(g))
    \] 
\end{dfn}

\begin{prop}
    Soient $(V, \rho)$ et  $(V', \rho')$ deux représentations isomorphes. Alors,  $\chi_V=\chi_{V'}$.
\end{prop}

\begin{proof}
    Toute représentation $V$ est  isomorphe à une représentation $G \longrightarrow \GL_n(\C)$. On peut donc supposer $\rho, \rho': G \longrightarrow \GL_n(\C)$. 
    \[
        \rho, \rho' \text{ isomorphes } \iff \exists P \in  \GL_n(\C), \forall  g \in  G, \rho'(g)=P\rho(g)P^{-1}
    \]
    Alors, \[
        \chi_{V'}(g)=\tr(\rho'(g))=\tr(\rho(g))=\chi_V(g)
    \]
\end{proof}

\begin{ex}
    Soit $V$ une représentation de degré  $1$. Elle est isomorphe à  $\chi : G \longrightarrow \C^\star$ (morphisme de groupes). Alors, $\chi_V(g)=\tr(\chi(g))=\chi(g)$ donc  $\chi_V=\chi$. On dit que  $\chi_V$ est un caractère linéaire de  $G$.
\end{ex}

\begin{prop}
    Soit $(V, \rho)$ une représentation de  $G$. Alors pour tout  $g \in  G$, $\rho(g)$ est diagonalisable et  $\Sp(\rho(g))\subset \U_n$, avec $n=\#G$.
\end{prop}

\begin{proof}
    $\rho^n(g)=\rho(g^n)=\rho(e)=\id$.
\end{proof}

\begin{prop}
    Soit $(V, \rho)$ représentation de  $G$. Alors,  \begin{enumerate}
        \item $\forall  g in G, |\chi_V(g)|\leq \dim V$ avec égalité si et seulement si $\rho(g)$ est une homothétie
        \item $\chi_V(g)=\dim V \iff  \rho(g)=\id$
        \item $\ker \rho = \left\{ g \in  G | \chi_V(g)=\dim V \right\} $
    \end{enumerate}
\end{prop}

\begin{proof}
Fixons $g \in  G$. \begin{enumerate}
    \item Soient $\lambda_1, \cdots , \lambda_n$ les valeurs propres de $\rho(g)$.  $\chi_V(g)=\tr(\rho(g))=\sum \lambda_i$ donc \[|\chi_V(g)|\leq \sum_{j=1}^n|\lambda_j|=n\]
        Le cas d'égalité correspond à l'égalité dans l'inégalité triangulaire.
    \item $\chi_V(g)=\dim V=n\lambda$ donc $ \lambda=1$ et $\chi_V(g)=\id$
    \item Immédiat
\end{enumerate}
\end{proof}

\begin{prop}
\begin{enumerate}
    \item $\chi_{V\oplus W}=\chi_V+\chi_W$ 
    \item $\chi_{V\otimes W}=\chi_V\times \chi_W$
    \item  $\chi_{\Hom(V, W)}=\bar{\chi_V}\chi_W$
    \item $\chi_{V^\star}=\bar{\chi_V}$
    \item Si $\varphi: G \longrightarrow  \C^\star$ est un caractère linéaire, alors $\chi_{V(\varphi)=\psi\chi_V}$
\end{enumerate}
\end{prop}

\begin{proof}

\end{proof}
