\ifsolo
    ~

    \vspace{1cm}

    \begin{center}
        \textbf{\LARGE Groupes} \\[1em]
    \end{center}
    \tableofcontents
\else
    \chapter{Groupes}

    \minitoc
\fi
\thispagestyle{empty}

\section{Définitions}

\begin{dfn}
    Un groupe\index{groupe} est un triplet $(G, \star, e)$ où  $G$ est un ensemble,  $\star$ est une loi de composition interne associative,  $e$ est neutre pour  $\star$ et tous les éléments de  $G$ possèdent un inverse pour  $\star$.
\end{dfn}

\begin{rem}
\begin{itemize}
    \item L'élément neutre est unique
    \item L'inverse est unique (on note $a ^{-1}$ l'inverse de $a$)
    \item On dit que $G$ est abélien ou commutatif si $ \forall  a,b \in  G, a\star b=b\star a$
    \item On appelle ordre de $G$ son cardinal
\end{itemize}
\end{rem}

\begin{dfn}
    $H\subset G$ est un sous-groupe\index{sous-groupe} de  $G$ si  $e \in  H$ et la restriction de la loi de $G$ à  $H$ est une loi de groupe
\end{dfn}

\begin{rem}
C'est équivalent à \begin{itemize}
    \item $e \in  H$
    \item $\forall  x,y \in  H, xy \in H$
    \item $\forall  x \in  H, x^{-1} \in  H$
\end{itemize}
Ces deux dernières conditions peuvent encore s'écrire $\forall  x,y \in  H, xy^{-1}\in H$.
\end{rem}

\begin{prop}
Les sous-groupes de $\Z$ sont de la forme $n\Z$ pour $n \in  \N$.
\end{prop}

\begin{proof}
    Les $n\Z$ sont des sous-groupes. Si $H\subset \Z$ est un sous-groupe non trivial alors $n=\inf (H\cap \N^\star) $ est tel que $n\Z\subset H$ et par division euclidienne, $n\Z=H$ (si $x=qn+r$ est la division de  $x \in  H$ par $n$, alors $r\in H$ et $0\leq r<n$ donc $r=0$ et  $x \in  n\Z$)
\end{proof}

\begin{prop}
Une intersection de sous-groupes est un sous-groupe.
\end{prop}

\begin{dfn}
Si $X\subset G$, le sous-groupe engendré par  $X$, noté  $\langle X\rangle$ est l'intersection de tous les sous-groupes contenant  $X$.
\end{dfn}

\begin{prop}
    $\langle X\rangle$ est l'ensemble des éléments de  $G$ qui s'écrivent  $x_1^{n_1}\cdots x_p^{n_p}$, $p \in \N, n_i \in  \Z, x_i \in  X$.
\end{prop}

\begin{dfn}[Groupe monogène]
    Un groupe est monogène\index{monogène} s'il existe $x \in  G$ tel que $\langle x\rangle =G$. Si  $G$ est fini et monogène, on dira qu'il est cyclique\index{cyclique}.
\end{dfn}

\section{Morphismes de groupes}

\begin{dfn}
    Soient $G, G'$ deux groupes. Un morphisme\index{morphisme} de groupes $G \longrightarrow  G'$ est une application $f:G\longrightarrow G'$ qui vérifie $f(e)=e'$ et  $f(x\cdot y)=f(x)\cdot f(y)$ pour tous  $x,y \in  G$.
\end{dfn}

\begin{rem}
    Automatiquement, $f(x^{-1})=f(x)^{-1}$
\end{rem}

\begin{dfn}
    On appelle noyau\index{noyau} du morphisme $f$ l'ensemble  $f^{-1} (\left\{ e' \right\} )$, et image l'ensemble  $f(G)$.
\end{dfn}

\begin{prop}
L'image et le noyau d'un morphismes sont des sous-groupes
\end{prop}

\begin{prop}
$f$ est injective  si et seulement si $\ker f= \left\{ e \right\} $, surjective si et seulement si $\im f=G'$.
\end{prop}

\begin{dfn}
    Un morphisme bijectif est appelé isomorphisme\index{isomorphisme}
\end{dfn}

\begin{prop}
L'inverse d'un isomorphisme est un morphisme
\end{prop}

\begin{prop}
Si $G$ est un groupe monogène, alors  $G\cong \Z$ ou $G\cong \frac{\Z}{n\Z}$ pour un $n \in  \N^\star$.
\end{prop}

\begin{proof}
    Si  $G=\langle x\rangle$, on pose  $f(n)=x^n$. C'est un morphisme de groupes surjectif. Si  $f$ est injectif, alors  $G\cong \Z$. Sinon $\ker f=n\Z$ pour un $n \in  \N^\star$, et $\frac{\Z}{n\Z}\cong G$
\end{proof}

\section{Classe définie par un sous-groupe}

\begin{defprop}
Si $H\subset G$ est un sous-groupe, alors on note  $x\sim y \iff  \exists h \in  H, x=yh$. C'est une relation d'équivalence (facile) et on note $G/H$ l'ensemble des classes d'équivalence pour cette relation. Ce sont les classes à gauche La classe de $x$ s'écrit  $xH$. 

De même, on peut définir les classes à droite.
\end{defprop}

\begin{dfn}
Si $G / H$ est fini, on dit que  $H$ est d'indice fini dans  $G$ et on note  \[
    [G:H]=\# \sfrac GH
\] 
l'indice de $H$ dans  $G$. Sinon, on dit que  $H$ est d'indice infini.
\end{dfn}

\begin{prop}
Toutes les classes à gauche modulo $H$ sont en bijection avec $H$.
\end{prop}

\begin{proof}
    $f_a:aH\longrightarrow H, x\longmapsto a^{-1}x$ convient (bijectif car $f_a\circ f_{a ^{-1}}=\id$ et $f_{a^{-1}}\circ f_a=\id$).
\end{proof}

\begin{cor}
    Si $G$ est fini, alors  \[\#G=[G:H]\cdot \#H\]
\end{cor}

\begin{thm}[Lagrange]
Si $G$ est fini et  $H$ est un sous-groupe de  $G$ alors  $\#H$ divise $\#G$
\end{thm}

\begin{cor}
Si $x \in  G$ et $G$ est fini alors l'ordre de  $x$ divise $\#G$
\end{cor}

\section{Sous-groupe distingué}

\begin{dfn}
$H$ est un sous-groupe distingué de  $G$  si et seulement si c'est un sous-groupe et $\forall  a \in G, aH=Ha$
\end{dfn}

\begin{prop}
Si $f$ est un morphisme de groupes, alors  $\ker f$ est distingué
\end{prop}
