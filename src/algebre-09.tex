\ifsolo
    ~

    \vspace{1cm}

    \begin{center}
        \textbf{\LARGE Théorie des caractères} \\[1em]
    \end{center}
    \tableofcontents
\else
    \chapter{Théorie des caractères}

    \minitoc
\fi
\thispagestyle{empty}

\section{Définitions et premières propriétés}

\begin{dfn}[Caractère d'une représentation]
    Soit $G$ un groupe fini et  $(V, \rho)$ une représentation linéaire de  $G$. Le caractère\index{caractère} de  $V$ est la fonction  $\chi_V: G \longrightarrow  \C$ défini par \[
        \forall  g \in  G, \qquad  \chi_V(g)=\tr(\rho(g))
    \] 
\end{dfn}

\begin{prop}
    Soient $(V, \rho)$ et  $(V', \rho')$ deux représentations isomorphes. Alors,  $\chi_V=\chi_{V'}$.
\end{prop}

\begin{proof}
    Toute représentation $V$ est  isomorphe à une représentation $G \longrightarrow \GL_n(\C)$. On peut donc supposer $\rho, \rho': G \longrightarrow \GL_n(\C)$. 
    \[
        \rho, \rho' \text{ isomorphes } \iff \exists P \in  \GL_n(\C), \forall  g \in  G, \rho'(g)=P\rho(g)P^{-1}
    \]
    Alors, \[
        \chi_{V'}(g)=\tr(\rho'(g))=\tr(\rho(g))=\chi_V(g)
    \]
\end{proof}

\begin{ex}
    Soit $V$ une représentation de degré  $1$. Elle est isomorphe à  $\chi : G \longrightarrow \C^\star$ (morphisme de groupes). Alors, $\chi_V(g)=\tr(\chi(g))=\chi(g)$ donc  $\chi_V=\chi$. On dit que  $\chi_V$ est un caractère linéaire de  $G$.
\end{ex}

\begin{prop}
    Soit $(V, \rho)$ une représentation de  $G$. Alors pour tout  $g \in  G$, $\rho(g)$ est diagonalisable et  $\Sp(\rho(g))\subset \U_n$, avec $n=\#G$.
\end{prop}

\begin{proof}
    $\rho^n(g)=\rho(g^n)=\rho(e)=\id$.
\end{proof}

\begin{prop}
    Soit $(V, \rho)$ représentation de  $G$. Alors,  \begin{enumerate}
        \item $\forall  g \in  G, |\chi_V(g)|\leq \dim V$ avec égalité si et seulement si $\rho(g)$ est une homothétie
        \item $\chi_V(g)=\dim V \iff  \rho(g)=\id$
        \item $\ker \rho = \left\{ g \in  G | \chi_V(g)=\dim V \right\} $
    \end{enumerate}
\end{prop}

\begin{proof}
Fixons $g \in  G$. \begin{enumerate}
    \item Soient $\lambda_1, \cdots , \lambda_n$ les valeurs propres de $\rho(g)$.  $\chi_V(g)=\tr(\rho(g))=\sum \lambda_i$ donc \[|\chi_V(g)|\leq \sum_{j=1}^n|\lambda_j|=n\]
        Le cas d'égalité correspond à l'égalité dans l'inégalité triangulaire.
    \item $\chi_V(g)=\dim V=n\lambda$ donc $ \lambda=1$ et $\chi_V(g)=\id$
    \item Immédiat
\end{enumerate}
\end{proof}

\begin{prop}
\begin{enumerate}
    \item $\chi_{V\oplus W}=\chi_V+\chi_W$ 
    \item $\chi_{V\otimes W}=\chi_V\times \chi_W$
    \item  $\chi_{\Hom(V, W)}=\bar{\chi_V}\chi_W$
    \item $\chi_{V^\star}=\bar{\chi_V}$
\item Si $\varphi: G \longrightarrow  \C^\star$ est un caractère linéaire, alors $\chi_{V(\varphi)}=\varphi\chi_V$
\end{enumerate}
\end{prop}

\begin{proof}
\begin{enumerate}
    \setcounter{enumi}{2}
\item Soit $g \in  G$, soit $(e_1, \cdots , e_m)$ base de $V$ et  $(f_1, \cdots , f_n)$ base de $W$, telles que  \[
\begin{dcases}
    \rho_V(g)e_i=\lambda_i e_i & \forall  i \in  \llbracket 1, m \rrbracket \\
    \rho_W(g)f_j=\mu_jf_j & \forall  j \in  \llbracket 1, n \rrbracket 
\end{dcases}
\] 
On va construire une base de diagonalisation de $\Hom_G(V, W)$. Pour $i \in  \llbracket 1,m \rrbracket , j \in  \llbracket 1,n \rrbracket $, on note \[
u_{i,j}: \begin{dcases}
    e_i\longmapsto f_j\\
    e_k \longmapsto 0 & \text{ lorsque }k\neq i
\end{dcases}
\]
Alors, comme $(g.u_{i,j})(v)=g.(u_{i,j}(g^{-1}.v))$ (cf. action de groupe de la représentation $\Hom(V, W)$) \[
    \rho_{\Hom(V, W)}(g)(u_{i,j})(e_i)=g.u_{i,j}(g^{-1}e_i)=\bar{\lambda} g.u_{i,j}(e_i)=\bar{\lambda_i}\mu_j f_j
\]
De même, si $k\neq i$,  $\rho_{\Hom(V, W)}(g)(u_{i,j})(e_k)=0$ donc \[\rho_{\Hom(V, W)}(g)(u_{i,j})=\bar{\lambda_i}\mu_ju_{i,j}\] et \[
\chi_{\Hom(V, W)}(g)=\sum_{i,j}\bar{\lambda_i}\mu_j=\bar{\chi_V(g)}\chi_W(g)
\] 
\end{enumerate}
\end{proof}

\begin{prop}
    Soit $(V, \rho)$ une représentation de  $G$. Alors  \[
        \pi_G=\frac1{\#G} \sum_{g \in  G}\rho(G) \in  \mathcal  L(V)
    \] 
    est un projecteur $G$-équivalent d'image  $V^G$
\end{prop}

\begin{proof}
    Soit  $h \in  G, x \in  V$. On a \[
        h.\pi_G(x)=\frac1{\#G}\sum_{g \in  G}hg.x=\frac1{\#G}\sum_{g \in  G}g.x=\pi_G(x) \in V^G
    \] 
    Si $x \in  V^G$, alors $\pi_G(x)=\frac{\#G}{\#G}x=x$. Puis, pour la  $G$-équivalence, on se donne  $h \in  H$ et $x \in  V$, et \[
        \pi_G(h.x)=\frac1{\#G}\sum_{g \in  G}g.(h.x)=\pi_G(x)=h.\pi_G(x)
    \] 
\end{proof}

\begin{cor}
    Pour toute représentation $(V, \rho)$ de $G$, on a \[
        \dim(V^G)=\frac1{\#G}\sum_{g \in  G}\chi_V(g)
    \] 
\end{cor}

\begin{proof}
La dimension recherchée est la trace de $\pi_G$
\end{proof}

\begin{ex}
$G\actg X$ fini.  $V_X=\C^X=\bigoplus_{x \in  X}\C e_x$. On note $\chi_X=\chi_{V_X}$. Pour $g \in  G$, \[
    \chi_X(g)=\tr(e_x\mapsto e_{g.x})=\# \Fix(g)
\] 
De plus \[
    \dim (V_X^G)=\frac1{\#G}\sum_{g \in  G}\chi_X(g)=\frac1{\#G}\sum_{g \in  G}\#\Fix(g)
\] 
La dimension de $V_X^G$ est en fait le nombre d'orbite de  $G\actg X$, donc on (re)trouve la formule de Burnside.
\end{ex}

\section{Notion d'espace hermitien}

\begin{dfn}
Un espace hermitien est un $ \C$-espace vectoriel $E$ de dimension finie muni d'une application  $\scalar{\;}{\;} : E^2 \longrightarrow \C$ \begin{itemize}
    \item Sesquilinéaire: \[
            \forall  x \in  E, \qquad  y \in  E \longmapsto \scalar{x}{y} \in  \C \quad \text{ est  }\C-\text{linéaire}
    \] 
    \[
            \forall  y \in  E, \qquad  x \in  E \longmapsto \scalar{x}{y} \in  \C \quad \text{ est  }\R-\text{linéaire}
    \] 
    et $\forall  a \in  \C, \scalar{ax}{y} = \bar{a} \scalar{x}{y} $ 
    \item Hermitienne: $\forall  x, y \in E, \quad  \bar{\scalar{x}{y} }= \scalar{y}{x} $. En particulier, $\scalar{x}{x} \in  \R$
    \item Définie positive: $\forall  x \in  E, \scalar{x}{x} \geq 0$ avec égalité si et seulement si $x=0$
\end{itemize}
\end{dfn}
